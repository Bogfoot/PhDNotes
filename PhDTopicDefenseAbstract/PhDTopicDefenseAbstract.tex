\documentclass{article}
\usepackage{lipsum} % For generating dummy text
\usepackage{titling} % For customizing title
\usepackage{abstract} % For abstract formatting

% Customizing title
\renewcommand{\maketitlehooka}{\centering}
\renewcommand{\maketitlehookb}{\vspace{-1.5em}}

% Customizing abstract
\renewcommand{\abstractnamefont}{\normalfont\large\bfseries}
\renewcommand{\abstracttextfont}{\normalfont\normalsize}

\title{Generating and teleporting entanglement for quantum networks}
\author{Adrian Udovičić}
\date{}

\begin{document}

\maketitle

\begin{abstract}
Entanglement is a key resource of future quantum technologies. For that reason, it will be essential to distribute 
it in quantum networks between many and possibly very distant communication parties. To this end, it is essential 
to generate the photons at a wavelength that is compatible with existing fiber network infrastructure. Such networks typically feature 
very low loss for photons in the O and C band (1310nm and 1550nm, respectively). To more 
efficiently use telecom fibers for many users, the available bandwidth is split into frequency windows to enable dense wavelength 
division multiplexing (DWDM). In the present thesis, we will implement a Sagnac source of entanglement for photons around 1560nm 
that is sufficiently narrowband for the entangled photons to fit into specific DWDM frequency channels. To generate the 
entangled photons, we will use a 50 mm long nonlinear crystal inside a Sagnac interferometer. We will first 
implement and characterize this source in our laboratory and later use it for demonstrating entanglement distribution over in an existing 
fiber network. The wavelength of the pump laser will be stabilized to an absorption line in a Rubidium gas cell. 
With our help, an identical source will be set up by partners at the Jozef Stefan Institute. This will allow us to demonstrate the teleportation of entanglement 
(entanglement swapping) by performing a Bell-state measurement on two entangled photons from those two independent and distant sources. This technique is a prerequisite for quantum 
repeaters, which will be essential to distribute entanglement over arbitrary long distances in future global quantum networks. In particular, even the low losses of photons in 
the C band will exponentially grow with the distance. This limits the efficient distribution of entanglement to distances of a few hundred kilometers.
The present work will not only feature the first realization of a source of entanglement in Slovenia but also the first realization of teleportation.\\
\textbf{Key words: Quantum Entanglement, Quantum Key Distribution, Entanglement Swapping}
\end{abstract}

\end{document}
