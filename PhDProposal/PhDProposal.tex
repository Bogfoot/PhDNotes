\documentclass{article}
%%%%%%%%%%%%%%%%%%%%%%%%%%%%%%%%%
% PACKAGE IMPORTS
%%%%%%%%%%%%%%%%%%%%%%%%%%%%%%%%%
\usepackage[framemethod=TikZ]{mdframed}
\usepackage{amsthm}
\usepackage{tikzsymbols}
\usepackage[tmargin=2.5cm,rmargin=2cm,lmargin=2.5cm,margin=0.85in,bmargin=2.5cm,footskip=.2in]{geometry}
\linespread{1.3}
\usepackage{amsmath,amsfonts,amsthm,amssymb,mathtools}
\usepackage[varbb]{newpxmath}
\usepackage{xfrac}
\usepackage[makeroom]{cancel}
\usepackage{bookmark}
\usepackage{enumitem}
\usepackage{hyperref,theoremref}
\hypersetup{
	pdftitle={Assignment},
	colorlinks=true, linkcolor=doc!90,
	bookmarksnumbered=true,
	bookmarksopen=true
}
\usepackage[most,many,breakable]{tcolorbox}
\usepackage{xcolor}
\usepackage{varwidth}
\usepackage{etoolbox}
\usepackage{nameref}
\usepackage{multicol,array}
\usepackage{tikz-cd}
\usepackage[ruled,vlined,linesnumbered]{algorithm2e}
\usepackage{comment} % enables the use of multi-line comments (\ifx \fi) 
\usepackage{import}
\usepackage{xifthen}
\usepackage{pdfpages}
\usepackage{transparent}
\usepackage[bottom]{footmisc}
\usepackage[utf8]{inputenc} % allow utf-8 input
\usepackage{braket}
\usepackage{wrapfig}			  %Kontekstom osjetljivo navođenje
\usepackage{csquotes}			  %Kontekstom osjetljivo navođenje
\usepackage{graphicx}       %Slike i slično
\usepackage[T1]{fontenc}    % use 8-bit T1 fonts
\usepackage{url}            % simple URL typesetting
\usepackage{booktabs}       % professional-quality tables
\usepackage{nicefrac}       % compact symbols for 1/2, etc.
\usepackage{float}
\usepackage{caption}
\usepackage{subcaption}
\usepackage{listings}
\usepackage[english,croatian]{babel}
\usepackage{titlesec}				%Za naslovnu stranicu

\counterwithin{figure}{section}
\urlstyle{same}
\numberwithin{equation}{section}
\hyphenation{pergamon} % riječ u argumentu (pergamon) se ne rastavlja s crticom; ne smije imat specijalna slova. č. ž... - rastavljam ju naredbom \- (npr. išče\-zava)
\setlength\parindent{0pt} % u novom paragrafu: indent=0
\setlength{\parskip}{10pt} % postavlja željeni vertikalni razmak između paragrafa
\setlength{\skip\footins}{2cm} % razmak između glavnog teksta i fusnota
\renewcommand{\thefootnote}{$\ddagger$} % želim dagger za oznaku fusnota
\newcommand{\HRule}{\rule{\linewidth}{0.4mm}} % nova naredba za horizontalne linije na naslovnoj stranici
\newlength{\mylen}
\setcounter{secnumdepth}{4}
\titleformat{\paragraph}
{\normalfont\normalsize\bfseries}{\theparagraph}{1em}{}


%%%%%%%%%%%%%%%%%%%%%%%%%%%%%%%%%%%%%%%%%%%
% TABLE OF CONTENTS
%%%%%%%%%%%%%%%%%%%%%%%%%%%%%%%%%%%%%%%%%%%

\usepackage{tikz}
\definecolor{doc}{RGB}{0,60,110}
\usepackage{titletoc}
\contentsmargin{0cm}
\titlecontents{chapter}[3.7pc]
{\addvspace{30pt}%
	\begin{tikzpicture}[remember picture, overlay]%
		\draw[fill=doc!60,draw=doc!60] (-7,-.1) rectangle (-0.9,.5);%
		\pgftext[left,x=-3.5cm,y=0.2cm]{\color{white}\Large\sc\bfseries Chapter\ \thecontentslabel};%
	\end{tikzpicture}\color{doc!60}\large\sc\bfseries}%
{}
{}
{\;\titlerule\;\large\sc\bfseries Page \thecontentspage
	\begin{tikzpicture}[remember picture, overlay]
		\draw[fill=doc!60,draw=doc!60] (2pt,0) rectangle (4,0.1pt);
	\end{tikzpicture}}%
\titlecontents{section}[3.7pc]
{\addvspace{2pt}}
{\contentslabel[\thecontentslabel]{2pc}}
{}
{\hfill\small \thecontentspage}
[]
\titlecontents*{subsection}[3.7pc]
{\addvspace{-1pt}\small}
{}
{}
{\ --- \small\thecontentspage}
[ \textbullet\ ][]

\makeatletter
\renewcommand{\tableofcontents}{%
	\chapter{%
	  \vspace{-20\p@}%
	  \begin{tikzpicture}[remember picture, overlay]%
		  \pgftext[right,x=15cm,y=0.2cm]{\color{doc!60}\Huge\sc\bfseries \contentsname};%
		  \draw[fill=doc!60,draw=doc!60] (13,-.75) rectangle (20,1);%
		  \clip (13,-.75) rectangle (20,1);
		  \pgftext[right,x=15cm,y=0.2cm]{\color{white}\Huge\sc\bfseries \contentsname};%
	  \end{tikzpicture}}%
	\@starttoc{toc}}
\makeatother


%%%%%%%%%%%%
%% KUTIJE %%
%%%%%%%%%%%%
%%%%%%%%%%%%%%%%%%%%%%%%%%%%%%
%Theorem
\newcounter{teorem}[section] \setcounter{teorem}{0}
\renewcommand{\theteorem}{\arabic{section}.\arabic{teorem}}
\newenvironment{teorem}[2][]{%
	\refstepcounter{teorem}%
	\ifstrempty{#1}%
	{\mdfsetup{%
			frametitle={%
					\tikz[baseline=(current bounding box.east),outer sep=0pt]
					\node[anchor=east,rectangle,fill=blue!20]
					{\strut Teorem~\thetheo};}}
	}%
	{\mdfsetup{%
			frametitle={%
					\tikz[baseline=(current bounding box.east),outer sep=0pt]
					\node[anchor=east,rectangle,fill=blue!20]
					{\strut Teorem~\thetheo:~#1};}}%
	}%
	\mdfsetup{innertopmargin=10pt,linecolor=blue!20,%
		linewidth=2pt,topline=true,%
		frametitleaboveskip=\dimexpr-\ht\strutbox\relax
	}
	\begin{mdframed}[]\relax%
		\label{#2}}{\end{mdframed}}
%%%%%%%%%%%%%%%%%%%%%%%%%%%%%%
%Proof
\newcounter{dokaz}[section]\setcounter{dokaz}{0}
\renewcommand{\thedokaz}{\arabic{section}.\arabic{dokaz}}
\newenvironment{dokaz}[2][]{%
	\refstepcounter{dokaz}%
	\ifstrempty{#1}%
	{\mdfsetup{%
			frametitle={%
					\tikz[baseline=(current bounding box.east),outer sep=0pt]
					\node[anchor=east,rectangle,fill=red!20]
					{\strut Dokaz~\theprf};}}
	}%
	{\mdfsetup{%
			frametitle={%
					\tikz[baseline=(current bounding box.east),outer sep=0pt]
					\node[anchor=east,rectangle,fill=red!20]
					{\strut Dokaz~\theprf:~#1};}}%
	}%
	\mdfsetup{innertopmargin=10pt,linecolor=red!20,%
		linewidth=2pt,topline=true,%
		frametitleaboveskip=\dimexpr-\ht\strutbox\relax
	}
	\begin{mdframed}[]\relax%
		\label{#2}}{\qed\end{mdframed}}
%%%%%%%%%%%%%%%%%%%%%%%%%%%%%%

\newcommand{\eps}{\epsilon}
\newcommand{\veps}{\varepsilon}
\newcommand{\ol}{\overline}
\newcommand{\ul}{\underline}
\newcommand{\wt}{\widetilde}
\newcommand{\wh}{\widehat}
\newcommand{\vocab}[1]{\textbf{\color{blue} #1}}
\providecommand{\half}{\frac{1}{2}}
\newcommand{\dang}{\measuredangle} %% Directed angle
\newcommand{\ray}[1]{\overrightarrow{#1}}
\newcommand{\seg}[1]{\overline{#1}}
\newcommand{\arc}[1]{\wideparen{#1}}
\DeclareMathOperator{\cis}{cis}
\DeclareMathOperator*{\lcm}{lcm}
\DeclareMathOperator*{\argmin}{arg min}
\DeclareMathOperator*{\argmax}{arg max}
\newcommand{\cycsum}{\sum_{\mathrm{cyc}}}
\newcommand{\symsum}{\sum_{\mathrm{sym}}}
\newcommand{\cycprod}{\prod_{\mathrm{cyc}}}
\newcommand{\symprod}{\prod_{\mathrm{sym}}}
\newcommand{\Qed}{\begin{flushright}\qed\end{flushright}}
\newcommand{\parinn}{\setlength{\parindent}{1cm}}
\newcommand{\parinf}{\setlength{\parindent}{0cm}}
% \newcommand{\norm}{\|\cdot\|}
\newcommand{\inorm}{\norm_{\infty}}
\newcommand{\opensets}{\{V_{\alpha}\}_{\alpha\in I}}
\newcommand{\oset}{V_{\alpha}}
\newcommand{\opset}[1]{V_{\alpha_{#1}}}
\newcommand{\lub}{\text{lub}}
\newcommand{\del}[2]{\frac{\partial #1}{\partial #2}}
\newcommand{\Del}[3]{\frac{\partial^{#1} #2}{\partial^{#1} #3}}
\newcommand{\deld}[2]{\dfrac{\partial #1}{\partial #2}}
\newcommand{\Deld}[3]{\dfrac{\partial^{#1} #2}{\partial^{#1} #3}}
\newcommand{\lm}{\lambda}
\newcommand{\uin}{\mathbin{\rotatebox[origin=c]{90}{$\in$}}}
\newcommand{\usubset}{\mathbin{\rotatebox[origin=c]{90}{$\subset$}}}
\newcommand{\lt}{\left}
\newcommand{\rt}{\right}
\newcommand{\bs}[1]{\boldsymbol{#1}}
\newcommand{\exs}{\exists}
\newcommand{\st}{\strut}
\newcommand{\dps}[1]{\displaystyle{#1}}

\newcommand{\sol}{\setlength{\parindent}{0cm}\textbf{\textit{Solution:}}\setlength{\parindent}{1cm} }
\newcommand{\solve}[1]{\setlength{\parindent}{0cm}\textbf{\textit{Solution: }}\setlength{\parindent}{1cm}#1 \Qed}

%From M275 "Topology" at SJSU
\newcommand{\id}{\mathrm{id}}
\newcommand{\taking}[1]{\xrightarrow{#1}}
\newcommand{\inv}{^{-1}}

%From M170 "Introduction to Graph Theory" at SJSU
\DeclareMathOperator{\diam}{diam}
\DeclareMathOperator{\ord}{ord}
\newcommand{\defeq}{\overset{\mathrm{def}}{=}}

%From the USAMO .tex files
\newcommand{\ts}{\textsuperscript}
\newcommand{\dg}{^\circ}
\newcommand{\ii}{\item}

% % From Math 55 and Math 145 at Harvard
% \newenvironment{subproof}[1][Proof]{%
% \begin{proof}[#1] \renewcommand{\qedsymbol}{$\blacksquare$}}%
% {\end{proof}}

\newcommand{\liff}{\leftrightarrow}
\newcommand{\lthen}{\rightarrow}
\newcommand{\opname}{\operatorname}
\newcommand{\surjto}{\twoheadrightarrow}
\newcommand{\injto}{\hookrightarrow}
\newcommand{\On}{\mathrm{On}} % ordinals
\DeclareMathOperator{\img}{im} % Image
\DeclareMathOperator{\Img}{Im} % Image
\DeclareMathOperator{\coker}{coker} % Cokernel
\DeclareMathOperator{\Coker}{Coker} % Cokernel
\DeclareMathOperator{\Ker}{Ker} % Kernel
\DeclareMathOperator{\rank}{rank}
\DeclareMathOperator{\Spec}{Spec} % spectrum
\DeclareMathOperator{\Tr}{Tr} % trace
\DeclareMathOperator{\pr}{pr} % projection
\DeclareMathOperator{\ext}{ext} % extension
\DeclareMathOperator{\pred}{pred} % predecessor
\DeclareMathOperator{\dom}{dom} % domain
\DeclareMathOperator{\ran}{ran} % range
\DeclareMathOperator{\Hom}{Hom} % homomorphism
\DeclareMathOperator{\Mor}{Mor} % morphisms
\DeclareMathOperator{\End}{End} % endomorphism

% Things Lie
\newcommand{\kb}{\mathfrak b}
\newcommand{\kg}{\mathfrak g}
\newcommand{\kh}{\mathfrak h}
\newcommand{\kn}{\mathfrak n}
\newcommand{\ku}{\mathfrak u}
\newcommand{\kz}{\mathfrak z}
\DeclareMathOperator{\Ext}{Ext} % Ext functor
\DeclareMathOperator{\Tor}{Tor} % Tor functor
\newcommand{\gl}{\opname{\mathfrak{gl}}} % frak gl group
\renewcommand{\sl}{\opname{\mathfrak{sl}}} % frak sl group chktex 6

% More script letters etc.
\newcommand{\SA}{\mathcal A}
\newcommand{\SB}{\mathcal B}
\newcommand{\SC}{\mathcal C}
\newcommand{\SF}{\mathcal F}
\newcommand{\SG}{\mathcal G}
\newcommand{\SH}{\mathcal H}
\newcommand{\OO}{\mathcal O}

\newcommand{\SCA}{\mathscr A}
\newcommand{\SCB}{\mathscr B}
\newcommand{\SCC}{\mathscr C}
\newcommand{\SCD}{\mathscr D}
\newcommand{\SCE}{\mathscr E}
\newcommand{\SCF}{\mathscr F}
\newcommand{\SCG}{\mathscr G}
\newcommand{\SCH}{\mathscr H}

% Mathfrak primes
\newcommand{\km}{\mathfrak m}
\newcommand{\kp}{\mathfrak p}
\newcommand{\kq}{\mathfrak q}

% number sets
\newcommand{\RR}[1][]{\ensuremath{\ifstrempty{#1}{\mathbb{R}}{\mathbb{R}^{#1}}}}
\newcommand{\NN}[1][]{\ensuremath{\ifstrempty{#1}{\mathbb{N}}{\mathbb{N}^{#1}}}}
\newcommand{\ZZ}[1][]{\ensuremath{\ifstrempty{#1}{\mathbb{Z}}{\mathbb{Z}^{#1}}}}
\newcommand{\QQ}[1][]{\ensuremath{\ifstrempty{#1}{\mathbb{Q}}{\mathbb{Q}^{#1}}}}
\newcommand{\CC}[1][]{\ensuremath{\ifstrempty{#1}{\mathbb{C}}{\mathbb{C}^{#1}}}}
\newcommand{\PP}[1][]{\ensuremath{\ifstrempty{#1}{\mathbb{P}}{\mathbb{P}^{#1}}}}
\newcommand{\HH}[1][]{\ensuremath{\ifstrempty{#1}{\mathbb{H}}{\mathbb{H}^{#1}}}}
\newcommand{\FF}[1][]{\ensuremath{\ifstrempty{#1}{\mathbb{F}}{\mathbb{F}^{#1}}}}
% expected value
\newcommand{\EE}{\ensuremath{\mathbb{E}}}
\newcommand{\charin}{\text{ char }}
\DeclareMathOperator{\sign}{sign}
\DeclareMathOperator{\Aut}{Aut}
\DeclareMathOperator{\Inn}{Inn}
\DeclareMathOperator{\Syl}{Syl}
\DeclareMathOperator{\Gal}{Gal}
\DeclareMathOperator{\GL}{GL} % General linear group
\DeclareMathOperator{\SL}{SL} % Special linear group

%---------------------------------------
% BlackBoard Math Fonts :-
%---------------------------------------

%Captital Letters
\newcommand{\bbA}{\mathbb{A}}	\newcommand{\bbB}{\mathbb{B}}
\newcommand{\bbC}{\mathbb{C}}	\newcommand{\bbD}{\mathbb{D}}
\newcommand{\bbE}{\mathbb{E}}	\newcommand{\bbF}{\mathbb{F}}
\newcommand{\bbG}{\mathbb{G}}	\newcommand{\bbH}{\mathbb{H}}
\newcommand{\bbI}{\mathbb{I}}	\newcommand{\bbJ}{\mathbb{J}}
\newcommand{\bbK}{\mathbb{K}}	\newcommand{\bbL}{\mathbb{L}}
\newcommand{\bbM}{\mathbb{M}}	\newcommand{\bbN}{\mathbb{N}}
\newcommand{\bbO}{\mathbb{O}}	\newcommand{\bbP}{\mathbb{P}}
\newcommand{\bbQ}{\mathbb{Q}}	\newcommand{\bbR}{\mathbb{R}}
\newcommand{\bbS}{\mathbb{S}}	\newcommand{\bbT}{\mathbb{T}}
\newcommand{\bbU}{\mathbb{U}}	\newcommand{\bbV}{\mathbb{V}}
\newcommand{\bbW}{\mathbb{W}}	\newcommand{\bbX}{\mathbb{X}}
\newcommand{\bbY}{\mathbb{Y}}	\newcommand{\bbZ}{\mathbb{Z}}

%---------------------------------------
% MathCal Fonts :-
%---------------------------------------

%Captital Letters
\newcommand{\mcA}{\mathcal{A}}	\newcommand{\mcB}{\mathcal{B}}
\newcommand{\mcC}{\mathcal{C}}	\newcommand{\mcD}{\mathcal{D}}
\newcommand{\mcE}{\mathcal{E}}	\newcommand{\mcF}{\mathcal{F}}
\newcommand{\mcG}{\mathcal{G}}	\newcommand{\mcH}{\mathcal{H}}
\newcommand{\mcI}{\mathcal{I}}	\newcommand{\mcJ}{\mathcal{J}}
\newcommand{\mcK}{\mathcal{K}}	\newcommand{\mcL}{\mathcal{L}}
\newcommand{\mcM}{\mathcal{M}}	\newcommand{\mcN}{\mathcal{N}}
\newcommand{\mcO}{\mathcal{O}}	\newcommand{\mcP}{\mathcal{P}}
\newcommand{\mcQ}{\mathcal{Q}}	\newcommand{\mcR}{\mathcal{R}}
\newcommand{\mcS}{\mathcal{S}}	\newcommand{\mcT}{\mathcal{T}}
\newcommand{\mcU}{\mathcal{U}}	\newcommand{\mcV}{\mathcal{V}}
\newcommand{\mcW}{\mathcal{W}}	\newcommand{\mcX}{\mathcal{X}}
\newcommand{\mcY}{\mathcal{Y}}	\newcommand{\mcZ}{\mathcal{Z}}



%---------------------------------------
% Bold Math Fonts :-
%---------------------------------------

%Captital Letters
\newcommand{\bmA}{\boldsymbol{A}}	\newcommand{\bmB}{\boldsymbol{B}}
\newcommand{\bmC}{\boldsymbol{C}}	\newcommand{\bmD}{\boldsymbol{D}}
\newcommand{\bmE}{\boldsymbol{E}}	\newcommand{\bmF}{\boldsymbol{F}}
\newcommand{\bmG}{\boldsymbol{G}}	\newcommand{\bmH}{\boldsymbol{H}}
\newcommand{\bmI}{\boldsymbol{I}}	\newcommand{\bmJ}{\boldsymbol{J}}
\newcommand{\bmK}{\boldsymbol{K}}	\newcommand{\bmL}{\boldsymbol{L}}
\newcommand{\bmM}{\boldsymbol{M}}	\newcommand{\bmN}{\boldsymbol{N}}
\newcommand{\bmO}{\boldsymbol{O}}	\newcommand{\bmP}{\boldsymbol{P}}
\newcommand{\bmQ}{\boldsymbol{Q}}	\newcommand{\bmR}{\boldsymbol{R}}
\newcommand{\bmS}{\boldsymbol{S}}	\newcommand{\bmT}{\boldsymbol{T}}
\newcommand{\bmU}{\boldsymbol{U}}	\newcommand{\bmV}{\boldsymbol{V}}
\newcommand{\bmW}{\boldsymbol{W}}	\newcommand{\bmX}{\boldsymbol{X}}
\newcommand{\bmY}{\boldsymbol{Y}}	\newcommand{\bmZ}{\boldsymbol{Z}}
%Small Letters
\newcommand{\bma}{\boldsymbol{a}}	\newcommand{\bmb}{\boldsymbol{b}}
\newcommand{\bmc}{\boldsymbol{c}}	\newcommand{\bmd}{\boldsymbol{d}}
\newcommand{\bme}{\boldsymbol{e}}	\newcommand{\bmf}{\boldsymbol{f}}
\newcommand{\bmg}{\boldsymbol{g}}	\newcommand{\bmh}{\boldsymbol{h}}
\newcommand{\bmi}{\boldsymbol{i}}	\newcommand{\bmj}{\boldsymbol{j}}
\newcommand{\bmk}{\boldsymbol{k}}	\newcommand{\bml}{\boldsymbol{l}}
\newcommand{\bmm}{\boldsymbol{m}}	\newcommand{\bmn}{\boldsymbol{n}}
\newcommand{\bmo}{\boldsymbol{o}}	\newcommand{\bmp}{\boldsymbol{p}}
\newcommand{\bmq}{\boldsymbol{q}}	\newcommand{\bmr}{\boldsymbol{r}}
\newcommand{\bms}{\boldsymbol{s}}	\newcommand{\bmt}{\boldsymbol{t}}
\newcommand{\bmu}{\boldsymbol{u}}	\newcommand{\bmv}{\boldsymbol{v}}
\newcommand{\bmw}{\boldsymbol{w}}	\newcommand{\bmx}{\boldsymbol{x}}
\newcommand{\bmy}{\boldsymbol{y}}	\newcommand{\bmz}{\boldsymbol{z}}

%---------------------------------------
% Scr Math Fonts :-
%---------------------------------------

\newcommand{\sA}{{\mathscr{A}}}   \newcommand{\sB}{{\mathscr{B}}}
\newcommand{\sC}{{\mathscr{C}}}   \newcommand{\sD}{{\mathscr{D}}}
\newcommand{\sE}{{\mathscr{E}}}   \newcommand{\sF}{{\mathscr{F}}}
\newcommand{\sG}{{\mathscr{G}}}   \newcommand{\sH}{{\mathscr{H}}}
\newcommand{\sI}{{\mathscr{I}}}   \newcommand{\sJ}{{\mathscr{J}}}
\newcommand{\sK}{{\mathscr{K}}}   \newcommand{\sL}{{\mathscr{L}}}
\newcommand{\sM}{{\mathscr{M}}}   \newcommand{\sN}{{\mathscr{N}}}
\newcommand{\sO}{{\mathscr{O}}}   \newcommand{\sP}{{\mathscr{P}}}
\newcommand{\sQ}{{\mathscr{Q}}}   \newcommand{\sR}{{\mathscr{R}}}
\newcommand{\sS}{{\mathscr{S}}}   \newcommand{\sT}{{\mathscr{T}}}
\newcommand{\sU}{{\mathscr{U}}}   \newcommand{\sV}{{\mathscr{V}}}
\newcommand{\sW}{{\mathscr{W}}}   \newcommand{\sX}{{\mathscr{X}}}
\newcommand{\sY}{{\mathscr{Y}}}   \newcommand{\sZ}{{\mathscr{Z}}}


%---------------------------------------
% Math Fraktur Font
%---------------------------------------

%Captital Letters
\newcommand{\mfA}{\mathfrak{A}}	\newcommand{\mfB}{\mathfrak{B}}
\newcommand{\mfC}{\mathfrak{C}}	\newcommand{\mfD}{\mathfrak{D}}
\newcommand{\mfE}{\mathfrak{E}}	\newcommand{\mfF}{\mathfrak{F}}
\newcommand{\mfG}{\mathfrak{G}}	\newcommand{\mfH}{\mathfrak{H}}
\newcommand{\mfI}{\mathfrak{I}}	\newcommand{\mfJ}{\mathfrak{J}}
\newcommand{\mfK}{\mathfrak{K}}	\newcommand{\mfL}{\mathfrak{L}}
\newcommand{\mfM}{\mathfrak{M}}	\newcommand{\mfN}{\mathfrak{N}}
\newcommand{\mfO}{\mathfrak{O}}	\newcommand{\mfP}{\mathfrak{P}}
\newcommand{\mfQ}{\mathfrak{Q}}	\newcommand{\mfR}{\mathfrak{R}}
\newcommand{\mfS}{\mathfrak{S}}	\newcommand{\mfT}{\mathfrak{T}}
\newcommand{\mfU}{\mathfrak{U}}	\newcommand{\mfV}{\mathfrak{V}}
\newcommand{\mfW}{\mathfrak{W}}	\newcommand{\mfX}{\mathfrak{X}}
\newcommand{\mfY}{\mathfrak{Y}}	\newcommand{\mfZ}{\mathfrak{Z}}
%Small Letters
\newcommand{\mfa}{\mathfrak{a}}	\newcommand{\mfb}{\mathfrak{b}}
\newcommand{\mfc}{\mathfrak{c}}	\newcommand{\mfd}{\mathfrak{d}}
\newcommand{\mfe}{\mathfrak{e}}	\newcommand{\mff}{\mathfrak{f}}
\newcommand{\mfg}{\mathfrak{g}}	\newcommand{\mfh}{\mathfrak{h}}
\newcommand{\mfi}{\mathfrak{i}}	\newcommand{\mfj}{\mathfrak{j}}
\newcommand{\mfk}{\mathfrak{k}}	\newcommand{\mfl}{\mathfrak{l}}
\newcommand{\mfm}{\mathfrak{m}}	\newcommand{\mfn}{\mathfrak{n}}
\newcommand{\mfo}{\mathfrak{o}}	\newcommand{\mfp}{\mathfrak{p}}
\newcommand{\mfq}{\mathfrak{q}}	\newcommand{\mfr}{\mathfrak{r}}
\newcommand{\mfs}{\mathfrak{s}}	\newcommand{\mft}{\mathfrak{t}}
\newcommand{\mfu}{\mathfrak{u}}	\newcommand{\mfv}{\mathfrak{v}}
\newcommand{\mfw}{\mathfrak{w}}	\newcommand{\mfx}{\mathfrak{x}}
\newcommand{\mfy}{\mathfrak{y}}	\newcommand{\mfz}{\mathfrak{z}}


% Customizing title
\renewcommand{\maketitlehooka}{\centering}
\renewcommand{\maketitlehookb}{\vspace{-1.5em}}

% Customizing abstract
\renewcommand{\abstractnamefont}{\normalfont\large\bfseries}
\renewcommand{\abstracttextfont}{\normalfont\normalsize}
\graphicspath{{./Images/}} % Where to take images from
%%%%%%%%%%%%%%%%%%%%%%%%%% Define some useful colors %%%%%%%%%%%%%%%%%%%%%%%%%%
%%%%%%%%%%%%%%%%%%%%%%%%%%%%%%%%%%%%%%%%%%%%%%%%%%%%%%%%%%%%%%%%%%%%%%%%%%%%%%%
\begin{document}
%%%%%%%%%%%%%%%%%%%%%%%%%%%%%%% Title & Author %%%%%%%%%%%%%%%%%%%%%%%%%%%%%%%%
\begin{titlepage}
	\begin{center}
		{\LARGE Univerza \textit{v Ljubljani}} \\[0.1cm]
		{\LARGE Fakulteta za \textit{matematiko in fiziko}} \\[1cm]
		\begin{figure}[h]
			\centering
			\def\svgwidth{0.5\columnwidth}
			\input{./Images/fmf-logo-home-sl.pdf_tex}
			\label{fig:UL logo}
		\end{figure}
		{\large \textbf{Adrian Udovičić, mag. phys.}}\\[0.1cm]
		{\sc DISPOZICIJA DOKTORSKE DISERTACIJE}\\
		\vspace{1cm}

		{\bf \Large Ustvarjanje in teleportiranje prepletenosti za kvantna omrežja}\\
		\vspace{1cm}

		{\bf \Large Generating and teleporting entanglement for quantum networks}\\


		\vspace{3cm}

		{\large ADVISER: Rainer O. Kaltenbaek, assoc. prof. dr.\\

			\vspace{2cm}
			Znanstveno področje: Fizika\\
			\vspace{1cm}
			Ljubljana, 2025\date{}}
	\end{center}
\end{titlepage}
%%%%%%%%%%%%%%%%%%%%%%%%%%%%%%%%%%%%%%%%%%%%%%%%%%%%%%%%%%%%%%%%%%%%%%%%%%%%%%%
%-----------------------------------------------------------------------------------------------
%         APPLICATION
%----------------------------------------------------------------------------------------------
\clearpage
\pagestyle{plain}
\pagenumbering{roman}

\vspace{1cm}

\noindent Senat UL FMF\\
Fakulteta za matematiko in fiziko\\
Jadranska ulica 19\\
1000 Ljubljana\\

\vspace{.5cm}

\begin{center}
	\textbf{Zadeva: Prošnja za odobritev teme doktorske disertacije}
\end{center}
\vspace{.5cm}

Spoštovani člani odbora,
\vspace{1cm}

Pišem vam, da bi se uradno prijavil za temo svoje doktorske disertacije. Sem na doktorskem študijskem programu Fizika.
Moje raziskave na področju kvantne komunikacije potekajo pod mentorstvom izrednega profesorja Dr. Rainerja Oliverja Kaltenbaeka,
Osredotočam se na razvoj visoko zmogljivega vira prepletenih fotonov. Naslov disertacije je „Generiranje in teleportiranje prepletenosti za kvantna omrežja“.
Cilj moje študije je zasnovati vir, ki je dovolj širokopasovni, da lahko hkrati služi več odjemalcem,
s čimer bi izboljšali praktičnost in razširljivost kvantnih omrežij.

Svoje raziskave želim nadaljevati na Univerzi v Ljubljani,
Fakulteti za matematiko in fiziko, in se veselim te priložnosti,
da bom lahko prispeval k temu vznemirljivemu področju.

Zahvaljujem se vam za vaš čas in razmislek. Veselim se vašega odgovora.



\vspace{1cm}
S spoštovanjem,\\
Adrian Udovičić\\
adrian.udovicic@fmf.uni-lj.si\\
Ulica Ante Kovačića 12A, 23000 Zadar, Hrvaška\\
Fakulteta za matematiko in fiziko, Oddelek za fiziko

%-----------------------------------------------------------------------------------------------
%         Življenjepis or CV or Biography
%----------------------------------------------------------------------------------------------

\clearpage
\pagestyle{plain}

\vspace{1cm}

\begin{center}
	\textbf{Short CV}
\end{center}
I am a PhD candidate in physics at the University of Ljubljana, Faculty of Mathematics and Physics (FMF),
working in the Laboratory for Quantum Optics under the supervision of Assoc. Prof. Dr. Rainer O. Kaltenbaek.
My research focuses on developing a high-yield, broadband source of entangled photons for quantum communication.
I received a Master’s degree in physics from the University of Rijeka,
where I conducted research on transient signals in dark matter detection,
and a Bachelor’s degree in physics, with a thesis on spectral analysis of AGN Markarian 421 in the very high-energy gamma region.
I have experience in quantum and nonlinear optics from my work at my supervisors laboratory at FMF.

\vspace{1cm}

\begin{center}
	\textbf{Kratki življenjepis}
\end{center}
Sem doktorski kandidat fizike na Fakulteti za matematiko in fiziko Univerze v Ljubljani (FMF),
kjer delam v Laboratoriju za kvantno optiko pod mentorstvom doc. dr. Rainerja O. Kaltenbaeka.
Moje raziskave se osredotočajo na razvoj visokozmogljivega, širokopasovnega vira prepletenih fotonov za kvantno komunikacijo.
Na Univerzi na Reki sem magistriral iz fizike, kjer sem raziskoval prehodne signale pri odkrivanju temne snovi,
in diplomiral iz fizike z nalogo o spektralni analizi AGN Markarian 421 v območju zelo visokih energij gama.
Izkušnje na področju kvantne in nelinearne optike imam iz dela v laboratoriju svojega mentorja na FMF.



%-----------------------------------------------------------------------------------------------
%         Predlog za odobritev pisanja doktorske disertacije v angleščini
%----------------------------------------------------------------------------------------------

\clearpage
\pagestyle{plain}
\setcounter{page}{4}
\begin{center}
	\textbf{Application for writing a doctoral dissertation in English}\\
\end{center}

\noindent Senat UL FMF\\
Faculty of mathematics and physics\\
Jadranska ulica 19\\
1000 Ljubljana\\

\vspace{1cm} % Adjust spacing below the title if needed
Dear Committee Members,
\vspace{1cm}

I hope this message finds you well. I am writing to formally request permission to write my PhD thesis in English.
As an international student and non-native speaker of Slovenian, I believe that completing my thesis in English would be beneficial for both academic and practical reasons.
Firstly, English is the primary language in my field of study, and the majority of relevant literature, research articles, and publications are available in English.
Writing my thesis in English would enable me to engage more directly with this body of work and ensure that my research is positioned within the global academic discourse.
Secondly, my supervisor, Assoc. Prof. Dr. Rainer Oliver Kaltenbaek, who is also a non-native speaker of Slovenian, has advised that conducting and evaluating the research in
English would facilitate clearer communication and collaboration throughout the thesis process. Furthermore, writing in English would allow for smoother peer review
and potential publication in international journals.
Lastly, most, if not all of the literature that I am using in my doctoral studies are in English and I believe it would slow down my progress to translate
all of the terminoligy and nomenclature to Slovenian.
I greatly appreciate your understanding and consideration of this request. I am confident that writing my thesis in English will enhance its academic
impact and contribute positively to my development as a researcher. Please let me know if further clarification or documentation is required to support this appeal.
Thank you for your time and attention. I look forward to your response.

\vspace{1cm}
Yours sincerely,\\
Adrian Udovičić\\
Faculty of Mathematics and Physics, Department of Physics\\
adrian.udovicic@fmf.uni-lj.si

%-----------------------------------------------------------------------------------------------
%         Ph.D. thesis disposition (in English)
%----------------------------------------------------------------------------------------------

\clearpage
\pagestyle{plain}

\begin{center}
	\textbf{\Large Disposition of doctoral dissertation (in English)}
\end{center}

\pagenumbering{arabic}

\section{Description of the immediate research area and its problems}

In the rapidly advancing fields of quantum communication and quantum computing, sensing, and simulators
the efficient transfer of secure quantum information is of great importance.
A key quantum resource is entanglement, which enables experiments such as quantum teleportation and entanglement swapping.
Conducting these experiments over long distances through optical fibers presents significant challenges due to transmission losses.
To mitigate this, photons must be generated at wavelengths compatible with existing fiber-optic networks,
particularly in the C band (near-infrared) where transmission losses are minimal.
One also has to use standard Wavelength Division Multiplexing due to high traffic of such networks.
% These advances contribute to the broader goal of realizing quantum networks,
% which require robust capabilities for generating and characterizing aforementioned quantum processes.
The requirements to the advancement of the field would be robust generation and characterization of the aforementioned quantum processes.
Future quantum networks may rely on the distribution of entanglement and the teleportation of quantum states across network nodes \cite{Kimble_2008}.
Additionally, in areas where fiber connections are sparse,
free-space communication methods are being explored for applications in metropolitan areas \cite{Kržić_et_al_2023}. Sometimes,
these solutions are more practical and cost effective.
\par While these technologies are essential for local and metropolitan quantum networks, scaling to a global level requires overcoming
the inherent limitations of photon loss over long distances. This is where quantum repeaters and high-yield entanglement sources
will play a critical role for the future global quantum internet.
In particular, quantum repeaters rely on splitting distances into shorter segments. High-yield entanglement sources will distribute entanglement over these shorter segments.
Quantum memories can be used to store that entanglement before it is teleported along adjoining segments.
This way quantum repeaters can enable the distribution of entanglement over arbitrarily long distances by overcoming exponential loss in fibers.
With no such in between nodes, entanglement distribution via fiber networks would be limited to distances of only a few hundred kilometers.
\par This work seeks to establish the technical bedrock for future scalable quantum networks. These efforts do not exist in isolation;
they directly feed into the broader goal of transforming theoretical quantum advantages into real-world systems.
I aim to achieve not only the first realization of a high-yield polarization entanglement source at non-degenerate
frequencies in Slovenia but also to demonstrate quantum teleportation and entanglement swapping using
continuous wave pump lasers.
In the future we will aim to use a pulsed laser to increased the efficiency of entanglement swapping.
% Ongoing research efforts aim to bridge the gap between theoretical advancements and practical applications,
% driving the quest for more efficient and accessible quantum systems that could transform various sectors.
% As quantum technologies mature, they are poised to redefine our understanding of information
% processing and secure communications in the quantum era.
\par\textbf{Key words: Quantum Entanglement, Quantum Communication, Entanglement Swapping}

\section{Overview of related research and relevant literature}
Quantum entanglement sources are pivotal components in the field of quantum technologies, enabling the generation of entangled states
that are essential for a range of applications, including distributed quantum computing, quantum cryptography, and quantum communication.
These sources can produce pairs of entangled photons through various techniques such as micro-ring resonators \cite{Wakabayashi_2015},
quantum dot \cite{Rota_2024}, and by far the most widely used method being Spontaneous Parametric Down-Conversion \cite{jesseSPDC} (SPDC).
The ability to create reliable and efficient entangled states has garnered significant interest due to their
implications for advancing quantum technologies and facilitating secure information transfer across long distances.
\par After one of the first \cite{Kwiat_1995} demonstrations of a high-intensity polarization entangled source was realized,
research in this field advanced quickly. Some notable mentions using a similar design with a Sagnac loop
\cite{Neumann_Buchner_Bulla_Bohmann_Ursin_2022_CW,Chen_Ecker_Wengerowsky_Bulla_Joshi_Steinlechner_Ursin_2018_CW}
in which the entanglement is generated due to the ambiguity of the origin of the photons overlapped at a Polarization Beam-Splitter (PBS).
There are also many linear, or single-pass designs such as \cite{Lee_Kim_Cha_Moon_2016}
where the entanglement is a result of the photon pairs originating from two overlapping cones of SPDC light.
\par Important measures of a sources performance are its brightness, bandwidth, and heralding \cite{Bennink_2010,Ljunggren_Tengner_Marsden_Pelton_2006}.
The brightness is a measure of how many photon pairs are being produced, bandwidth corresponds to how well defined they are in frequency,
as this is a limiting factor for certain interference measurements like
Hong, Ou, and Mandel (HOM) \cite{Hong_Ou_Mandel_1987}, and also for coupling to quantum memories,
and the heralding being the probability, when measuring two photon correlations, of finding a correlated photon when detecting the 1st one.
Brightness and heralding should be as high as possible to mitigate loss in fiber for fiber based networks,
reduce preprocessing load, and the bandwidth to be as narrow as needed for efficient coupling to quantum devices such as quantum memories or repeaters,
or for measurements as HOM, which will need to be performed for a full characterisation of the source.
Afterwards, one can also perform a Hanbury-Brown and Twiss \cite{Brown_Twiss_1954} experiment to check the line width of the filtered photons.
\par It was shown \cite{Zukowski_1993} that two independent entangled photon sources can be used to implement
"event-ready" Einstein-Podolsky-Rosen (EPR) experiments \cite{EPR1935}, a scheme now known as entanglement swapping.
This process allows two particles that have never directly interacted to become entangled by performing a
joint measurement on their respective partners, effectively transferring entanglement from one pair to another.
Among the first such realizations of entanglement swapping was performed using pulsed lasers \cite{Kaltenbaek_2006}.
One of the first \cite{Halder_Beveratos_Jorel_Zbinden_Simon_Scarani_Gisin_2007} experimental demonstrations of entanglement
swapping by use of Continuous Wave (CW) lasers has shown that it is entirely possible to not use pulsed lasers for this purpose, but with less efficiency.
Afterwards, there were very few reports
on this. Somewhat recently, two \cite{Samara_Maring_Martin_Raja_Kippenberg_Zbinden_Thew_2021,Tsujimoto_Tanaka_Iwasaki_Ikuta_Miki_Yamashita_Terai_Yamamoto_Koashi_Imoto_2018}
interesting papers came out showing entanglement swapping using a micro-ring resonator and a PPLN waveguide. In the case of the micro-ring resonator, four-wave mixing
was used to generate the entangled pairs in the two source setups, while in the case of the PPLN waveguide a single laser was used for pumping both SPDC crystals.
% To the best of my knowledge, we will be the first to try to show entanglement swapping by two completely independent bulk sources, using completely independent measurement
% and analysis tools while pumping with a CW laser.

\par Another possible approach for the future is to miniaturize the source and create
an on chip variant, as has been shown by different groups for frequency-bin entanglement \cite{S_G_S_C_F_B_L_G_B_2022},
polarization entanglement \cite{L_Z_F_F_L_L_W_R_D_X_etal._2017}, and also for hybrid frequency-polarization
entangled states \cite{F_R_D_F_L_M_A_B_D_2023}.
Creating a chip based source is another possible avenue for the future which may be explored at some time in our group.

\section{Research questions and goals}
This thesis aims to develop a high-yield, broadband source of entangled photons suitable for both
laboratory-based quantum research and future quantum networks. A bright and spectrally broad source is essential for
supporting multiple simultaneous experiments while ensuring that further spectral filtering does not reduce the signal to an unusable level.
Such capabilities are particularly important for quantum networks, where narrower bandwidths are often required
for efficient coupling to quantum memories and quantum repeaters. The primary motivation for using a
continuous-wave (CW) laser instead of a pulsed laser (PL) stems from our involvement in the EuroQCI project,
specifically the Slovenian Quantum Communication Infrastructure Demonstration (\href{https://siquid.fmf.uni-lj.si/}{SiQUID}),
which focuses on entanglement-based Quantum Key Distribution (QKD) protocols. Additionally, this work will demonstrate
entanglement swapping between two completely independent sources, a crucial step toward scalable quantum
networks capable of distributing entangled photon pairs to multiple users with high fidelity and efficiency.
The main goals (C) and questions (R) of this thesis are listed as follows:

\begin{minipage}{0.95\textwidth}
	\begin{itemize}
		\item[] \textbf{\textit{$\mathcal{C}$\textcal{1}}} - Develop a state of the art source of broadband entanglement that generates states with high tangle
		      and fidelity to maximally entangled Bell states.
		      The source should be bright enough to support multiple concurrent experiments and users in a research setting.
		      It will also be used for a number of other experiments by other students in the group.

		\item[] \textbf{\textit{$\mathcal{C}$\textcal{2}}} - Use the source to demonstrate quantum teleportation and entanglement swapping between independent
		      sources at quantum nodes at FMF and JSI, contributing to the development of real-world quantum network architectures.

		\item[] \textbf{\textit{$\mathcal{C}$\textcal{3}}} - Investigate the feasibility of implementing entanglement swapping over a short distance free-space link.
		      This could serve as a valuable test-bed for alternative methods of entanglement distribution in regions with no readily available fiber networks.

		\item[] \textbf{\textit{$\mathcal{R}$\textcal{1}}} - Can this entanglement source achieve performance surpassing the current state of the
		      art relating to CW entanglement swapping using existing technology?

		      % \item[] \textbf{\textit{$\mathcal{R}$\textcal{2}}} - How large must the cavities be in order to reduce 
		      % the bandwidth of the SPDC photons for the HOM interference measurement?

		\item[] \textbf{\textit{$\mathcal{R}$\textcal{2}}} - Under which conditions can Dense Wavelength Division Multiplexing (DWDM) channels,
		      each with a 100 GHz bandwidth, be utilized to perform quantum optical measurements such as HOM interference?
		      If not, what can be done in addition to observe HOM interference?
		      After filtering what would be the new bandwidth and brightness of the source for that channel?
		      % Try to find optimum between the two.
	\end{itemize}
\end{minipage}
\par These objectives and research questions define the scope of this thesis,
guiding the experimental work and theoretical analysis required to advance the state of the art
of broadband entanglement sources for quantum communication and networking.
% \newpage
\section{Outline of research and research methods}
The focus of this thesis is to implement a Sagnac interferometer source of polarization-entangled photons centered around 1560 nm,
designed to be sufficiently broadband to accommodate multiple DWDM frequency channels.
In the case of the current thesis I will use a 50 mm Periodically polled Lithium Niobate (PPLN)
Type-0 SPDC ($e_{\mathrm{pump}} \rightarrow e_{\mathrm{signal}} + e_{\mathrm{idler}}$) crystal placed in a Sagnac interferometer
which will be bi-directionally pumped by a CW 780.24 nm laser.
\par The CW pump will be set to a diagonal polarization state ($\frac{1}{\sqrt{2}}(\ket{H_{\mathrm{pump}}} + \ket{V_{\mathrm{pump}}})$). On arriving to the
PBS the beam is split into two. The reflected ($\ket{V_{\mathrm{pump}}}$) beam first passes through a half-wave plate (HWP) to rotate the
polarization from $\ket{V_{\mathrm{pump}}}$ to $\ket{H_{\mathrm{pump}}}$, as is required by phase matching conditions (in the actual setup the crystal is rotated by 90° along the beam axis), then through
the crystal where it generates two $\ket{H_{\mathrm{signal}\ (\mathrm{idler})}}$ photons around 1560 nm, and then through the PBSs $\ket{H}$ output where it gets recombined
with the now two $\ket{V_{\mathrm{signal}\ (\mathrm{idler})}}$ photons from the counter propagating branch.

After the two bi-photon pairs pass through the PBS they are reflected by a dichroic mirror, and diverted into a collimating lens
after which they are finally coupled into single mode fiber. The photons from opposing directions will then be in a
$\Phi = \frac{1}{\sqrt{2}}\left( \ket{H_{\mathrm{signal}}H_{\mathrm{idler}}} + e^{i \phi}\ket{V_{\mathrm{signal}}V_{\mathrm{idler}}}\right)$ Bell state\footnote{There
	are four Bell states which form an orthonormal basis for two qubit systems
	$\Psi^{+,-}=\frac{1}{\sqrt{2}}\left(\ket{HV} \pm \ket{VH}\right),\\\Phi^{+,-} = \frac{1}{\sqrt{2}} \left( \ket{HH} \pm \ket{VV} \right)\}$.}.
In order to choose any of the two available Bell states, in the pump beam path, we will have
a relative phase setter between the two counter propagating paths, consisting of a two quarter-wave plates (QWP), and a HWP between them.
The QWPs are set to $\frac{\pi}{4}$, and the HWP is used to set the appropriate phase to select one of the $\Phi$ Bell states.

\par To maximize the efficiency of these telecom networks for multiple users, the available bandwidth (approximately 7400 GHz, or 60 nm)
is divided into many frequency channels using a DWDM.
For all of the tests a DWDM of 100 GHz channel bandwidth, or roughly 0.81 nm, will be used.
For simply checking whether the source produces entangled pairs it is enough
to violate the Bell inequality by doing a CHSH \cite{Clauser_Horne_Shimony_Holt_1969} inequality.
We have chosen to do this in the linear basis as it requires
only two HWPs and two PBSs. The measurement bases here are then offset by $\frac{\pi}{8}$. After this
we will perform a QST measurement to reconstruct \cite{James_Kwiat_Munro_White_2001} the density matrix of the entangled state.
To measure the actual bandwidth of our DWDM channels a HBT measurement will be performed.
Subsequently, a HOM measurement will be conducted, as well as quantum teleportation \cite{Bouwmeester_Pan_Mattle_Eibl_Weinfurter_Zeilinger_1997}
and entanglement swapping \cite{Jennewein_Weihs_Pan_Zeilinger_2001} experiments will be conducted in collaboration with the Jožef Stefan Institute (JSI)
entanglement lab, who will develop a similar entanglement source with our assistance.

To ensure stability, the pump laser wavelength will be locked to an absorption line in Rubidium gas using atomic spectroscopy
via the $\mathrm{^{87}Rb\ D_2}$ transition \cite{metger2017sas}. A small percentage of the pump beam will be diverted into a separate
setup where it will be further split into counter two propagating beams passing through a gas cell for cancelling Doppler broadening,
and a reference beam going straight through the gas cell. The reference beam and one of the counter propagating signal beams will be
coupled to a balanced photodiode, and the error signal will be fed into the laser controller for locking.

In order for two distant parties to be able to measure the correct time tags for analysis, each party will have to synchronize
to a reference to minimize the relative timing jitter between measurements in the two labs. Options are currently being explored on how to exactly do this.
A proposed idea is to use GPS clocks which are disciplined regularly in both locations which will be set as the external clock input of the time taggers.
Currently, tests are being made to see whether this approach will works for us. Another idea was to use an atomic reference.

Regarding the short distance free space test, we will likely engineer or buy basic telescope sender and receiver stations.
We will follow the procedure developed here by \cite{Kržič_2024}. Upon completion, I will test entanglement swapping via free space
link. This setup will be used primarily for another project of the group, where 1324 nm photons will be generated in a non-degenerate SPDC process
along with 852 nm photons to couple to a quantum memory device built by JSI. For this purpose, they must send the 852 nm photon
to the corresponding JSI group. This will also give us experience for the future if given the opportunity to create a connection between
far away nodes via a ground-space-ground link.

The last part of the thesis will be about active polarization control
\cite{CCSHDCDRS}. To measure the correct states, the
idea is to use an electronic polarization controller in the experimental network to create an algorithm which will be able to
ensure the correct polarization state is being received on the measurement stage.

All of the measurements for this thesis will be performed using Superconducting Nanowire Single Photon Detectors (SNSPDs) ID281 from IDQuantique. The
measurements at JSI will be performed using SNSPDs from Single Quantum.


\section{Expected results and original contributions to science}
Lastly, I will talk about the expected entanglement swapping rate that we hope to achieve, and also the HOM interference visibility, which
is an important measurement for achieving entanglement swapping. Currently, without any optimizations in regards to heralding
or brightness the source is capable of producing around 35000 correlated pairs per second per mW, per branch of the Sagnac loop.
To the best of my knowledge, this is already a considerably high amount of correlated pairs for such a source, and I will aim to improve it
farther. Using this, and the estimated coincidence window of 300 ps, a rate of around 0.3 four-folds per second should be achievable.
This would mean that to gather enough four-folds for a HOM measurement, one would have to integrate for roughly 10 minutes per point.
To the best of my knowledge, this would be a new record result for CW HOM, and it might open other possibilities for further research.
I have assumed that the jitter of the detectors and time tagger is Gaussian. The goal of the thesis is also to improve the above mentioned
values to their respective theoretical limits.
\par To assess the impact of timing jitter on the quality of interference measurements, consider the bandwidth of the SPDC photons, which we assume
to be 100 GHz. This corresponds to a coherence time of approximately 10 ps, and assuming the total measurement timing jitter is estimated
to be 44 ps by using the Root Sum of Variances method.
The maximum visibility of the HOM interference measurement that one can hope for then is around 20\%, which is well below the minimal value for violating
Bells inequality.
This then leads to a rough estimate for the minimal value of the coherence time for the SPDC photons to 100 ps for the HOM measurement, possibly even larger.
Hypothetically, assuming such a total timing jitter without including synchronization jitter we can expect to get a visibility of at most 91\%. Clearly,
adding in synchronization will reduce this number even farther for that coherence time, and increasing the coherence time would then also increase it proportionally.
\par In order for the measurement to not take too long, but still give desirable results, there might need to be a compromise
between the bandwidth and integration time for the HOM measurement. For Type-0 SPDC there should be a linear relationship between the brightness
and the amount of filtering being done, as the process is quite broad and flat around the maximum.
Filtering will of course also reduce the amount of coincidences we will be able to see.

% Possible TODO Working together with a company (mention maybe the experimental network from SiQuid)

\section{Draft plan for management of research data}
During my doctoral research, I will collect and analyze time tagger data,
which records photon detection events with precise timestamps.
The data will be stored in CSV files and analyzed using C++ and Python scripts.

To ensure data integrity and reproducibility, I will organize my research data as follows:
\begin{itemize}
	\item Raw data (time tagger outputs) will be stored in a structured directory on one of our laboratory computers, sorted by experiment date and parameters.
	\item Processed data (results of filtering, calibration, and analysis) will be saved in separate CSV files, maintaining a clear relationship with the raw data.
	\item Analysis scripts (C++ and Python code) will be version-controlled using Git.
\end{itemize}
For long-term storage and accessibility, I plan to deposit my research data in an appropriate open-access data repository,
such as the University of Ljubljana Repository. The dataset will include:
% such as Zenodo, Figshare, or the University of Ljubljana Repository. The dataset will include:

\begin{itemize}
	\item Metadata describing the experiment setup, parameters, and conditions.
	\item Raw and processed CSV data,
\end{itemize}

Documentation explaining the data structure and how to reproduce the results using the provided scripts.
The data will be made available upon request unless confidentiality or ethical restrictions apply.
When sharing, I will ensure compliance with FAIR principles (Findability, Accessibility,
Interoperability, and Reusability) by providing proper documentation and referencing my datasets in publications.
\newpage
\bibliographystyle{IEEEtran}
\bibliography{reference}
\end{document}
