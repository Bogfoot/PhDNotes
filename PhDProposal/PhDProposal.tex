\documentclass{article}
%%%%%%%%%%%%%%%%%%%%%%%%%%%%%%%%%
% PACKAGE IMPORTS
%%%%%%%%%%%%%%%%%%%%%%%%%%%%%%%%%
\usepackage[framemethod=TikZ]{mdframed}
\usepackage{amsthm}
\usepackage{tikzsymbols}
\usepackage[tmargin=2.5cm,rmargin=2cm,lmargin=2.5cm,margin=0.85in,bmargin=2.5cm,footskip=.2in]{geometry}
\linespread{1.3}
\usepackage{amsmath,amsfonts,amsthm,amssymb,mathtools}
\usepackage[varbb]{newpxmath}
\usepackage{xfrac}
\usepackage[makeroom]{cancel}
\usepackage{bookmark}
\usepackage{enumitem}
\usepackage{hyperref,theoremref}
\hypersetup{
	pdftitle={Assignment},
	colorlinks=true, linkcolor=doc!90,
	bookmarksnumbered=true,
	bookmarksopen=true
}
\usepackage[most,many,breakable]{tcolorbox}
\usepackage{xcolor}
\usepackage{varwidth}
\usepackage{etoolbox}
\usepackage{nameref}
\usepackage{multicol,array}
\usepackage{tikz-cd}
\usepackage[ruled,vlined,linesnumbered]{algorithm2e}
\usepackage{comment} % enables the use of multi-line comments (\ifx \fi) 
\usepackage{import}
\usepackage{xifthen}
\usepackage{pdfpages}
\usepackage{transparent}
\usepackage[bottom]{footmisc}
\usepackage[utf8]{inputenc} % allow utf-8 input
\usepackage{braket}
\usepackage{wrapfig}			  %Kontekstom osjetljivo navođenje
\usepackage{csquotes}			  %Kontekstom osjetljivo navođenje
\usepackage{graphicx}       %Slike i slično
\usepackage[T1]{fontenc}    % use 8-bit T1 fonts
\usepackage{url}            % simple URL typesetting
\usepackage{booktabs}       % professional-quality tables
\usepackage{nicefrac}       % compact symbols for 1/2, etc.
\usepackage{float}
\usepackage{caption}
\usepackage{subcaption}
\usepackage{listings}
\usepackage[english,croatian]{babel}
\usepackage{titlesec}				%Za naslovnu stranicu

\counterwithin{figure}{section}
\urlstyle{same}
\numberwithin{equation}{section}
\hyphenation{pergamon} % riječ u argumentu (pergamon) se ne rastavlja s crticom; ne smije imat specijalna slova. č. ž... - rastavljam ju naredbom \- (npr. išče\-zava)
\setlength\parindent{0pt} % u novom paragrafu: indent=0
\setlength{\parskip}{10pt} % postavlja željeni vertikalni razmak između paragrafa
\setlength{\skip\footins}{2cm} % razmak između glavnog teksta i fusnota
\renewcommand{\thefootnote}{$\ddagger$} % želim dagger za oznaku fusnota
\newcommand{\HRule}{\rule{\linewidth}{0.4mm}} % nova naredba za horizontalne linije na naslovnoj stranici
\newlength{\mylen}
\setcounter{secnumdepth}{4}
\titleformat{\paragraph}
{\normalfont\normalsize\bfseries}{\theparagraph}{1em}{}


%%%%%%%%%%%%%%%%%%%%%%%%%%%%%%%%%%%%%%%%%%%
% TABLE OF CONTENTS
%%%%%%%%%%%%%%%%%%%%%%%%%%%%%%%%%%%%%%%%%%%

\usepackage{tikz}
\definecolor{doc}{RGB}{0,60,110}
\usepackage{titletoc}
\contentsmargin{0cm}
\titlecontents{chapter}[3.7pc]
{\addvspace{30pt}%
	\begin{tikzpicture}[remember picture, overlay]%
		\draw[fill=doc!60,draw=doc!60] (-7,-.1) rectangle (-0.9,.5);%
		\pgftext[left,x=-3.5cm,y=0.2cm]{\color{white}\Large\sc\bfseries Chapter\ \thecontentslabel};%
	\end{tikzpicture}\color{doc!60}\large\sc\bfseries}%
{}
{}
{\;\titlerule\;\large\sc\bfseries Page \thecontentspage
	\begin{tikzpicture}[remember picture, overlay]
		\draw[fill=doc!60,draw=doc!60] (2pt,0) rectangle (4,0.1pt);
	\end{tikzpicture}}%
\titlecontents{section}[3.7pc]
{\addvspace{2pt}}
{\contentslabel[\thecontentslabel]{2pc}}
{}
{\hfill\small \thecontentspage}
[]
\titlecontents*{subsection}[3.7pc]
{\addvspace{-1pt}\small}
{}
{}
{\ --- \small\thecontentspage}
[ \textbullet\ ][]

\makeatletter
\renewcommand{\tableofcontents}{%
	\chapter{%
	  \vspace{-20\p@}%
	  \begin{tikzpicture}[remember picture, overlay]%
		  \pgftext[right,x=15cm,y=0.2cm]{\color{doc!60}\Huge\sc\bfseries \contentsname};%
		  \draw[fill=doc!60,draw=doc!60] (13,-.75) rectangle (20,1);%
		  \clip (13,-.75) rectangle (20,1);
		  \pgftext[right,x=15cm,y=0.2cm]{\color{white}\Huge\sc\bfseries \contentsname};%
	  \end{tikzpicture}}%
	\@starttoc{toc}}
\makeatother


%%%%%%%%%%%%
%% KUTIJE %%
%%%%%%%%%%%%
%%%%%%%%%%%%%%%%%%%%%%%%%%%%%%
%Theorem
\newcounter{teorem}[section] \setcounter{teorem}{0}
\renewcommand{\theteorem}{\arabic{section}.\arabic{teorem}}
\newenvironment{teorem}[2][]{%
	\refstepcounter{teorem}%
	\ifstrempty{#1}%
	{\mdfsetup{%
			frametitle={%
					\tikz[baseline=(current bounding box.east),outer sep=0pt]
					\node[anchor=east,rectangle,fill=blue!20]
					{\strut Teorem~\thetheo};}}
	}%
	{\mdfsetup{%
			frametitle={%
					\tikz[baseline=(current bounding box.east),outer sep=0pt]
					\node[anchor=east,rectangle,fill=blue!20]
					{\strut Teorem~\thetheo:~#1};}}%
	}%
	\mdfsetup{innertopmargin=10pt,linecolor=blue!20,%
		linewidth=2pt,topline=true,%
		frametitleaboveskip=\dimexpr-\ht\strutbox\relax
	}
	\begin{mdframed}[]\relax%
		\label{#2}}{\end{mdframed}}
%%%%%%%%%%%%%%%%%%%%%%%%%%%%%%
%Proof
\newcounter{dokaz}[section]\setcounter{dokaz}{0}
\renewcommand{\thedokaz}{\arabic{section}.\arabic{dokaz}}
\newenvironment{dokaz}[2][]{%
	\refstepcounter{dokaz}%
	\ifstrempty{#1}%
	{\mdfsetup{%
			frametitle={%
					\tikz[baseline=(current bounding box.east),outer sep=0pt]
					\node[anchor=east,rectangle,fill=red!20]
					{\strut Dokaz~\theprf};}}
	}%
	{\mdfsetup{%
			frametitle={%
					\tikz[baseline=(current bounding box.east),outer sep=0pt]
					\node[anchor=east,rectangle,fill=red!20]
					{\strut Dokaz~\theprf:~#1};}}%
	}%
	\mdfsetup{innertopmargin=10pt,linecolor=red!20,%
		linewidth=2pt,topline=true,%
		frametitleaboveskip=\dimexpr-\ht\strutbox\relax
	}
	\begin{mdframed}[]\relax%
		\label{#2}}{\qed\end{mdframed}}
%%%%%%%%%%%%%%%%%%%%%%%%%%%%%%

\newcommand{\eps}{\epsilon}
\newcommand{\veps}{\varepsilon}
\newcommand{\ol}{\overline}
\newcommand{\ul}{\underline}
\newcommand{\wt}{\widetilde}
\newcommand{\wh}{\widehat}
\newcommand{\vocab}[1]{\textbf{\color{blue} #1}}
\providecommand{\half}{\frac{1}{2}}
\newcommand{\dang}{\measuredangle} %% Directed angle
\newcommand{\ray}[1]{\overrightarrow{#1}}
\newcommand{\seg}[1]{\overline{#1}}
\newcommand{\arc}[1]{\wideparen{#1}}
\DeclareMathOperator{\cis}{cis}
\DeclareMathOperator*{\lcm}{lcm}
\DeclareMathOperator*{\argmin}{arg min}
\DeclareMathOperator*{\argmax}{arg max}
\newcommand{\cycsum}{\sum_{\mathrm{cyc}}}
\newcommand{\symsum}{\sum_{\mathrm{sym}}}
\newcommand{\cycprod}{\prod_{\mathrm{cyc}}}
\newcommand{\symprod}{\prod_{\mathrm{sym}}}
\newcommand{\Qed}{\begin{flushright}\qed\end{flushright}}
\newcommand{\parinn}{\setlength{\parindent}{1cm}}
\newcommand{\parinf}{\setlength{\parindent}{0cm}}
% \newcommand{\norm}{\|\cdot\|}
\newcommand{\inorm}{\norm_{\infty}}
\newcommand{\opensets}{\{V_{\alpha}\}_{\alpha\in I}}
\newcommand{\oset}{V_{\alpha}}
\newcommand{\opset}[1]{V_{\alpha_{#1}}}
\newcommand{\lub}{\text{lub}}
\newcommand{\del}[2]{\frac{\partial #1}{\partial #2}}
\newcommand{\Del}[3]{\frac{\partial^{#1} #2}{\partial^{#1} #3}}
\newcommand{\deld}[2]{\dfrac{\partial #1}{\partial #2}}
\newcommand{\Deld}[3]{\dfrac{\partial^{#1} #2}{\partial^{#1} #3}}
\newcommand{\lm}{\lambda}
\newcommand{\uin}{\mathbin{\rotatebox[origin=c]{90}{$\in$}}}
\newcommand{\usubset}{\mathbin{\rotatebox[origin=c]{90}{$\subset$}}}
\newcommand{\lt}{\left}
\newcommand{\rt}{\right}
\newcommand{\bs}[1]{\boldsymbol{#1}}
\newcommand{\exs}{\exists}
\newcommand{\st}{\strut}
\newcommand{\dps}[1]{\displaystyle{#1}}

\newcommand{\sol}{\setlength{\parindent}{0cm}\textbf{\textit{Solution:}}\setlength{\parindent}{1cm} }
\newcommand{\solve}[1]{\setlength{\parindent}{0cm}\textbf{\textit{Solution: }}\setlength{\parindent}{1cm}#1 \Qed}

%From M275 "Topology" at SJSU
\newcommand{\id}{\mathrm{id}}
\newcommand{\taking}[1]{\xrightarrow{#1}}
\newcommand{\inv}{^{-1}}

%From M170 "Introduction to Graph Theory" at SJSU
\DeclareMathOperator{\diam}{diam}
\DeclareMathOperator{\ord}{ord}
\newcommand{\defeq}{\overset{\mathrm{def}}{=}}

%From the USAMO .tex files
\newcommand{\ts}{\textsuperscript}
\newcommand{\dg}{^\circ}
\newcommand{\ii}{\item}

% % From Math 55 and Math 145 at Harvard
% \newenvironment{subproof}[1][Proof]{%
% \begin{proof}[#1] \renewcommand{\qedsymbol}{$\blacksquare$}}%
% {\end{proof}}

\newcommand{\liff}{\leftrightarrow}
\newcommand{\lthen}{\rightarrow}
\newcommand{\opname}{\operatorname}
\newcommand{\surjto}{\twoheadrightarrow}
\newcommand{\injto}{\hookrightarrow}
\newcommand{\On}{\mathrm{On}} % ordinals
\DeclareMathOperator{\img}{im} % Image
\DeclareMathOperator{\Img}{Im} % Image
\DeclareMathOperator{\coker}{coker} % Cokernel
\DeclareMathOperator{\Coker}{Coker} % Cokernel
\DeclareMathOperator{\Ker}{Ker} % Kernel
\DeclareMathOperator{\rank}{rank}
\DeclareMathOperator{\Spec}{Spec} % spectrum
\DeclareMathOperator{\Tr}{Tr} % trace
\DeclareMathOperator{\pr}{pr} % projection
\DeclareMathOperator{\ext}{ext} % extension
\DeclareMathOperator{\pred}{pred} % predecessor
\DeclareMathOperator{\dom}{dom} % domain
\DeclareMathOperator{\ran}{ran} % range
\DeclareMathOperator{\Hom}{Hom} % homomorphism
\DeclareMathOperator{\Mor}{Mor} % morphisms
\DeclareMathOperator{\End}{End} % endomorphism

% Things Lie
\newcommand{\kb}{\mathfrak b}
\newcommand{\kg}{\mathfrak g}
\newcommand{\kh}{\mathfrak h}
\newcommand{\kn}{\mathfrak n}
\newcommand{\ku}{\mathfrak u}
\newcommand{\kz}{\mathfrak z}
\DeclareMathOperator{\Ext}{Ext} % Ext functor
\DeclareMathOperator{\Tor}{Tor} % Tor functor
\newcommand{\gl}{\opname{\mathfrak{gl}}} % frak gl group
\renewcommand{\sl}{\opname{\mathfrak{sl}}} % frak sl group chktex 6

% More script letters etc.
\newcommand{\SA}{\mathcal A}
\newcommand{\SB}{\mathcal B}
\newcommand{\SC}{\mathcal C}
\newcommand{\SF}{\mathcal F}
\newcommand{\SG}{\mathcal G}
\newcommand{\SH}{\mathcal H}
\newcommand{\OO}{\mathcal O}

\newcommand{\SCA}{\mathscr A}
\newcommand{\SCB}{\mathscr B}
\newcommand{\SCC}{\mathscr C}
\newcommand{\SCD}{\mathscr D}
\newcommand{\SCE}{\mathscr E}
\newcommand{\SCF}{\mathscr F}
\newcommand{\SCG}{\mathscr G}
\newcommand{\SCH}{\mathscr H}

% Mathfrak primes
\newcommand{\km}{\mathfrak m}
\newcommand{\kp}{\mathfrak p}
\newcommand{\kq}{\mathfrak q}

% number sets
\newcommand{\RR}[1][]{\ensuremath{\ifstrempty{#1}{\mathbb{R}}{\mathbb{R}^{#1}}}}
\newcommand{\NN}[1][]{\ensuremath{\ifstrempty{#1}{\mathbb{N}}{\mathbb{N}^{#1}}}}
\newcommand{\ZZ}[1][]{\ensuremath{\ifstrempty{#1}{\mathbb{Z}}{\mathbb{Z}^{#1}}}}
\newcommand{\QQ}[1][]{\ensuremath{\ifstrempty{#1}{\mathbb{Q}}{\mathbb{Q}^{#1}}}}
\newcommand{\CC}[1][]{\ensuremath{\ifstrempty{#1}{\mathbb{C}}{\mathbb{C}^{#1}}}}
\newcommand{\PP}[1][]{\ensuremath{\ifstrempty{#1}{\mathbb{P}}{\mathbb{P}^{#1}}}}
\newcommand{\HH}[1][]{\ensuremath{\ifstrempty{#1}{\mathbb{H}}{\mathbb{H}^{#1}}}}
\newcommand{\FF}[1][]{\ensuremath{\ifstrempty{#1}{\mathbb{F}}{\mathbb{F}^{#1}}}}
% expected value
\newcommand{\EE}{\ensuremath{\mathbb{E}}}
\newcommand{\charin}{\text{ char }}
\DeclareMathOperator{\sign}{sign}
\DeclareMathOperator{\Aut}{Aut}
\DeclareMathOperator{\Inn}{Inn}
\DeclareMathOperator{\Syl}{Syl}
\DeclareMathOperator{\Gal}{Gal}
\DeclareMathOperator{\GL}{GL} % General linear group
\DeclareMathOperator{\SL}{SL} % Special linear group

%---------------------------------------
% BlackBoard Math Fonts :-
%---------------------------------------

%Captital Letters
\newcommand{\bbA}{\mathbb{A}}	\newcommand{\bbB}{\mathbb{B}}
\newcommand{\bbC}{\mathbb{C}}	\newcommand{\bbD}{\mathbb{D}}
\newcommand{\bbE}{\mathbb{E}}	\newcommand{\bbF}{\mathbb{F}}
\newcommand{\bbG}{\mathbb{G}}	\newcommand{\bbH}{\mathbb{H}}
\newcommand{\bbI}{\mathbb{I}}	\newcommand{\bbJ}{\mathbb{J}}
\newcommand{\bbK}{\mathbb{K}}	\newcommand{\bbL}{\mathbb{L}}
\newcommand{\bbM}{\mathbb{M}}	\newcommand{\bbN}{\mathbb{N}}
\newcommand{\bbO}{\mathbb{O}}	\newcommand{\bbP}{\mathbb{P}}
\newcommand{\bbQ}{\mathbb{Q}}	\newcommand{\bbR}{\mathbb{R}}
\newcommand{\bbS}{\mathbb{S}}	\newcommand{\bbT}{\mathbb{T}}
\newcommand{\bbU}{\mathbb{U}}	\newcommand{\bbV}{\mathbb{V}}
\newcommand{\bbW}{\mathbb{W}}	\newcommand{\bbX}{\mathbb{X}}
\newcommand{\bbY}{\mathbb{Y}}	\newcommand{\bbZ}{\mathbb{Z}}

%---------------------------------------
% MathCal Fonts :-
%---------------------------------------

%Captital Letters
\newcommand{\mcA}{\mathcal{A}}	\newcommand{\mcB}{\mathcal{B}}
\newcommand{\mcC}{\mathcal{C}}	\newcommand{\mcD}{\mathcal{D}}
\newcommand{\mcE}{\mathcal{E}}	\newcommand{\mcF}{\mathcal{F}}
\newcommand{\mcG}{\mathcal{G}}	\newcommand{\mcH}{\mathcal{H}}
\newcommand{\mcI}{\mathcal{I}}	\newcommand{\mcJ}{\mathcal{J}}
\newcommand{\mcK}{\mathcal{K}}	\newcommand{\mcL}{\mathcal{L}}
\newcommand{\mcM}{\mathcal{M}}	\newcommand{\mcN}{\mathcal{N}}
\newcommand{\mcO}{\mathcal{O}}	\newcommand{\mcP}{\mathcal{P}}
\newcommand{\mcQ}{\mathcal{Q}}	\newcommand{\mcR}{\mathcal{R}}
\newcommand{\mcS}{\mathcal{S}}	\newcommand{\mcT}{\mathcal{T}}
\newcommand{\mcU}{\mathcal{U}}	\newcommand{\mcV}{\mathcal{V}}
\newcommand{\mcW}{\mathcal{W}}	\newcommand{\mcX}{\mathcal{X}}
\newcommand{\mcY}{\mathcal{Y}}	\newcommand{\mcZ}{\mathcal{Z}}



%---------------------------------------
% Bold Math Fonts :-
%---------------------------------------

%Captital Letters
\newcommand{\bmA}{\boldsymbol{A}}	\newcommand{\bmB}{\boldsymbol{B}}
\newcommand{\bmC}{\boldsymbol{C}}	\newcommand{\bmD}{\boldsymbol{D}}
\newcommand{\bmE}{\boldsymbol{E}}	\newcommand{\bmF}{\boldsymbol{F}}
\newcommand{\bmG}{\boldsymbol{G}}	\newcommand{\bmH}{\boldsymbol{H}}
\newcommand{\bmI}{\boldsymbol{I}}	\newcommand{\bmJ}{\boldsymbol{J}}
\newcommand{\bmK}{\boldsymbol{K}}	\newcommand{\bmL}{\boldsymbol{L}}
\newcommand{\bmM}{\boldsymbol{M}}	\newcommand{\bmN}{\boldsymbol{N}}
\newcommand{\bmO}{\boldsymbol{O}}	\newcommand{\bmP}{\boldsymbol{P}}
\newcommand{\bmQ}{\boldsymbol{Q}}	\newcommand{\bmR}{\boldsymbol{R}}
\newcommand{\bmS}{\boldsymbol{S}}	\newcommand{\bmT}{\boldsymbol{T}}
\newcommand{\bmU}{\boldsymbol{U}}	\newcommand{\bmV}{\boldsymbol{V}}
\newcommand{\bmW}{\boldsymbol{W}}	\newcommand{\bmX}{\boldsymbol{X}}
\newcommand{\bmY}{\boldsymbol{Y}}	\newcommand{\bmZ}{\boldsymbol{Z}}
%Small Letters
\newcommand{\bma}{\boldsymbol{a}}	\newcommand{\bmb}{\boldsymbol{b}}
\newcommand{\bmc}{\boldsymbol{c}}	\newcommand{\bmd}{\boldsymbol{d}}
\newcommand{\bme}{\boldsymbol{e}}	\newcommand{\bmf}{\boldsymbol{f}}
\newcommand{\bmg}{\boldsymbol{g}}	\newcommand{\bmh}{\boldsymbol{h}}
\newcommand{\bmi}{\boldsymbol{i}}	\newcommand{\bmj}{\boldsymbol{j}}
\newcommand{\bmk}{\boldsymbol{k}}	\newcommand{\bml}{\boldsymbol{l}}
\newcommand{\bmm}{\boldsymbol{m}}	\newcommand{\bmn}{\boldsymbol{n}}
\newcommand{\bmo}{\boldsymbol{o}}	\newcommand{\bmp}{\boldsymbol{p}}
\newcommand{\bmq}{\boldsymbol{q}}	\newcommand{\bmr}{\boldsymbol{r}}
\newcommand{\bms}{\boldsymbol{s}}	\newcommand{\bmt}{\boldsymbol{t}}
\newcommand{\bmu}{\boldsymbol{u}}	\newcommand{\bmv}{\boldsymbol{v}}
\newcommand{\bmw}{\boldsymbol{w}}	\newcommand{\bmx}{\boldsymbol{x}}
\newcommand{\bmy}{\boldsymbol{y}}	\newcommand{\bmz}{\boldsymbol{z}}

%---------------------------------------
% Scr Math Fonts :-
%---------------------------------------

\newcommand{\sA}{{\mathscr{A}}}   \newcommand{\sB}{{\mathscr{B}}}
\newcommand{\sC}{{\mathscr{C}}}   \newcommand{\sD}{{\mathscr{D}}}
\newcommand{\sE}{{\mathscr{E}}}   \newcommand{\sF}{{\mathscr{F}}}
\newcommand{\sG}{{\mathscr{G}}}   \newcommand{\sH}{{\mathscr{H}}}
\newcommand{\sI}{{\mathscr{I}}}   \newcommand{\sJ}{{\mathscr{J}}}
\newcommand{\sK}{{\mathscr{K}}}   \newcommand{\sL}{{\mathscr{L}}}
\newcommand{\sM}{{\mathscr{M}}}   \newcommand{\sN}{{\mathscr{N}}}
\newcommand{\sO}{{\mathscr{O}}}   \newcommand{\sP}{{\mathscr{P}}}
\newcommand{\sQ}{{\mathscr{Q}}}   \newcommand{\sR}{{\mathscr{R}}}
\newcommand{\sS}{{\mathscr{S}}}   \newcommand{\sT}{{\mathscr{T}}}
\newcommand{\sU}{{\mathscr{U}}}   \newcommand{\sV}{{\mathscr{V}}}
\newcommand{\sW}{{\mathscr{W}}}   \newcommand{\sX}{{\mathscr{X}}}
\newcommand{\sY}{{\mathscr{Y}}}   \newcommand{\sZ}{{\mathscr{Z}}}


%---------------------------------------
% Math Fraktur Font
%---------------------------------------

%Captital Letters
\newcommand{\mfA}{\mathfrak{A}}	\newcommand{\mfB}{\mathfrak{B}}
\newcommand{\mfC}{\mathfrak{C}}	\newcommand{\mfD}{\mathfrak{D}}
\newcommand{\mfE}{\mathfrak{E}}	\newcommand{\mfF}{\mathfrak{F}}
\newcommand{\mfG}{\mathfrak{G}}	\newcommand{\mfH}{\mathfrak{H}}
\newcommand{\mfI}{\mathfrak{I}}	\newcommand{\mfJ}{\mathfrak{J}}
\newcommand{\mfK}{\mathfrak{K}}	\newcommand{\mfL}{\mathfrak{L}}
\newcommand{\mfM}{\mathfrak{M}}	\newcommand{\mfN}{\mathfrak{N}}
\newcommand{\mfO}{\mathfrak{O}}	\newcommand{\mfP}{\mathfrak{P}}
\newcommand{\mfQ}{\mathfrak{Q}}	\newcommand{\mfR}{\mathfrak{R}}
\newcommand{\mfS}{\mathfrak{S}}	\newcommand{\mfT}{\mathfrak{T}}
\newcommand{\mfU}{\mathfrak{U}}	\newcommand{\mfV}{\mathfrak{V}}
\newcommand{\mfW}{\mathfrak{W}}	\newcommand{\mfX}{\mathfrak{X}}
\newcommand{\mfY}{\mathfrak{Y}}	\newcommand{\mfZ}{\mathfrak{Z}}
%Small Letters
\newcommand{\mfa}{\mathfrak{a}}	\newcommand{\mfb}{\mathfrak{b}}
\newcommand{\mfc}{\mathfrak{c}}	\newcommand{\mfd}{\mathfrak{d}}
\newcommand{\mfe}{\mathfrak{e}}	\newcommand{\mff}{\mathfrak{f}}
\newcommand{\mfg}{\mathfrak{g}}	\newcommand{\mfh}{\mathfrak{h}}
\newcommand{\mfi}{\mathfrak{i}}	\newcommand{\mfj}{\mathfrak{j}}
\newcommand{\mfk}{\mathfrak{k}}	\newcommand{\mfl}{\mathfrak{l}}
\newcommand{\mfm}{\mathfrak{m}}	\newcommand{\mfn}{\mathfrak{n}}
\newcommand{\mfo}{\mathfrak{o}}	\newcommand{\mfp}{\mathfrak{p}}
\newcommand{\mfq}{\mathfrak{q}}	\newcommand{\mfr}{\mathfrak{r}}
\newcommand{\mfs}{\mathfrak{s}}	\newcommand{\mft}{\mathfrak{t}}
\newcommand{\mfu}{\mathfrak{u}}	\newcommand{\mfv}{\mathfrak{v}}
\newcommand{\mfw}{\mathfrak{w}}	\newcommand{\mfx}{\mathfrak{x}}
\newcommand{\mfy}{\mathfrak{y}}	\newcommand{\mfz}{\mathfrak{z}}


% Customizing title
\renewcommand{\maketitlehooka}{\centering}
\renewcommand{\maketitlehookb}{\vspace{-1.5em}}

% Customizing abstract
\renewcommand{\abstractnamefont}{\normalfont\large\bfseries}
\renewcommand{\abstracttextfont}{\normalfont\normalsize}
\graphicspath{{./Images/}} % Where to take images from
%%%%%%%%%%%%%%%%%%%%%%%%%% Define some useful colors %%%%%%%%%%%%%%%%%%%%%%%%%%
\definecolor{ocre}{RGB}{243,102,25}
\definecolor{mygray}{RGB}{243,243,244}
\definecolor{deepGreen}{RGB}{26,111,0}
\definecolor{shallowGreen}{RGB}{235,255,255}
\definecolor{deepBlue}{RGB}{61,124,222}
\definecolor{shallowBlue}{RGB}{235,249,255}
%%%%%%%%%%%%%%%%%%%%%%%%%%%%%%%%%%%%%%%%%%%%%%%%%%%%%%%%%%%%%%%%%%%%%%%%%%%%%%%

%%%%%%%%%%%%%%%%%%%%%%%%%% Define an orangebox command %%%%%%%%%%%%%%%%%%%%%%%%
\newcommand\orangebox[1]{\fcolorbox{ocre}{mygray}{\hspace{1em}#1\hspace{1em}}}
%%%%%%%%%%%%%%%%%%%%%%%%%%%%%%%%%%%%%%%%%%%%%%%%%%%%%%%%%%%%%%%%%%%%%%%%%%%%%%%

%%%%%%%%%%%%%%%%%%%%%%%%%%%% English Environments %%%%%%%%%%%%%%%%%%%%%%%%%%%%%
\newtheoremstyle{mytheoremstyle}{3pt}{3pt}{\normalfont}{0cm}{\rmfamily\bfseries}{}{1em}{{\color{black}\thmname{#1}~\thmnumber{#2}}\thmnote{\,--\,#3}}
\newtheoremstyle{myproblemstyle}{3pt}{3pt}{\normalfont}{0cm}{\rmfamily\bfseries}{}{1em}{{\color{black}\thmname{#1}~\thmnumber{#2}}\thmnote{\,--\,#3}}
\theoremstyle{mytheoremstyle}
\newmdtheoremenv[linewidth=1pt,backgroundcolor=shallowGreen,linecolor=deepGreen,leftmargin=0pt,innerleftmargin=20pt,innerrightmargin=20pt,]{theorem}{Theorem}[section]
\theoremstyle{mytheoremstyle}
\newmdtheoremenv[linewidth=1pt,backgroundcolor=shallowBlue,linecolor=deepBlue,leftmargin=0pt,innerleftmargin=20pt,innerrightmargin=20pt,]{definition}{Definition}[section]
\theoremstyle{myproblemstyle}
\newmdtheoremenv[linecolor=black,leftmargin=0pt,innerleftmargin=10pt,innerrightmargin=10pt,]{problem}{Problem}[section]
%%%%%%%%%%%%%%%%%%%%%%%%%%%%%%%%%%%%%%%%%%%%%%%%%%%%%%%%%%%%%%%%%%%%%%%%%%%%%%%
\begin{document}
%%%%%%%%%%%%%%%%%%%%%%%%%%%%%%% Title & Author %%%%%%%%%%%%%%%%%%%%%%%%%%%%%%%%
\begin{titlepage}
	\begin{center}
		{\LARGE \textbf{University of Ljubljana}} \\[0.1cm]
		{\LARGE \textbf{Faculty of Mathematics and Physics}} \\[3cm]
		{\large \textbf{Adrian Udovičić, mag. phys.}}\\[0.1cm]
		{\large \textbf{Rainer O. Kaltenbaek, asoc. prof. dr.}}\\[3cm]
		{\LARGE \textbf{PhD topic proposal}} \\[0.1cm]
		{\LARGE \textbf{Generating and teleporting entanglement for quantum networks}} \\[4cm]
		{\large \textbf{Scientific field: Physics}} \\[0.1cm]
		\vfill
		\large\textbf{{Ljubljana, 2025\date{}}}
	\end{center}
\end{titlepage}
%%%%%%%%%%%%%%%%%%%%%%%%%%%%%%%%%%%%%%%%%%%%%%%%%%%%%%%%%%%%%%%%%%%%%%%%%%%%%%%
\pagenumbering{roman}
\newpage
\tableofcontents
\newpage
\pagenumbering{arabic}
\section{Description of the immediate research area and its problems}
% (approximately half a page)
%• State and briefly describe the immediate research area in which research will be conducted, and to which an original contribution is expected.
% Emphasise the importance and timeliness of the narrower field.
%• In a few sentences, describe the essence of the problem that will be tackled in the doctoral research and motivate it.
% (a more detailed description is required later in the "dispozicija").

In the rapidly advancing fields of quantum communication and quantum computing, sensing, and simulators
the efficient transfer of secure quantum information is of great importance.
A key quantum resource is entanglement, which facilitates experiments such as quantum teleportation and entanglement swapping.
Conducting these experiments over long distances through optical fibers presents significant challenges due to transmission losses.
To mitigate this, photons must be generated at wavelengths compatible with existing fiber-optic networks,
particularly in the C near-infrared band where transmission losses are minimal.
These advances contribute to the broader goal of realizing quantum networks,
which require robust capabilities for generating and characterizing afforementioned quantum processes. %) quantum coherence and entanglement.
Such networks rely on quantum interconnects, which convert quantum states between physical systems in a reversible manner,
enabling the distribution of entanglement and the teleportation of quantum states across network nodes \cite{Kimble_2008}.
Additionally, free-space communication methods \cite{Kržić_et_al_2023} are being explored for applications in metropolitan areas.
\par While these technologies are essential for local and metropolitan quantum networks, scaling to a global level requires overcoming
the inherent limitations of photon loss over long distances. This is where quantum repeaters and high-yield entanglement sources
play a critical role for the future global quantum internet.
Quantum repeaters may enable the distribution of entanglement over arbitrarily long distances by overcoming exponential loss scaling,
even in fiber networks where attenuation is low. Without such repeaters, entanglement distribution is limited to distances
of only a few hundred kilometers.
\par This work seeks to establish the technical bedrock for future scalable quantum networks. These efforts do not exist in isolation;
they directly feed into the broader mission of transforming theoretical quantum advantages into real-world systems.
I aim to achieve not only the first realization of a high-yield polarization entanglement source at non-degenerate
frequencies in Slovenia but also to demonstrate quantum teleportation and entanglement swapping.
Ongoing research efforts aim to bridge the gap between theoretical advancements and practical applications,
driving the quest for more efficient and accessible quantum systems that could transform various sectors,
including telecommunications and healthcare. As these technologies mature, they are poised to redefine our
understanding of information processing and secure communications in the quantum era.
\par\textbf{Key words: Quantum Entanglement, Quantum Communication, Entanglement Swapping}

\section{Overview of related research and relevant literature}
% (approximately one page)
% • Briefly summarise the state-of-the-art. Review and briefly analyse the relevant literature including the most recent research in the area.
% • Explain how the proposed topic leads on from existing research, and outline the importance of the proposed line of research and the challenges it poses.
% • Explain the relevance and timeliness of the proposed research in the context of the literature reviewed above.
Quantum entanglement sources are pivotal components in the field of quantum mechanics, enabling the generation of entangled states
that are essential for a range of applications, including quantum computing, cryptography, simulations, and communication.
These sources can produce pairs of entangled photons through various techniques such as cavity-enhanced configurations,
quantum dot mechanisms, and by far the most widely used method being Spontaneous Parametric Down-Conversion (SPDC).
The ability to create reliable and efficient entangled states has garnered significant interest due to their 
implications for advancing quantum technologies and facilitating secure information transfer across long distances.
\par The field has witnessed rapid advancements, particularly in the development of innovative materials and techniques that enhance the
performance of entanglement sources. Recent breakthroughs, including quantum repeaters and high-fidelity entanglement distribution methods,
have addressed challenges related to signal loss and fidelity in quantum communication networks.
Despite these advances, challenges remain in terms of scalability, resource efficiency, and operational reliability.
Many existing systems struggle to meet the demands for large-scale entanglement required for practical applications,
with issues such as room temperature operation and the complexity of quantum protocols posing significant hurdles.
The importance of entanglement sources in the realm of quantum technologies cannot be overstated,
as they serve as the foundation for cutting-edge innovations in the quantum information fields.
\par After one of the first \cite{Kwiat_1995} demonstrations of a high-intensity polarization entangled source was realized it also
became apparent that they can be fully done on chip \cite{S_G_S_C_F_B_L_G_B_2022} for frequency-bin entanglement,
polarization entangled \cite{L_Z_F_F_L_L_W_R_D_X_etal._2017} and also for hybrid frequency-polarization
entangled states \cite{F_R_D_F_L_M_A_B_D_2023}. Latest research in this field is advancing quickly, specifically for
Pulsed Laser (PL) sources which offer higher peak power and fewer synchronization constraints.
A benefit to using Continous Wave lasers (CW) as compared to PW is less maintainace and lower cost, for instance in an industrial or government
setting where access may be limited. Some notable mentions using a similar design with a Sagnac loop
\cite{Neumann_Buchner_Bulla_Bohmann_Ursin_2022_CW,Chen_Ecker_Wengerowsky_Bulla_Joshi_Steinlechner_Ursin_2018_CW}
in which the entanglement is generated due to an ambiguity of the origin of the photons.
There are also many linear, or single pass, designs such as \cite{Lee_Kim_Cha_Moon_2016,Kwiat_Mattle_Weinfurter_Zeilinger_Sergienko_Shih_1995}
where the entanglement is a product of ambiguity of momentum conservation, as only specific cross sections of the two generated SPDC light
cones are spatially indistinguishable.
\par An important measure on wether a source is performing well is its brightness, bandwidth, and heralding \cite{Bennink_2010,Ljunggren_Tengner_Marsden_Pelton_2006}.
The brightness being a measure of how many photon pairs are produced, bandwidth corresponds to how defined they are in frequency,
as this is a limiting factor for certain interference measurements and also for coupling to quantum memories,
and the heralding being the probability, when measuring two photon correlations, of finding a correlated photon when detecting the 1st one.
Brightness and heralding should be as high as possible in order to mitigate loss in fiber for fiber based networks,
reduce preprocessing load, and the bandwidth to be as narrow as needed for efficient coupling to quantum devices such as quantum memories or repeaters,
or for certain measurements as HBT and HOM, which will need to be performed for a full characterisation of the source.
% NOTE: What you want to do, how to do it, why important and how different from others
The advantage of using SPDC for entanglement generation is that one can relatively efficiently generate the necesarry biphoton pairs
compared to other methods. 

\section{Statement of hypotheses, research questions and research goals}
% (approximately one page)
% First, briefly recall the research problem.
% Next, clearly present the principal hypotheses (H) and/or research questions (R) and/or detailed research objectives (C).
% In most cases, just one of these three categories will be relevant. Clearly state which category it is; list the individual hypotheses (H1, H2, ..),
% research questions (R1, R2, ..) and/or detailed objectives (C1, C2, ..); and explain them briefly.
% In most PhD projects, the options of defining hypotheses or setting research questions are recommended.
% However, the possibility of stating detailed research objectives is permitted under the new University rules.
% Doctoral research projects are usually based on around three main research hypotheses,
% research questions or more detailed objectives, but this depends on the nature of the research.

% NOTE: What you want to do, how to do it, why important and how different from others

In the current state of the field there exist various sources which 
Research hypotheses (H), research questions (R), and research goals (C)

H1 - It is possible to build an enetanglement source which would be bright enough to supply the demand entangled photons with high fidelity to Bell States and
high value of tangle.

H2 - Be able to perform various quantum tests on specific combinations of DWDM channels and get satisfying results for the bandwidth which we can
produce.

C1 - Build a Sagnac source of entanglement and reach the currently know state of the art in performance metrics such as brightness and heralding. 
This goal is directly connected to H1 

% Introduce HBT, HOM, Teleportation, Entanglement Swapping, Zeillinger, Rainer, others,
C2 - Once we have a fully working source, we need to characterize it using QST, CHSH, HBT, and HOM for different pairs of DWDM channels (references). Directly 
% TODO Check frequency stabilization literature, doppler broadening, ..., PDH? Single pass, double pass, toptica homepage

C3 - might need to introduce extra filtering -- 100 GHz not enough -> Filtering cavities
Long-term stability, locking, ...? Demonstrate Quantum Teleportation within the lab, and Entanglement swapping (we will use the same $\Phi^+$ or $\Phi^-$ Bell state,
then perform a Bell measurement on each part of the pair) with IJS, IJS Reactor, Beyond Semiconductor, where to do
long distance stuff (need fibers) other distant parties,

C4* - Free space application using OAM modes, try to do entanglement in this regime?? topological photons
Put an SLM in one branch of the Sagnac and generate also OAM modes
Currently source creates Polarization and Frequency entanglement, adding the SLM thin film would make it a 3 for 1 source. A pulsed laser source would give
also the possibility of time bin entanglement.

% NOTE: Possibly put this in the future \ket{H;+;\omega_s}\ket{H;+;\omega_i}+\ket{V;-;\omega_s}\ket{V;-;\omega_i}

\newpage
\section{Outline of research and research methods}
% (approximately one page)
% • Briefly outline the research and the planned research methodology.
% • Provide a more detailed outline of the planned research and its methodology; for example, approaches, methods (e.g., theoretical, experimental, simulation), phases of research, if applicable, also the planned scheduling of the phases.


% NOTE: Why use CW instead of Pulsed Wave (PW)?
Why use CW instead of Pulsed Wave (PW)?
The focus of this thesis is to implement a Sagnac interferometer source of polarization-entangled photons centered around 1560 nm,
designed to be sufficiently broadband to accommodate multiple DWDM frequency channels.
These photons will be generated via SPDC \cite{jesseSPDC} from a
50 mm periodically poled lithium niobate (PPLN) nonlinear crystal located at the center of the interferometer.
The source will first be implemented and characterized in the laboratory through techniques such as QST,
CHSH inequality measurements \cite{Clauser_Horne_Shimony_Holt_1969},
HBT interferometry \cite{Brown_Twiss_1954}, and HOM interferometry \cite{Hong_Ou_Mandel_1987}.
Subsequently, quantum teleportation \cite{Bouwmeester_Pan_Mattle_Eibl_Weinfurter_Zeilinger_1997}
and entanglement swapping \cite{Jennewein_Weihs_Pan_Zeilinger_2001} experiments will be conducted in collaboration with the Jožef Stefan Institute,
% NOTE: Maybe write the group names here
which will develop an identical entanglement source as a part of the collaboration.
We will use a Continous Wave (CW) 780 nm Toptica DLPro laser to pump a 50 mm nonlinear PPLN crystal which is phase matched for Type-0 SPDC ($e_{pump} \rightarrow e_{signal} + e_{idler}$, e meaning extraordinary polarization)
for generating entangled photons.
An existing Dark Fiber (DF) will be used for network testing once all of the local tests have been made, including QST, HBT, and HOM.

The DFs location is currently undisclosed.

In the case of the current thesis I will use a 50 mm PPLN Type-0 SPDC crystal placed in a bulk Sagnac interferometer which will be bi-directionally
pumped by a CW 780,24 nm laser. The pump will be set to a diagonal state ($\frac{1}{\sqrt{2}}(\ket{H} + \ket{V})$). On ariving to the
PBS the beam is split into two. The reflected ($\ket{V}$) beam first passes through a halfwaveplate in order to rotate the polarization from $\ket{V}$ to $\ket{H}$
then through the crystal where it generates two $\ket{H}$ photons around 1560 nm, and then through the PBS $\ket{H}$ output where it gets combined
with the now two $\ket{V}$ photons from the counter propagating branch.

% NOTE: Somewhere when talking about the generated photons
To maximize the efficiency of these telecom networks for multiple users,
the available bandwidth is divided into many frequency channels using Dense Wavelength Division Multiplexing (DWDM).

% TODO: How to stabilize the 780 laser to the Rubidium Gas Cell, some references for locking and such which we plan to use
To ensure stability, the pump laser wavelength will be locked to an absorption line in Rubidium gas using atomic spectroscopy via the $^{87}Rb$ $D_2$ transition \cite{metger2017sas}.
The entanglement will be generated by bi-directionally pumping the PPLN crystal in the center of the Sagnac interferometer,
introducing photon indistinguishability essential for these protocols.
Advantages of Continuous Wave compared to Pulsed Laser entanglement sources:
% TODO: Talk about how to do this
How to then generate entanglement

% Active polarization control
The last part of the thesis will be about active polarization control via a closed loop algorithm such as neural networks.
Another important part of a Quantum Network is polarization control \cite{CCSHDCDRS}. In order to measure the correct states the
idea is to use an electronic polarization controller in the experimental network to create an algorithm which will be able to
ensure the correct polarization state is being received on the measurement stage.

\section{Expected results and original contributions to science}
% (approximately one side)
% • State the expected results of the research.
% • Clearly identify the expected original contributions to science deriving from the above results.
% (It is recommended to enumerate these contributions as an itemised list of usually up to five contributions.)
% • Briefly elaborate on the expected contributions. Since they are typically hypothetical at this stage, the contributions may be expressed in very broad or narrow terms, as appropriate.
% Original contributions include, e.g., original knowledge or findings, new theoretical models, original experimental or simulation methods, new approaches, newly opened areas, etc.
% Original contributions do not include, e.g., implementation of known methods, solutions of practical problems with known methods and similar.
% Developing and demonstrating methods of postprocessing for entanglement swapping from CW polarization entangled to be used in existing telecom infrastructure.
Depending on timing jitter how good the HOM would be, try to get as good as we can
Maybe need a compromise between integrating and stuff
Depending on visibility of the HOM this reduces the tangle of the source and whatnot
Check when visibility destroys entanglement

Original contributions: More engineering - filtering, jitter,
Getting good SNR with CW -> In future maybe go to PL - working principle may be the same and might bring great imrpovement of HOM

Currently making good progress with brightness - but needs improvement of coupling/heralding

Working together with a company (mention maybe the experimental network from SiQuid)

\section{Draft plan for management of research data}
% (Note the data management plan is a required part of the "dispozicija" from 1/10/2021.)
% (approximately half a page)
% • Outline a preliminary plan for the management of any research data that will be obtained or created during the doctoral project.
% • Declare in which data repositories research data is expected to be made available, how the data will be organised,
% etc. (* For more information, see Article 50 of the Rules on Doctoral Study at the University of Ljubljana since 2020, copied below,
% as well as other relevant published instructions of the University of Ljubljana.).

\newpage
\bibliographystyle{IEEEtran}
\bibliography{reference}
\end{document}
