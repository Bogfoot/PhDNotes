\documentclass{article}

%%%%%%%%%%%%%%%%%%%%%%%%%%%%%%%%%
% PACKAGE IMPORTS
%%%%%%%%%%%%%%%%%%%%%%%%%%%%%%%%%
\usepackage[framemethod=TikZ]{mdframed}
\usepackage{amsthm}
\usepackage{tikzsymbols}
\usepackage[tmargin=2.5cm,rmargin=2cm,lmargin=2.5cm,margin=0.85in,bmargin=2.5cm,footskip=.2in]{geometry}
\linespread{1.3}
\usepackage{amsmath,amsfonts,amsthm,amssymb,mathtools}
\usepackage[varbb]{newpxmath}
\usepackage{xfrac}
\usepackage[makeroom]{cancel}
\usepackage{bookmark}
\usepackage{enumitem}
\usepackage{hyperref,theoremref}
\hypersetup{
	pdftitle={Assignment},
	colorlinks=true, linkcolor=doc!90,
	bookmarksnumbered=true,
	bookmarksopen=true
}
\usepackage[most,many,breakable]{tcolorbox}
\usepackage{xcolor}
\usepackage{varwidth}
\usepackage{etoolbox}
\usepackage{nameref}
\usepackage{multicol,array}
\usepackage{tikz-cd}
\usepackage[ruled,vlined,linesnumbered]{algorithm2e}
\usepackage{comment} % enables the use of multi-line comments (\ifx \fi) 
\usepackage{xifthen}
\usepackage{pdfpages}
\usepackage{transparent}
\usepackage[bottom]{footmisc}
\usepackage[utf8]{inputenc} % allow utf-8 input
\usepackage{braket}
\usepackage{titling} % For customizing title
\usepackage{abstract} % For abstract formatting
\usepackage{wrapfig}			  %Kontekstom osjetljivo navođenje
\usepackage{csquotes}			  %Kontekstom osjetljivo navođenje
\usepackage{graphicx}       %Slike i slično
\usepackage{url}            % simple URL typesetting
\usepackage{booktabs}       % professional-quality tables
\usepackage{nicefrac}       % compact symbols for 1/2, etc.
\usepackage{float}
\usepackage{caption}
\usepackage{subcaption}
\usepackage{listings}
\usepackage[english]{babel}
\usepackage{titlesec}				%Za naslovnu stranicu
\usepackage[T1]{fontenc}
\usepackage{import}
\usepackage{svg}



\counterwithin{figure}{section}
\urlstyle{same}
\numberwithin{equation}{section}
\hyphenation{pergamon} % riječ u argumentu (pergamon) se ne rastavlja s crticom; ne smije imat specijalna slova. č. ž... - rastavljam ju naredbom \- (npr. išče\-zava)
\setlength\parindent{0pt} % u novom paragrafu: indent=0
\setlength{\parskip}{10pt} % postavlja željeni vertikalni razmak između paragrafa
\setlength{\skip\footins}{2cm} % razmak između glavnog teksta i fusnota
\renewcommand{\thefootnote}{$\ddagger$} % želim dagger za oznaku fusnota
\newcommand{\HRule}{\rule{\linewidth}{0.4mm}} % nova naredba za horizontalne linije na naslovnoj stranici
\newlength{\mylen}
\setcounter{secnumdepth}{4}
\titleformat{\paragraph}
{\normalfont\normalsize\bfseries}{\theparagraph}{1em}{}


%%%%%%%%%%%%%%%%%%%%%%%%%%%%%%%%%%%%%%%%%%%
% TABLE OF CONTENTS
%%%%%%%%%%%%%%%%%%%%%%%%%%%%%%%%%%%%%%%%%%%

\usepackage{tikz}
\definecolor{doc}{RGB}{0,60,110}
\usepackage{titletoc}
\contentsmargin{0cm}
\titlecontents{chapter}[3.7pc]
{\addvspace{30pt}
	\begin{tikzpicture}[remember picture, overlay]
		\draw[fill=doc!60,draw=doc!60] (-7,-.1) rectangle (-0.9,.5);
		\pgftext[left,x=-3.5cm,y=0.2cm]{\color{white}\Large\sc\bfseries Chapter\ \thecontentslabel};
	\end{tikzpicture}\color{doc!60}\large\sc\bfseries}
{}
{}
{\;\titlerule\;\large\sc\bfseries Page \thecontentspage
	\begin{tikzpicture}[remember picture, overlay]
		\draw[fill=doc!60,draw=doc!60] (2pt,0) rectangle (4,0.1pt);
	\end{tikzpicture}}
\titlecontents{section}[1.7pc]
{\addvspace{8pt}}
{\contentslabel[\thecontentslabel]{1pc}}
{}
{\ \hrulefill\ \small\thecontentspage}
[\vspace{-0.4cm}]
\titlecontents*{subsection}[3.7pc]
{\addvspace{-0pt}\small}
{}
{}
{\dotfill \small\thecontentspage\\}
[][\vspace{-1cm}]

\makeatletter
\renewcommand{\tableofcontents}{
	\chapter{
	  \vspace{-20\p@}
	  \begin{tikzpicture}[remember picture, overlay]
		  \pgftext[right,x=15cm,y=0.2cm]{\color{doc!60}\Huge\sc\bfseries \contentsname};
		  \draw[fill=doc!60,draw=doc!60] (13,-.75) rectangle (20,1);
		  \clip (13,-.75) rectangle (20,1);
		  \pgftext[right,x=15cm,y=0.2cm]{\color{white}\Huge\sc\bfseries \contentsname};
	  \end{tikzpicture}}
	\@starttoc{toc}}
\makeatother


%%%%%%%%%%%%
%% KUTIJE %%
%%%%%%%%%%%%
%%%%%%%%%%%%%%%%%%%%%%%%%%%%%%
%Theorem
\newcounter{teorem}[section] \setcounter{teorem}{0}
\renewcommand{\theteorem}{\arabic{section}.\arabic{teorem}}
\newenvironment{teorem}[2][]{%
	\refstepcounter{teorem}%
	\ifstrempty{#1}%
	{\mdfsetup{%
			frametitle={%
					\tikz[baseline=(current bounding box.east),outer sep=0pt]
					\node[anchor=east,rectangle,fill=blue!20]
					{\strut Teorem~\thetheo};}}
	}%
	{\mdfsetup{%
			frametitle={%
					\tikz[baseline=(current bounding box.east),outer sep=0pt]
					\node[anchor=east,rectangle,fill=blue!20]
					{\strut Teorem~\thetheo:~#1};}}%
	}%
	\mdfsetup{innertopmargin=10pt,linecolor=blue!20,%
		linewidth=2pt,topline=true,%
		frametitleaboveskip=\dimexpr-\ht\strutbox\relax
	}
	\begin{mdframed}[]\relax%
		\label{#2}}{\end{mdframed}}
%%%%%%%%%%%%%%%%%%%%%%%%%%%%%%
%Proof
\newcounter{dokaz}[section]\setcounter{dokaz}{0}
\renewcommand{\thedokaz}{\arabic{section}.\arabic{dokaz}}
\newenvironment{dokaz}[2][]{%
	\refstepcounter{dokaz}%
	\ifstrempty{#1}%
	{\mdfsetup{%
			frametitle={%
					\tikz[baseline=(current bounding box.east),outer sep=0pt]
					\node[anchor=east,rectangle,fill=red!20]
					{\strut Dokaz~\theprf};}}
	}%
	{\mdfsetup{%
			frametitle={%
					\tikz[baseline=(current bounding box.east),outer sep=0pt]
					\node[anchor=east,rectangle,fill=red!20]
					{\strut Dokaz~\theprf:~#1};}}%
	}%
	\mdfsetup{innertopmargin=10pt,linecolor=red!20,%
		linewidth=2pt,topline=true,%
		frametitleaboveskip=\dimexpr-\ht\strutbox\relax
	}
	\begin{mdframed}[]\relax%
		\label{#2}}{\qed\end{mdframed}}

%%%%%%%%%%%%%%%%%%%%%%%%%%%%%%
%Primjer
\newcounter{primjer}[section]\setcounter{primjer}{0}
\renewcommand{\theprimjer}{\arabic{section}.\arabic{primjer}}
\newenvironment{primjer}[2][]{%
	\refstepcounter{primjer}%
	\ifstrempty{#1}%
	{\mdfsetup{%
			frametitle={%
					\tikz[baseline=(current bounding box.east),outer sep=0pt]
					\node[anchor=east,rectangle,fill=red!20]
					{\strut Primjer~\theprf};}}
	}%
	{\mdfsetup{%
			frametitle={%
					\tikz[baseline=(current bounding box.east),outer sep=0pt]
					\node[anchor=east,rectangle,fill=red!20]
					{\strut Primjer~\theprf:~#1};}}%
	}%
	\mdfsetup{innertopmargin=10pt,linecolor=red!20,%
		linewidth=2pt,topline=true,%
		frametitleaboveskip=\dimexpr-\ht\strutbox\relax
	}
	\begin{mdframed}[]\relax%
		\label{#2}}{\qed\end{mdframed}}
%%%%%%%%%%%%%%%%%%%%%%%%%%%%%%
%================================
% NOTE BOX
%================================

\usetikzlibrary{arrows,calc,shadows.blur}
\tcbuselibrary{skins}
\newtcolorbox{Bilješka}[1][]{%
	enhanced jigsaw,
	colback=gray!20!white,%
	colframe=gray!80!black,
	size=small,
	boxrule=1pt,
	title=Bilješka,
	halign title=flush center,
	coltitle=black,
	breakable,
	drop shadow=black!50!white,
	attach boxed title to top left={xshift=1cm,yshift=-\tcboxedtitleheight/2,yshifttext=-\tcboxedtitleheight/2},
	minipage boxed title=1.5cm,
	boxed title style={%
			colback=white,
			size=fbox,
			boxrule=1pt,
			boxsep=2pt,
			underlay={%
					\coordinate (dotA) at ($(interior.west) + (-0.5pt,0)$);
					\coordinate (dotB) at ($(interior.east) + (0.5pt,0)$);
					\begin{scope}
						\clip (interior.north west) rectangle ([xshift=3ex]interior.east);
						\filldraw [white, blur shadow={shadow opacity=60, shadow yshift=-.75ex}, rounded corners=2pt] (interior.north west) rectangle (interior.south east);
					\end{scope}
					\begin{scope}[gray!80!black]
						\fill (dotA) circle (2pt);
						\fill (dotB) circle (2pt);
					\end{scope}
				},
		},
	#1,
}

\input{macros}
\input{letterfonts}


% Customizing title
\renewcommand{\maketitlehooka}{\centering}
\renewcommand{\maketitlehookb}{\vspace{-1.5em}}

% Customizing abstract
\renewcommand{\abstractnamefont}{\normalfont\large\bfseries}
\renewcommand{\abstracttextfont}{\normalfont\normalsize}


\graphicspath{{./Images/}} % Where to take images from

%%%%%%%%%%%%%%%%%%%%%%%%%% Define some useful colors %%%%%%%%%%%%%%%%%%%%%%%%%%
\definecolor{ocre}{RGB}{243,102,25}
\definecolor{mygray}{RGB}{243,243,244}
\definecolor{deepGreen}{RGB}{26,111,0}
\definecolor{shallowGreen}{RGB}{235,255,255}
\definecolor{deepBlue}{RGB}{61,124,222}
\definecolor{shallowBlue}{RGB}{235,249,255}
%%%%%%%%%%%%%%%%%%%%%%%%%%%%%%%%%%%%%%%%%%%%%%%%%%%%%%%%%%%%%%%%%%%%%%%%%%%%%%%

%%%%%%%%%%%%%%%%%%%%%%%%%% Define an orangebox command %%%%%%%%%%%%%%%%%%%%%%%%
\newcommand\orangebox[1]{\fcolorbox{ocre}{mygray}{\hspace{1em}#1\hspace{1em}}}
%%%%%%%%%%%%%%%%%%%%%%%%%%%%%%%%%%%%%%%%%%%%%%%%%%%%%%%%%%%%%%%%%%%%%%%%%%%%%%%

%%%%%%%%%%%%%%%%%%%%%%%%%%%% English Environments %%%%%%%%%%%%%%%%%%%%%%%%%%%%%
\newtheoremstyle{mytheoremstyle}{3pt}{3pt}{\normalfont}{0cm}{\rmfamily\bfseries}{}{1em}{{\color{black}\thmname{#1}~\thmnumber{#2}}\thmnote{\,--\,#3}}
\newtheoremstyle{myproblemstyle}{3pt}{3pt}{\normalfont}{0cm}{\rmfamily\bfseries}{}{1em}{{\color{black}\thmname{#1}~\thmnumber{#2}}\thmnote{\,--\,#3}}
\theoremstyle{mytheoremstyle}
\newmdtheoremenv[linewidth=1pt,backgroundcolor=shallowGreen,linecolor=deepGreen,leftmargin=0pt,innerleftmargin=20pt,innerrightmargin=20pt,]{theorem}{Theorem}[section]
\theoremstyle{mytheoremstyle}
\newmdtheoremenv[linewidth=1pt,backgroundcolor=shallowBlue,linecolor=deepBlue,leftmargin=0pt,innerleftmargin=20pt,innerrightmargin=20pt,]{definition}{Definition}[section]
\theoremstyle{myproblemstyle}
\newmdtheoremenv[linecolor=black,leftmargin=0pt,innerleftmargin=10pt,innerrightmargin=10pt,]{problem}{Problem}[section]
%%%%%%%%%%%%%%%%%%%%%%%%%%%%%%%%%%%%%%%%%%%%%%%%%%%%%%%%%%%%%%%%%%%%%%%%%%%%%%%
\begin{document}
%%%%%%%%%%%%%%%%%%%%%%%%%%%%%%% Title & Author %%%%%%%%%%%%%%%%%%%%%%%%%%%%%%%%
\begin{titlepage}
	\begin{center}
		{\LARGE \textbf{University of Ljubljana}} \\[0.1cm]
		{\LARGE \textbf{Faculty of Mathematics and Physics}} \\[3cm]
		{\large \textbf{Adrian Udovičić, mag. phys.}}\\[0.1cm]
		{\large \textbf{Rainer O. Kaltenbaek, asoc. prof. dr.}}\\[5cm]
		\vspace{2cm}
		{\LARGE \textbf{PhD topic proposal}} \\[0.1cm]
		{\LARGE \textbf{Generating and teleporting entanglement for quantum networks}} \\[4cm]
		{\large \textbf{Scientific field: Physics}} \\[0.1cm]
		\vfill
		\large\textbf{{Ljubljana, 2025\date{}}}
	\end{center}
\end{titlepage}
%%%%%%%%%%%%%%%%%%%%%%%%%%%%%%%%%%%%%%%%%%%%%%%%%%%%%%%%%%%%%%%%%%%%%%%%%%%%%%%
\pagenumbering{roman}
\newpage
\tableofcontents
\newpage
\pagenumbering{arabic}
\section{Description of the immediate research area and its problems}
% (approximately half a page)
%• State and briefly describe the immediate research area in which research will be conducted, and to which an original contribution is expected.
% Emphasise the importance and timeliness of the narrower field.
%• In a few sentences, describe the essence of the problem that will be tackled in the doctoral research and motivate it.
% (a more detailed description is required later in the "dispozicija").
\section{Objectives and the proposed research}
In the quickly developing fields of quantum communication and quantum computing there exists a great need for efficient transfer of secure quantum information. This can be done
by harnessing a quantum resource called entanglement in the form of quantum teleportation and entanglement swapping. Since doing these experiments over large distances through
fibers imposes a lot of loss, it is important to generate photons at wavelengths compatible with existing fiber network infrastructures \cite{Kimble_2008} as they have the least losses,
and possiby even in free space \cite{Kržić_et_al_2023} means for metropolitain network areas. Fiber networks are typically in the O and C near-infrared band (1310nm and 1550nm, respectively). For
efficient use of these telecom networks for many users, the available bandwith may need to be split into many frequency channels to enable dense wavelength division multiplexing (DWDM).
In the present thesis I will implement a Sagnac interferometer source of polarization entanglement for photons around 1560nm, that is sufficiently broadband
to fit into many DWDM frequency channels. The photons are generated via Spontanious Parametric Down-Conversion (SPDC) from a 50 mm periodically polled Lithium Niobate
nonlinear crystal in the center of the interferometer. The source is first implemented and characterized in the laboratory by means of quantum state tomography,
Clauser-Horne-Shimony-Holt \cite{Clauser_Horne_Shimony_Holt_1969} (CHSH)  measurement, Hansbury-Brown and Twiss \cite{Brown_Twiss_1954} (HBT) Interferometry, and Hong-Ou-Mandel Interferometry \cite{Hong_Ou_Mandel_1987} (HOMI).
Later, we will demonstrate quantum teleportation \cite{Bouwmeester_Pan_Mattle_Eibl_Weinfurter_Zeilinger_1997} and entanglement swapping \cite{Jennewein_Weihs_Pan_Zeilinger_2001} using the knowledge 
gained from the experiments mentioned above with our collegues at the Jozef Stefan Institute, who will build an identical source with our help.
The wavelength of the pump laser will be stabilized to an absorption line in a Rubidium gas cell using the $^{87}Rb$ $D_2$ transition via atomic spectroscopy locking (requires references).
These techniques are a prerequisite for quantum repeaters (references),
which may be essential to distribute entanglement over arbitrarily long distances in the future global quantum internet networks. Even the low losses (0.18 dB/km (references)) of the O and C
will grow exponentially with distance. This limits the efficient distribution of entanglement to only a few hundred kilometers (add some references).
The goal of this thesis will not only be the first realization of a high yield source of entanglement at non-degenerate frequencies in Slovenia but also quantum teleportation, and entanglement swapping.
The entanglement is generated by bi-directionally pumping the PPLN crystal in the center of the Sagnac interferometer, thus introducing indistinguishability between the photons.\\
\textbf{Key words: Quantum Entanglement, Quantum Key Distribution, Entanglement Swapping}

% TODO Check frequency stabilization literature, doppler broadening, ..., PDH? Single pass, double pass, toptica homepage

\section{Overview of related research and relevant literature}
% (approximately one page)
% • Briefly summarise the state-of-the-art. Review and briefly analyse the relevant literature including the most recent research in the area.
% • Explain how the proposed topic leads on from existing research, and outline the importance of the proposed line of research and the challenges it poses.
% • Explain the relevance and timeliness of the proposed research in the context of the literature reviewed above. 

First go over how to use SPDC for entanglement generation, advantages, disadvantages, types, ...

Progress up to 2000s, 2010s, 2020, 2024, present.

What is the work based on? Neumanns paper, introduce Bennik and Boyd for parameters,.... Introduce HBT, HOM, Teleportation, Entanglement Swapping, Zeillinger, Rainer, others,

How to then generate entanglement

How to stabilize the 780 laser to the Rubidium Gas Cell, some references for locking and such which we plan to use

\section{Research goals}
% Statement of hypotheses, research questions and research goals
% (approximately one page)
%• First, briefly recall the research problem. 
%• Next, clearly present the principal hypotheses (H) and/or research questions (R) and/or detailed research objectives (C). 
% In most cases, just one of these three categories will be relevant. Clearly state which category it is; list the individual hypotheses (H1, H2, ..),
% research questions (R1, R2, ..) and/or detailed objectives (C1, C2, ..); and explain them briefly.
% In most PhD projects, the options of defining hypotheses or setting research questions are recommended.
% However, the possibility of stating detailed research objectives is permitted under the new University rules.
% Doctoral research projects are usually based on around three main research hypotheses, research questions or more detailed objectives, but this depends on the nature of the research.

We will use a Continous Wave (CW) 780 nm Toptica DLPro laser to pump a 50 mm nonlinear PPLN crystal which is phase matched for Type-0 SPDC (e $\rightarrow$ e + e, e meaning extraordinary polarization)
for generating entangled photons.

Goal 0 - Build a Sagnac source of entanglement and reach the currently know state of the art in performance metrics such as brightness and heralding.

Goal 1 - Once we have a working source, we need to characterize it using Quantum State Tomography (QST), CHSH, HBT, and HOMI for different pairs of DWDM channels (references).

Goal 2 - might need to introduce extra filtering -- 100 GHz not enough -> Filtering cavities
Long-term stability, locking, ...? Demonstrate Quantum Teleportation within the lab, and Entanglement swapping (we will use the same $\Phi^+$ or $\Phi^-$ Bell state,
then perform a Bell measurement on each part of the pair) with IJS, IJS Reactor, Beyond Semiconductor, where to do
long distance stuff (need fibers) other distant parties,

Goal 3 - Free space application using OAM modes, try to do entanglement in this regime??
topological photons

Why use CW instead of Pulsed Wave (PW)?
Advantages of Continuous Wave compared to Pulsed Laser entanglement sources:

% \section{Description of the planned work}
\section{Outline of research and research methods}
% (approximately one page)
% • Briefly outline the research and the planned research methodology.
% • Provide a more detailed outline of the planned research and its methodology; for example, approaches, methods (e.g., theoretical, experimental, simulation), phases of research, if applicable, also the planned scheduling of the phases.
An existing Dark Fiber (DF) will be used for network testing once all of the local tests have been made, including QST, HBT, and HOMI. 
The DFs location is currently undisclosed.

\section{Expected results and original contributions to science}
% (approximately one side)
% • State the expected results of the research. 
% • Clearly identify the expected original contributions to science deriving from the above results.
% (It is recommended to enumerate these contributions as an itemised list of usually up to five contributions.)
% • Briefly elaborate on the expected contributions. Since they are typically hypothetical at this stage, the contributions may be expressed in very broad or narrow terms, as appropriate.
% Original contributions include, e.g., original knowledge or findings, new theoretical models, original experimental or simulation methods, new approaches, newly opened areas, etc.
% Original contributions do not include, e.g., implementation of known methods, solutions of practical problems with known methods and similar.
% Developing and demonstrating methods of postprocessing for entanglement swapping from CW polarization entangled to be used in existing telecom infrastructure.

\section{Contents and introduction}
Today I will be speaking to you about a part of my thesis, which is generating and teleporting entanglement for quantum networks. I will begin with a bit of theory,
where I will explain the basics of the Spontaneous Parametric Downconversion (SPDC). Here I will speak a bit about Phase Matching and how we try to optimize it in the lab. I will also
speak a bit about how we plan on Distributing Entanglement, so Quantum Teleportation and Entanglement Swapping.
Then I will speak about the present state that we are dealing with in the lab and how far we have gone so far.

\section{Motivation}
My project is in the scope of the SiQUID project. The goal of it is to demonstrate that a slovenian quantum network is possible to make,
train young researchers in the field of Quantum Technologies, as well as creating the sources of entanglement needed for entanglement based Quantum Key Distribution.
The main goal for me in this regard would be to create a bright source of polarization entanglemed photon pairs using a Sagnac interferometer.
The reason we chose a Sagnac in this case is due to its vibrational stability, and it's a convenient way to do so as it allows for a compact design.
A possible downside to this is that the alignment of this type of interferometer is a bit tedious, as you have to perfectly overlap two beams within the interferometer,
and in the crystal, and there is coupling between the various degrees of freedom that we have available to us.

\section{Theory}
\subsection{Spontaneous Parametric Downconversion}
Before talking about entanglement, we need to know how to generate photon pairs. A good way to do this is SPDC.
SPDC is a process in which a pump photon of frequency $\omega_p$ gets "downconverted" to two photons of lower frequency,
usually denoted as $\omega_s$ and $\omega_i$. These output photons can have the same frequency,
and thus this can be a degenerate process, or they can have different frequencies, being non-degenerate.
When we started this entire design process, we wanted to have degenerate frequencies, as we would be able to separate
the two photons easily based on their polarization. This design utilited a type-II process, meaning that we would be able
to separate the two photons on the output of the PBS. Now we have decided to go for a type-0 process, thus altering the design a bit.

\subsection{Phase Matching}
What is Phase Matching? Phase matching refferrs to fixing the relative phase between two or more frequencies
of light as they propagate through the crystal. Phase matching are the conditions at which some process will happen.
In the case of SPDC the Phase Matching conditions are energy conservation ($\hbar \omega_p = \hbar \omega_s + \hbar \omega_i$),
and momentum conservation ($k_p = k_s + k_i$). The efficiency of this process is also dictated by how well we satisfy these
two conditions - being slightly off from $\Delta k = 0$ could result in large losses of photon generation efficiency.
% DONE NOTE: Add a slide after Phase Matching slide regarding type-0,-I,-II. DONE
% DONE NOTE: Mention the crystal properties - anisotropic stuff DONE
% NOTE: Add for quasi phase matching

\subsection{Crystals}
An important step is to choose the correct material for the job. In this case, no ordinary material can be used,
as the phase matching condition $k_1 + k_2 = k_3 = \frac{n \omega_3}{c} = \frac{n \omega_1}{c} + \frac{n \omega_2}{c}$ is
impossible to satisfy in most cases as
$n_1 \left(	\omega_1 \right) < n_2 \left( \omega_2 \right) < n_3 \left(	\omega_3 \right)$ for $\omega_1 < \omega_2 < \omega_3$.
For this, we choose a birefringent material, which has a fast and a slow axis.

\par{Crystal Size}
Before periodic poling you could only have a limited size of the nonlinear material along the propagation direction.
This means that you would only be able to generate broad signals out of the crystals and at lower intensities, as the
intensity and bandwidth of the process goes proportionally to $I \propto \text{sinc}^2\left(\frac{L \Delta k}{2}\right)$.
Today we have periodic poling available to us which allows us to create oppositely oriented fields inside the
crystal. This allows for higher photon generation rates, longer crystals and more narrow bandwidths.

\subsection{Types of processes}
There are three different types of phase matching.
\begin{itemize}
	\item Type-0 is one in which a photon of some linear polarization, say $\ket{H}$ gets downconverted
into two photons of lower energies of the same polarization - o $\rightarrow$ o + o. This process used to be physically impossible to produce
as the phase matching condition $k_1 + k_2 = k_3 = \frac{n \omega_3}{c} = \frac{n \omega_1}{c} + \frac{n \omega_2}{c}$ is impossible to achieve
in most cases.
	\item Type-I is similar to type-0, except that the produced two photons are orthogonally polarized to the input photon - o $\rightarrow$ e + e
	\item In type-II instead you get two orthogonal polarizations out from the crystal. e $\rightarrow$ e + o
\end{itemize}

\subsection{Phase Matching Temperature}
The first important step is to determine the correct Phase Matching Temperature for our crystal. For different types of poling
periods these temperatures will vary, and in general, the lower the poling period the higher the Phase Matching Temperature.
The opposite seems to be true for the temperature bandwidth, which would be the acceptable temperature range for the material.
Similarly, one should not heat up certain crystals too much as this could destroy the Periodic Poling properties of the material.
For Periodically Poled Lithium Niobate (PPLN), the safe operating regime up to around 200 °C.

\subsection{Bandwidths}
Next would be the bandwidth of the process. In general, you would like it to be as narrow as possible - especially for certain
applications as entanglement swapping, and there are other methods to achieve this which I won't go into now. In general,
type-II processes are much more narrow than type-0 ones. In the plot I have on the slide, there's about a factor of 120 in
bandwidth difference. This is why Type-II was the primary choice for our group. Unfortunately, this also comes at the price
of reduced intensity. The difference in intensity was around 25 times lower compared to the type-0 crystal which led to us
using it in the end design.

\subsection{Different Designs}
First figure
Schematic of one method to produce and select the polarization-entangled state from the down-conversion crystal.
The extra birefringent crystals Cl and C2, along with the half wave plate HWPO, are used to compensate the birefringent
walk-off effects from the production crystal. By appropriately setting half wave plate HWP1 and quarter wave plate QWP1,
one can produce all four of the orthogonal EPR-Bell states. Each polarizer P1 and P2 consisted of two stacked polarizing
beam splitters preceded by a rotatable half wave plate.

Interesting thing is that here with the use of various compensating crystals and wave plates they can create any
Bell state.

2nd figure
Similar to 1, but doesn't require realignment, also much brighter.

3rd figure
Shows a use of noncollinear phase matching at room temperature (not really,
it's 32 °C), robust source of high brightness

4th figure
They present a source of polarization-entangled photon pairs based on time-reversed
Hong-Ou-Mandel interference. By superimposing four pair-creation possibilities on a polarization beam
splitter, pairs of identical photons are separated into two spatial modes without the usual requirement for
wavelength distinguishability or noncollinear emission angles. Our source yields high-fidelity polarization entanglement and high pair-generation rates without any requirement for active interferometric
stabilization, which makes it an ideal candidate for a variety of applications, in particular those requiring
indistinguishable photons.

\section{Entanglement}
If we have two systems, $\ket{\psi_i}$ and $\ket{\phi_i}$ described by two separate Hilbert spaces, the state of those two systems can be described as
a tensor product $\ket{\psi_i} \otimes \ket{\phi_i}$ of their state spaces. You can write this as a Schmidt decomposition: $\ket{\xi} = \sum_{i,j}
a_{ij} \ket{\psi_i}\ket{phi_i} \rightarrow \sum_i b_i \ket{\psi_i'} \ket{\phi_i'}$. If $b_i \ne 0$ and $b_{i \ne j} = 0$ the state is said to be separable.
If more than one $b_i \ne 0$ then $\ket{\xi}$ is said to be entangled, and the states can no longer be described without their comprising states.
An example of entangled states are Bell states, they are also maximally entangled:

\begin{center}
	\begin{aligned}
		\begin{equation}
			$\ket{\Psi^-} &= \frac{1}{\sqrt{2}} \left( \ket{H}\ket{V} - \ket{V}\ket{H} \right)$\\
			$\ket{\Psi^+} &= \frac{1}{\sqrt{2}} \left( \ket{H}\ket{V} + \ket{V}\ket{H} \right)$\\
			$\ket{\Phi^-} &= \frac{1}{\sqrt{2}} \left( \ket{H}\ket{H} - \ket{V}\ket{V} \right)$\\
			$\ket{\Phi^+} &= \frac{1}{\sqrt{2}} \left( \ket{H}\ket{H} + \ket{V}\ket{V} \right)$
		\end{equation}
	\label{eq:bsm}
	\end{aligned}
\end{center}

\subsection{Why do we care about entanglement?}
Entanglement sources have many applications. Some notable ones are Quantum Computation, Quantum Imaging,
and Quantum Sensing (to be explained). % TODO:
As we wish to distribute these entangled states over long distances, due to fiber losses it isn't viable to do so
over distances larger than a couple hundred kilometers.
\begin{center}
	\begin{table}[h]
		\caption{Relevant fiber loss. \textit{Source: Thorlabs}}
		\label{tab:fiberloss}
		\begin{tabular}{|c|c|c|c|c|c|c|}
			\hline
			$\lambda$ [nm] & 430 & 532 & 780 & 1310 & 1550 & 1900\\
			\hline
			Loss [dB/km] & 50 & 30 & 12 & 0.32 &  0.18 & 5\\
			\hline
		\end{tabular}
	\end{table}
\end{center}
\begin{exampleblock}{Example: Loss in fiber for 1550/1560 nm}
	200 km of fiber $\rightarrow$ -36 dB $\rightarrow$ $10^4$ loss.
	Start with 1 W, end up with 0.0001 W.
\end{exampleblock}
It would be desirable to have a more robust way to transport photons from A to B.
In our case, we generate a pair of $\ket{V}$ polarized photons in each of the branches of the Sagnac interferometer,
or in the case of type-II we get $\ket{H} + \ket{V}$, totalling to 4 photons being created "at the same time".
This leads us to the entangled state for type-II/0 SPDC:
\begin{equation*}
	\ket{\Psi_{p}} = \frac{1}{\sqrt{2}} ( a_{H}^{\dagger} ( \omega_p ) + a_{V}^{\dagger} ( \omega_p ) )\ket{0}\\
\end{equation*}
\begin{minipage}[l]{0.48\textwidth}
	\begin{equation*}
		\begin{aligned}
			\ket{\Psi_{\text{Type-2}}} &= \frac{1}{\sqrt{2}}(a_{H}^{\dagger}(\omega_s)a_{V}^{\dagger}(\omega_i)+\\
								&a_{V}^{\dagger}(\omega_i)a_{H}^{\dagger}(\omega_s))\ket{0}\\
		\end{aligned}
	\end{equation*}
\end{minipage}
\begin{minipage}[r]{0.48\textwidth}
\begin{equation*}
	\begin{aligned}
		\ket{\Psi_{\text{Type-0}}} &= \frac{1}{\sqrt{2}}(a_{H}^{\dagger}(\omega_s)a_{H}^{\dagger}(\omega_i)+\\
								   &a_{V}^{\dagger}(\omega_i)a_{V}^{\dagger}(\omega_s))\ket{0}
	\end{aligned}
\end{equation*}
\end{minipage}

\par as one branch of the phons will be rotated in polarization due to a wave plate being in one of the branches of the Sagnac interferometer.
This leads us to one of the Bell States mentioned above.
The reason why this is important is that if the conditions for this to happen are satisfied, you can no longer
predict the result of the polarization measurement. You can no longer tell which photon is which and from where it came from.
But the measurements will be perfectly anti-correlated in polarization.

\subsection{Distributing Entanglement}
Due to condisderable losses in fibers it is not feasible to transport entangled photons over distances greater than a few hundred kilometers. Thus
we must find a different solution to long distance Entanglement Dirstribution. This could be solved with quantum repeater, but currently one does not exist,
or is difficult to create. That being said, it is reasonable to establish an entanglement swapping network beforehand.

\subsection{Quantum Teleportation}
The basis of Quantum Teleportation is that the sender and receiver share an initial entangled pair -
in our case polarization entangled photons of the sort described above, in one of the 4 Bell States
so that the photons are just as likely to be horizontally or vertically polarized relative to the pump,
then perform a Bell State Measurement (BSM) on the receiving entangled
photon and an initial state which we would like to teleport to the receiver.
By entangling the initial state photon with one of the other entangled photons, and measuring the type of entanglement that they
share, we can then send that information to B through a classical signal change B so that it has the same polarization as X,
all without ever knowing their polarizations.

\begin{description}
	\item[BSM]
		A BSM is a coincidence measurement between different detectors. The simplest one would
		be the $\ket{\Psi^-}$ of \ref{eq:bsm} measurement. In it you only have a 50:50 beam splitter and two detectors. The only
		time both photons will fall on their own separate detector is when either both of them get transmitted,
		or both get reflected. This can be improved and the 2nd state can be measured if a PBS is added one
		of the branches after the BS. Thus we'd have a 50\% complete BSM. By adding another PBS in the other
		branch we can increase the completeness to 75\%. In order to reach a complete BSM, one must use nonlinear
		elements\footnote{PhysRevLett.86.1370}.
\end{description}

Then we report the result of the BSM via a classical communications channel to the receiver.
This same procedure does not work if the entangled pair does not exist.
The amazing thing is that this should work regardless of the distance between the entangled states,
and doesn't vialate causality, as the cassical message is being sent at most at the speed of causality to the receiver.

It is also important to notice that the Bell-state measurement does not reveal any information
on the properties of any of the particles. This is the very reason why
quantum teleportation using coherent two-particle superpositions works,
while any measurement on one-particle superpositions would fail.
The fact that no information whatsoever is gained on either particle is also the reason
why quantum teleportation escapes the verdict of the no-cloning theorem. After successful teleportation
particle 1 is not available in its original state any more,
and therefore particle 3 is not a clone but is really the result of teleportation.

\subsection{Entanglement Swapping}
The principle behind Entanglement swapping can be described in the exact same way as Quantum Teleportation. The only difference is now that the arbitrary
state which we're trying to teleport is part of an entangled pair. This means that if the same process of Quantum Teleportation is performed by Alice, Bob,
Charlie and Dora, where Alice and Bob share one entangled pair, and Charlie and Dora share another. Now the teleported state will be the state of
one of the entangled photons, thus swapping the entanglement between Charlie and Bob.

\section{Present State}
\subsection{Parameters}
\subsection{Building a Sagnac interferometer}

\bibliographystyle{IEEEtran}
\bibliography{reference}
\end{document}
