\documentclass{article}

%%%%%%%%%%%%%%%%%%%%%%%%%%%%%%%%%
% PACKAGE IMPORTS
%%%%%%%%%%%%%%%%%%%%%%%%%%%%%%%%%
\usepackage[framemethod=TikZ]{mdframed}
\usepackage{amsthm}
\usepackage{tikzsymbols}
\usepackage[tmargin=2.5cm,rmargin=2cm,lmargin=2.5cm,margin=0.85in,bmargin=2.5cm,footskip=.2in]{geometry}
\linespread{1.3}
\usepackage{amsmath,amsfonts,amsthm,amssymb,mathtools}
\usepackage[varbb]{newpxmath}
\usepackage{xfrac}
\usepackage[makeroom]{cancel}
\usepackage{bookmark}
\usepackage{enumitem}
\usepackage{hyperref,theoremref}
\hypersetup{
	pdftitle={Assignment},
	colorlinks=true, linkcolor=doc!90,
	bookmarksnumbered=true,
	bookmarksopen=true
}
\usepackage[most,many,breakable]{tcolorbox}
\usepackage{xcolor}
\usepackage{varwidth}
\usepackage{etoolbox}
\usepackage{nameref}
\usepackage{multicol,array}
\usepackage{tikz-cd}
\usepackage[ruled,vlined,linesnumbered]{algorithm2e}
\usepackage{comment} % enables the use of multi-line comments (\ifx \fi) 
\usepackage{import}
\usepackage{xifthen}
\usepackage{pdfpages}
\usepackage{transparent}
\usepackage[bottom]{footmisc}
\usepackage[utf8]{inputenc} % allow utf-8 input
\usepackage{braket}
\usepackage{wrapfig}			  %Kontekstom osjetljivo navođenje
\usepackage{csquotes}			  %Kontekstom osjetljivo navođenje
\usepackage{graphicx}       %Slike i slično
\usepackage[T1]{fontenc}    % use 8-bit T1 fonts
\usepackage{url}            % simple URL typesetting
\usepackage{booktabs}       % professional-quality tables
\usepackage{nicefrac}       % compact symbols for 1/2, etc.
\usepackage{float}
\usepackage{caption}
\usepackage{subcaption}
\usepackage{listings}
\usepackage[english,croatian]{babel}
\usepackage{titlesec}				%Za naslovnu stranicu

\counterwithin{figure}{section}
\urlstyle{same}
\numberwithin{equation}{section}
\hyphenation{pergamon} % riječ u argumentu (pergamon) se ne rastavlja s crticom; ne smije imat specijalna slova. č. ž... - rastavljam ju naredbom \- (npr. išče\-zava)
\setlength\parindent{0pt} % u novom paragrafu: indent=0
\setlength{\parskip}{10pt} % postavlja željeni vertikalni razmak između paragrafa
\setlength{\skip\footins}{2cm} % razmak između glavnog teksta i fusnota
\renewcommand{\thefootnote}{$\ddagger$} % želim dagger za oznaku fusnota
\newcommand{\HRule}{\rule{\linewidth}{0.4mm}} % nova naredba za horizontalne linije na naslovnoj stranici
\newlength{\mylen}
\setcounter{secnumdepth}{4}
\titleformat{\paragraph}
{\normalfont\normalsize\bfseries}{\theparagraph}{1em}{}


%%%%%%%%%%%%%%%%%%%%%%%%%%%%%%%%%%%%%%%%%%%
% TABLE OF CONTENTS
%%%%%%%%%%%%%%%%%%%%%%%%%%%%%%%%%%%%%%%%%%%

\usepackage{tikz}
\definecolor{doc}{RGB}{0,60,110}
\usepackage{titletoc}
\contentsmargin{0cm}
\titlecontents{chapter}[3.7pc]
{\addvspace{30pt}%
	\begin{tikzpicture}[remember picture, overlay]%
		\draw[fill=doc!60,draw=doc!60] (-7,-.1) rectangle (-0.9,.5);%
		\pgftext[left,x=-3.5cm,y=0.2cm]{\color{white}\Large\sc\bfseries Chapter\ \thecontentslabel};%
	\end{tikzpicture}\color{doc!60}\large\sc\bfseries}%
{}
{}
{\;\titlerule\;\large\sc\bfseries Page \thecontentspage
	\begin{tikzpicture}[remember picture, overlay]
		\draw[fill=doc!60,draw=doc!60] (2pt,0) rectangle (4,0.1pt);
	\end{tikzpicture}}%
\titlecontents{section}[3.7pc]
{\addvspace{2pt}}
{\contentslabel[\thecontentslabel]{2pc}}
{}
{\hfill\small \thecontentspage}
[]
\titlecontents*{subsection}[3.7pc]
{\addvspace{-1pt}\small}
{}
{}
{\ --- \small\thecontentspage}
[ \textbullet\ ][]

\makeatletter
\renewcommand{\tableofcontents}{%
	\chapter{%
	  \vspace{-20\p@}%
	  \begin{tikzpicture}[remember picture, overlay]%
		  \pgftext[right,x=15cm,y=0.2cm]{\color{doc!60}\Huge\sc\bfseries \contentsname};%
		  \draw[fill=doc!60,draw=doc!60] (13,-.75) rectangle (20,1);%
		  \clip (13,-.75) rectangle (20,1);
		  \pgftext[right,x=15cm,y=0.2cm]{\color{white}\Huge\sc\bfseries \contentsname};%
	  \end{tikzpicture}}%
	\@starttoc{toc}}
\makeatother


%%%%%%%%%%%%
%% KUTIJE %%
%%%%%%%%%%%%
%%%%%%%%%%%%%%%%%%%%%%%%%%%%%%
%Theorem
\newcounter{teorem}[section] \setcounter{teorem}{0}
\renewcommand{\theteorem}{\arabic{section}.\arabic{teorem}}
\newenvironment{teorem}[2][]{%
	\refstepcounter{teorem}%
	\ifstrempty{#1}%
	{\mdfsetup{%
			frametitle={%
					\tikz[baseline=(current bounding box.east),outer sep=0pt]
					\node[anchor=east,rectangle,fill=blue!20]
					{\strut Teorem~\thetheo};}}
	}%
	{\mdfsetup{%
			frametitle={%
					\tikz[baseline=(current bounding box.east),outer sep=0pt]
					\node[anchor=east,rectangle,fill=blue!20]
					{\strut Teorem~\thetheo:~#1};}}%
	}%
	\mdfsetup{innertopmargin=10pt,linecolor=blue!20,%
		linewidth=2pt,topline=true,%
		frametitleaboveskip=\dimexpr-\ht\strutbox\relax
	}
	\begin{mdframed}[]\relax%
		\label{#2}}{\end{mdframed}}
%%%%%%%%%%%%%%%%%%%%%%%%%%%%%%
%Proof
\newcounter{dokaz}[section]\setcounter{dokaz}{0}
\renewcommand{\thedokaz}{\arabic{section}.\arabic{dokaz}}
\newenvironment{dokaz}[2][]{%
	\refstepcounter{dokaz}%
	\ifstrempty{#1}%
	{\mdfsetup{%
			frametitle={%
					\tikz[baseline=(current bounding box.east),outer sep=0pt]
					\node[anchor=east,rectangle,fill=red!20]
					{\strut Dokaz~\theprf};}}
	}%
	{\mdfsetup{%
			frametitle={%
					\tikz[baseline=(current bounding box.east),outer sep=0pt]
					\node[anchor=east,rectangle,fill=red!20]
					{\strut Dokaz~\theprf:~#1};}}%
	}%
	\mdfsetup{innertopmargin=10pt,linecolor=red!20,%
		linewidth=2pt,topline=true,%
		frametitleaboveskip=\dimexpr-\ht\strutbox\relax
	}
	\begin{mdframed}[]\relax%
		\label{#2}}{\qed\end{mdframed}}
%%%%%%%%%%%%%%%%%%%%%%%%%%%%%%

\newcommand{\eps}{\epsilon}
\newcommand{\veps}{\varepsilon}
\newcommand{\ol}{\overline}
\newcommand{\ul}{\underline}
\newcommand{\wt}{\widetilde}
\newcommand{\wh}{\widehat}
\newcommand{\vocab}[1]{\textbf{\color{blue} #1}}
\providecommand{\half}{\frac{1}{2}}
\newcommand{\dang}{\measuredangle} %% Directed angle
\newcommand{\ray}[1]{\overrightarrow{#1}}
\newcommand{\seg}[1]{\overline{#1}}
\newcommand{\arc}[1]{\wideparen{#1}}
\DeclareMathOperator{\cis}{cis}
\DeclareMathOperator*{\lcm}{lcm}
\DeclareMathOperator*{\argmin}{arg min}
\DeclareMathOperator*{\argmax}{arg max}
\newcommand{\cycsum}{\sum_{\mathrm{cyc}}}
\newcommand{\symsum}{\sum_{\mathrm{sym}}}
\newcommand{\cycprod}{\prod_{\mathrm{cyc}}}
\newcommand{\symprod}{\prod_{\mathrm{sym}}}
\newcommand{\Qed}{\begin{flushright}\qed\end{flushright}}
\newcommand{\parinn}{\setlength{\parindent}{1cm}}
\newcommand{\parinf}{\setlength{\parindent}{0cm}}
% \newcommand{\norm}{\|\cdot\|}
\newcommand{\inorm}{\norm_{\infty}}
\newcommand{\opensets}{\{V_{\alpha}\}_{\alpha\in I}}
\newcommand{\oset}{V_{\alpha}}
\newcommand{\opset}[1]{V_{\alpha_{#1}}}
\newcommand{\lub}{\text{lub}}
\newcommand{\del}[2]{\frac{\partial #1}{\partial #2}}
\newcommand{\Del}[3]{\frac{\partial^{#1} #2}{\partial^{#1} #3}}
\newcommand{\deld}[2]{\dfrac{\partial #1}{\partial #2}}
\newcommand{\Deld}[3]{\dfrac{\partial^{#1} #2}{\partial^{#1} #3}}
\newcommand{\lm}{\lambda}
\newcommand{\uin}{\mathbin{\rotatebox[origin=c]{90}{$\in$}}}
\newcommand{\usubset}{\mathbin{\rotatebox[origin=c]{90}{$\subset$}}}
\newcommand{\lt}{\left}
\newcommand{\rt}{\right}
\newcommand{\bs}[1]{\boldsymbol{#1}}
\newcommand{\exs}{\exists}
\newcommand{\st}{\strut}
\newcommand{\dps}[1]{\displaystyle{#1}}

\newcommand{\sol}{\setlength{\parindent}{0cm}\textbf{\textit{Solution:}}\setlength{\parindent}{1cm} }
\newcommand{\solve}[1]{\setlength{\parindent}{0cm}\textbf{\textit{Solution: }}\setlength{\parindent}{1cm}#1 \Qed}

%From M275 "Topology" at SJSU
\newcommand{\id}{\mathrm{id}}
\newcommand{\taking}[1]{\xrightarrow{#1}}
\newcommand{\inv}{^{-1}}

%From M170 "Introduction to Graph Theory" at SJSU
\DeclareMathOperator{\diam}{diam}
\DeclareMathOperator{\ord}{ord}
\newcommand{\defeq}{\overset{\mathrm{def}}{=}}

%From the USAMO .tex files
\newcommand{\ts}{\textsuperscript}
\newcommand{\dg}{^\circ}
\newcommand{\ii}{\item}

% % From Math 55 and Math 145 at Harvard
% \newenvironment{subproof}[1][Proof]{%
% \begin{proof}[#1] \renewcommand{\qedsymbol}{$\blacksquare$}}%
% {\end{proof}}

\newcommand{\liff}{\leftrightarrow}
\newcommand{\lthen}{\rightarrow}
\newcommand{\opname}{\operatorname}
\newcommand{\surjto}{\twoheadrightarrow}
\newcommand{\injto}{\hookrightarrow}
\newcommand{\On}{\mathrm{On}} % ordinals
\DeclareMathOperator{\img}{im} % Image
\DeclareMathOperator{\Img}{Im} % Image
\DeclareMathOperator{\coker}{coker} % Cokernel
\DeclareMathOperator{\Coker}{Coker} % Cokernel
\DeclareMathOperator{\Ker}{Ker} % Kernel
\DeclareMathOperator{\rank}{rank}
\DeclareMathOperator{\Spec}{Spec} % spectrum
\DeclareMathOperator{\Tr}{Tr} % trace
\DeclareMathOperator{\pr}{pr} % projection
\DeclareMathOperator{\ext}{ext} % extension
\DeclareMathOperator{\pred}{pred} % predecessor
\DeclareMathOperator{\dom}{dom} % domain
\DeclareMathOperator{\ran}{ran} % range
\DeclareMathOperator{\Hom}{Hom} % homomorphism
\DeclareMathOperator{\Mor}{Mor} % morphisms
\DeclareMathOperator{\End}{End} % endomorphism

% Things Lie
\newcommand{\kb}{\mathfrak b}
\newcommand{\kg}{\mathfrak g}
\newcommand{\kh}{\mathfrak h}
\newcommand{\kn}{\mathfrak n}
\newcommand{\ku}{\mathfrak u}
\newcommand{\kz}{\mathfrak z}
\DeclareMathOperator{\Ext}{Ext} % Ext functor
\DeclareMathOperator{\Tor}{Tor} % Tor functor
\newcommand{\gl}{\opname{\mathfrak{gl}}} % frak gl group
\renewcommand{\sl}{\opname{\mathfrak{sl}}} % frak sl group chktex 6

% More script letters etc.
\newcommand{\SA}{\mathcal A}
\newcommand{\SB}{\mathcal B}
\newcommand{\SC}{\mathcal C}
\newcommand{\SF}{\mathcal F}
\newcommand{\SG}{\mathcal G}
\newcommand{\SH}{\mathcal H}
\newcommand{\OO}{\mathcal O}

\newcommand{\SCA}{\mathscr A}
\newcommand{\SCB}{\mathscr B}
\newcommand{\SCC}{\mathscr C}
\newcommand{\SCD}{\mathscr D}
\newcommand{\SCE}{\mathscr E}
\newcommand{\SCF}{\mathscr F}
\newcommand{\SCG}{\mathscr G}
\newcommand{\SCH}{\mathscr H}

% Mathfrak primes
\newcommand{\km}{\mathfrak m}
\newcommand{\kp}{\mathfrak p}
\newcommand{\kq}{\mathfrak q}

% number sets
\newcommand{\RR}[1][]{\ensuremath{\ifstrempty{#1}{\mathbb{R}}{\mathbb{R}^{#1}}}}
\newcommand{\NN}[1][]{\ensuremath{\ifstrempty{#1}{\mathbb{N}}{\mathbb{N}^{#1}}}}
\newcommand{\ZZ}[1][]{\ensuremath{\ifstrempty{#1}{\mathbb{Z}}{\mathbb{Z}^{#1}}}}
\newcommand{\QQ}[1][]{\ensuremath{\ifstrempty{#1}{\mathbb{Q}}{\mathbb{Q}^{#1}}}}
\newcommand{\CC}[1][]{\ensuremath{\ifstrempty{#1}{\mathbb{C}}{\mathbb{C}^{#1}}}}
\newcommand{\PP}[1][]{\ensuremath{\ifstrempty{#1}{\mathbb{P}}{\mathbb{P}^{#1}}}}
\newcommand{\HH}[1][]{\ensuremath{\ifstrempty{#1}{\mathbb{H}}{\mathbb{H}^{#1}}}}
\newcommand{\FF}[1][]{\ensuremath{\ifstrempty{#1}{\mathbb{F}}{\mathbb{F}^{#1}}}}
% expected value
\newcommand{\EE}{\ensuremath{\mathbb{E}}}
\newcommand{\charin}{\text{ char }}
\DeclareMathOperator{\sign}{sign}
\DeclareMathOperator{\Aut}{Aut}
\DeclareMathOperator{\Inn}{Inn}
\DeclareMathOperator{\Syl}{Syl}
\DeclareMathOperator{\Gal}{Gal}
\DeclareMathOperator{\GL}{GL} % General linear group
\DeclareMathOperator{\SL}{SL} % Special linear group

%---------------------------------------
% BlackBoard Math Fonts :-
%---------------------------------------

%Captital Letters
\newcommand{\bbA}{\mathbb{A}}	\newcommand{\bbB}{\mathbb{B}}
\newcommand{\bbC}{\mathbb{C}}	\newcommand{\bbD}{\mathbb{D}}
\newcommand{\bbE}{\mathbb{E}}	\newcommand{\bbF}{\mathbb{F}}
\newcommand{\bbG}{\mathbb{G}}	\newcommand{\bbH}{\mathbb{H}}
\newcommand{\bbI}{\mathbb{I}}	\newcommand{\bbJ}{\mathbb{J}}
\newcommand{\bbK}{\mathbb{K}}	\newcommand{\bbL}{\mathbb{L}}
\newcommand{\bbM}{\mathbb{M}}	\newcommand{\bbN}{\mathbb{N}}
\newcommand{\bbO}{\mathbb{O}}	\newcommand{\bbP}{\mathbb{P}}
\newcommand{\bbQ}{\mathbb{Q}}	\newcommand{\bbR}{\mathbb{R}}
\newcommand{\bbS}{\mathbb{S}}	\newcommand{\bbT}{\mathbb{T}}
\newcommand{\bbU}{\mathbb{U}}	\newcommand{\bbV}{\mathbb{V}}
\newcommand{\bbW}{\mathbb{W}}	\newcommand{\bbX}{\mathbb{X}}
\newcommand{\bbY}{\mathbb{Y}}	\newcommand{\bbZ}{\mathbb{Z}}

%---------------------------------------
% MathCal Fonts :-
%---------------------------------------

%Captital Letters
\newcommand{\mcA}{\mathcal{A}}	\newcommand{\mcB}{\mathcal{B}}
\newcommand{\mcC}{\mathcal{C}}	\newcommand{\mcD}{\mathcal{D}}
\newcommand{\mcE}{\mathcal{E}}	\newcommand{\mcF}{\mathcal{F}}
\newcommand{\mcG}{\mathcal{G}}	\newcommand{\mcH}{\mathcal{H}}
\newcommand{\mcI}{\mathcal{I}}	\newcommand{\mcJ}{\mathcal{J}}
\newcommand{\mcK}{\mathcal{K}}	\newcommand{\mcL}{\mathcal{L}}
\newcommand{\mcM}{\mathcal{M}}	\newcommand{\mcN}{\mathcal{N}}
\newcommand{\mcO}{\mathcal{O}}	\newcommand{\mcP}{\mathcal{P}}
\newcommand{\mcQ}{\mathcal{Q}}	\newcommand{\mcR}{\mathcal{R}}
\newcommand{\mcS}{\mathcal{S}}	\newcommand{\mcT}{\mathcal{T}}
\newcommand{\mcU}{\mathcal{U}}	\newcommand{\mcV}{\mathcal{V}}
\newcommand{\mcW}{\mathcal{W}}	\newcommand{\mcX}{\mathcal{X}}
\newcommand{\mcY}{\mathcal{Y}}	\newcommand{\mcZ}{\mathcal{Z}}



%---------------------------------------
% Bold Math Fonts :-
%---------------------------------------

%Captital Letters
\newcommand{\bmA}{\boldsymbol{A}}	\newcommand{\bmB}{\boldsymbol{B}}
\newcommand{\bmC}{\boldsymbol{C}}	\newcommand{\bmD}{\boldsymbol{D}}
\newcommand{\bmE}{\boldsymbol{E}}	\newcommand{\bmF}{\boldsymbol{F}}
\newcommand{\bmG}{\boldsymbol{G}}	\newcommand{\bmH}{\boldsymbol{H}}
\newcommand{\bmI}{\boldsymbol{I}}	\newcommand{\bmJ}{\boldsymbol{J}}
\newcommand{\bmK}{\boldsymbol{K}}	\newcommand{\bmL}{\boldsymbol{L}}
\newcommand{\bmM}{\boldsymbol{M}}	\newcommand{\bmN}{\boldsymbol{N}}
\newcommand{\bmO}{\boldsymbol{O}}	\newcommand{\bmP}{\boldsymbol{P}}
\newcommand{\bmQ}{\boldsymbol{Q}}	\newcommand{\bmR}{\boldsymbol{R}}
\newcommand{\bmS}{\boldsymbol{S}}	\newcommand{\bmT}{\boldsymbol{T}}
\newcommand{\bmU}{\boldsymbol{U}}	\newcommand{\bmV}{\boldsymbol{V}}
\newcommand{\bmW}{\boldsymbol{W}}	\newcommand{\bmX}{\boldsymbol{X}}
\newcommand{\bmY}{\boldsymbol{Y}}	\newcommand{\bmZ}{\boldsymbol{Z}}
%Small Letters
\newcommand{\bma}{\boldsymbol{a}}	\newcommand{\bmb}{\boldsymbol{b}}
\newcommand{\bmc}{\boldsymbol{c}}	\newcommand{\bmd}{\boldsymbol{d}}
\newcommand{\bme}{\boldsymbol{e}}	\newcommand{\bmf}{\boldsymbol{f}}
\newcommand{\bmg}{\boldsymbol{g}}	\newcommand{\bmh}{\boldsymbol{h}}
\newcommand{\bmi}{\boldsymbol{i}}	\newcommand{\bmj}{\boldsymbol{j}}
\newcommand{\bmk}{\boldsymbol{k}}	\newcommand{\bml}{\boldsymbol{l}}
\newcommand{\bmm}{\boldsymbol{m}}	\newcommand{\bmn}{\boldsymbol{n}}
\newcommand{\bmo}{\boldsymbol{o}}	\newcommand{\bmp}{\boldsymbol{p}}
\newcommand{\bmq}{\boldsymbol{q}}	\newcommand{\bmr}{\boldsymbol{r}}
\newcommand{\bms}{\boldsymbol{s}}	\newcommand{\bmt}{\boldsymbol{t}}
\newcommand{\bmu}{\boldsymbol{u}}	\newcommand{\bmv}{\boldsymbol{v}}
\newcommand{\bmw}{\boldsymbol{w}}	\newcommand{\bmx}{\boldsymbol{x}}
\newcommand{\bmy}{\boldsymbol{y}}	\newcommand{\bmz}{\boldsymbol{z}}

%---------------------------------------
% Scr Math Fonts :-
%---------------------------------------

\newcommand{\sA}{{\mathscr{A}}}   \newcommand{\sB}{{\mathscr{B}}}
\newcommand{\sC}{{\mathscr{C}}}   \newcommand{\sD}{{\mathscr{D}}}
\newcommand{\sE}{{\mathscr{E}}}   \newcommand{\sF}{{\mathscr{F}}}
\newcommand{\sG}{{\mathscr{G}}}   \newcommand{\sH}{{\mathscr{H}}}
\newcommand{\sI}{{\mathscr{I}}}   \newcommand{\sJ}{{\mathscr{J}}}
\newcommand{\sK}{{\mathscr{K}}}   \newcommand{\sL}{{\mathscr{L}}}
\newcommand{\sM}{{\mathscr{M}}}   \newcommand{\sN}{{\mathscr{N}}}
\newcommand{\sO}{{\mathscr{O}}}   \newcommand{\sP}{{\mathscr{P}}}
\newcommand{\sQ}{{\mathscr{Q}}}   \newcommand{\sR}{{\mathscr{R}}}
\newcommand{\sS}{{\mathscr{S}}}   \newcommand{\sT}{{\mathscr{T}}}
\newcommand{\sU}{{\mathscr{U}}}   \newcommand{\sV}{{\mathscr{V}}}
\newcommand{\sW}{{\mathscr{W}}}   \newcommand{\sX}{{\mathscr{X}}}
\newcommand{\sY}{{\mathscr{Y}}}   \newcommand{\sZ}{{\mathscr{Z}}}


%---------------------------------------
% Math Fraktur Font
%---------------------------------------

%Captital Letters
\newcommand{\mfA}{\mathfrak{A}}	\newcommand{\mfB}{\mathfrak{B}}
\newcommand{\mfC}{\mathfrak{C}}	\newcommand{\mfD}{\mathfrak{D}}
\newcommand{\mfE}{\mathfrak{E}}	\newcommand{\mfF}{\mathfrak{F}}
\newcommand{\mfG}{\mathfrak{G}}	\newcommand{\mfH}{\mathfrak{H}}
\newcommand{\mfI}{\mathfrak{I}}	\newcommand{\mfJ}{\mathfrak{J}}
\newcommand{\mfK}{\mathfrak{K}}	\newcommand{\mfL}{\mathfrak{L}}
\newcommand{\mfM}{\mathfrak{M}}	\newcommand{\mfN}{\mathfrak{N}}
\newcommand{\mfO}{\mathfrak{O}}	\newcommand{\mfP}{\mathfrak{P}}
\newcommand{\mfQ}{\mathfrak{Q}}	\newcommand{\mfR}{\mathfrak{R}}
\newcommand{\mfS}{\mathfrak{S}}	\newcommand{\mfT}{\mathfrak{T}}
\newcommand{\mfU}{\mathfrak{U}}	\newcommand{\mfV}{\mathfrak{V}}
\newcommand{\mfW}{\mathfrak{W}}	\newcommand{\mfX}{\mathfrak{X}}
\newcommand{\mfY}{\mathfrak{Y}}	\newcommand{\mfZ}{\mathfrak{Z}}
%Small Letters
\newcommand{\mfa}{\mathfrak{a}}	\newcommand{\mfb}{\mathfrak{b}}
\newcommand{\mfc}{\mathfrak{c}}	\newcommand{\mfd}{\mathfrak{d}}
\newcommand{\mfe}{\mathfrak{e}}	\newcommand{\mff}{\mathfrak{f}}
\newcommand{\mfg}{\mathfrak{g}}	\newcommand{\mfh}{\mathfrak{h}}
\newcommand{\mfi}{\mathfrak{i}}	\newcommand{\mfj}{\mathfrak{j}}
\newcommand{\mfk}{\mathfrak{k}}	\newcommand{\mfl}{\mathfrak{l}}
\newcommand{\mfm}{\mathfrak{m}}	\newcommand{\mfn}{\mathfrak{n}}
\newcommand{\mfo}{\mathfrak{o}}	\newcommand{\mfp}{\mathfrak{p}}
\newcommand{\mfq}{\mathfrak{q}}	\newcommand{\mfr}{\mathfrak{r}}
\newcommand{\mfs}{\mathfrak{s}}	\newcommand{\mft}{\mathfrak{t}}
\newcommand{\mfu}{\mathfrak{u}}	\newcommand{\mfv}{\mathfrak{v}}
\newcommand{\mfw}{\mathfrak{w}}	\newcommand{\mfx}{\mathfrak{x}}
\newcommand{\mfy}{\mathfrak{y}}	\newcommand{\mfz}{\mathfrak{z}}


\usepackage{lipsum} % For generating dummy text
\usepackage{titling} % For customizing title
\usepackage{abstract} % For abstract formatting
\usepackage{braket}

% Customizing title
\renewcommand{\maketitlehooka}{\centering}
\renewcommand{\maketitlehookb}{\vspace{-1.5em}}

% Customizing abstract
\renewcommand{\abstractnamefont}{\normalfont\large\bfseries}
\renewcommand{\abstracttextfont}{\normalfont\normalsize}


\graphicspath{{./Images/}} % Where to take images from

%%%%%%%%%%%%%%%%%%%%%%%%%% Define some useful colors %%%%%%%%%%%%%%%%%%%%%%%%%%
\definecolor{ocre}{RGB}{243,102,25}
\definecolor{mygray}{RGB}{243,243,244}
\definecolor{deepGreen}{RGB}{26,111,0}
\definecolor{shallowGreen}{RGB}{235,255,255}
\definecolor{deepBlue}{RGB}{61,124,222}
\definecolor{shallowBlue}{RGB}{235,249,255}
%%%%%%%%%%%%%%%%%%%%%%%%%%%%%%%%%%%%%%%%%%%%%%%%%%%%%%%%%%%%%%%%%%%%%%%%%%%%%%%

%%%%%%%%%%%%%%%%%%%%%%%%%% Define an orangebox command %%%%%%%%%%%%%%%%%%%%%%%%
\newcommand\orangebox[1]{\fcolorbox{ocre}{mygray}{\hspace{1em}#1\hspace{1em}}}
%%%%%%%%%%%%%%%%%%%%%%%%%%%%%%%%%%%%%%%%%%%%%%%%%%%%%%%%%%%%%%%%%%%%%%%%%%%%%%%

%%%%%%%%%%%%%%%%%%%%%%%%%%%% English Environments %%%%%%%%%%%%%%%%%%%%%%%%%%%%%
\newtheoremstyle{mytheoremstyle}{3pt}{3pt}{\normalfont}{0cm}{\rmfamily\bfseries}{}{1em}{{\color{black}\thmname{#1}~\thmnumber{#2}}\thmnote{\,--\,#3}}
\newtheoremstyle{myproblemstyle}{3pt}{3pt}{\normalfont}{0cm}{\rmfamily\bfseries}{}{1em}{{\color{black}\thmname{#1}~\thmnumber{#2}}\thmnote{\,--\,#3}}
\theoremstyle{mytheoremstyle}
\newmdtheoremenv[linewidth=1pt,backgroundcolor=shallowGreen,linecolor=deepGreen,leftmargin=0pt,innerleftmargin=20pt,innerrightmargin=20pt,]{theorem}{Theorem}[section]
\theoremstyle{mytheoremstyle}
\newmdtheoremenv[linewidth=1pt,backgroundcolor=shallowBlue,linecolor=deepBlue,leftmargin=0pt,innerleftmargin=20pt,innerrightmargin=20pt,]{definition}{Definition}[section]
\theoremstyle{myproblemstyle}
\newmdtheoremenv[linecolor=black,leftmargin=0pt,innerleftmargin=10pt,innerrightmargin=10pt,]{problem}{Problem}[section]
%%%%%%%%%%%%%%%%%%%%%%%%%%%%%%%%%%%%%%%%%%%%%%%%%%%%%%%%%%%%%%%%%%%%%%%%%%%%%%%

%%%%%%%%%%%%%%%%%%%%%%%%%%%%%%% Title & Author %%%%%%%%%%%%%%%%%%%%%%%%%%%%%%%%
% \title{Generating and teleporting entanglement for quantum networks}
% \author{Adrian Udovičić, mag. phys\\dr. soc. Rainer O. Kaltenbaek}
%%%%%%%%%%%%%%%%%%%%%%%%%%%%%%%%%%%%%%%%%%%%%%%%%%%%%%%%%%%%%%%%%%%%%%%%%%%%%%%


\begin{document}

\begin{titlepage}
	\begin{center}
		{\LARGE \textbf{University of Ljubljana}} \\[0.1cm]
		{\LARGE \textbf{Faculty of Mathematics and Physics}} \\[3cm]
		{\large \textbf{Adrian Udovičić, mag. phys.}}\\[0.1cm]
		{\large \textbf{Rainer O. Kaltenbaek, asoc. prof. dr.}}\\[5cm]
		\vspace{2cm}
		{\LARGE \textbf{PhD topic proposal}} \\[0.1cm]
		{\LARGE \textbf{Generating and teleporting entanglement for quantum networks}} \\[4cm]
		{\large \textbf{Field of study: Physics}} \\[0.1cm]
		\vfill
		\large\textbf{{Ljubljana, 2025\date{}}}
	\end{center}
\end{titlepage}
\maketitle



\newpage
\tableofcontents
\newpage

% \begin{abstract}
% Entanglement is a key resource of future quantum technologies. For that reason, it will be essential to distribute 
% it in quantum networks between many and possibly very distant communication parties. To this end, it is essential 
% to generate the photons at a wavelength that is compatible with existing fiber network infrastructure. Such networks typically feature 
% very low loss for photons in the O and C band (1310nm and 1550nm, respectively). To more 
% efficiently use telecom fibers for many users, the available bandwidth is split into frequency windows to enable dense wavelength 
% division multiplexing (DWDM). In the present thesis, we will implement a Sagnac source of entanglement for photons around 1560nm 
% that is sufficiently narrowband for the entangled photons to fit into specific DWDM frequency channels. To generate the 
% entangled photons, we will use a 50 mm long nonlinear crystal inside a Sagnac interferometer. We will first 
% implement and characterize this source in our laboratory and later use it for demonstrating entanglement distribution over in an existing 
% fiber network. The wavelength of the pump laser will be stabilized to an absorption line in a Rubidium gas cell. 
% With our help, an identical source will be set up by partners at the Jozef Stefan Institute. This will allow us to demonstrate the teleportation of entanglement 
% (entanglement swapping) by performing a Bell-state measurement on two entangled photons from those two independent and distant sources. This technique is a prerequisite for quantum 
% repeaters, which will be essential to distribute entanglement over arbitrary long distances in future global quantum networks. In particular, even the low losses of photons in 
% the C band will exponentially grow with the distance. This limits the efficient distribution of entanglement to distances of a few hundred kilometers.
% The present work will not only feature the first realization of a source of entanglement in Slovenia but also the first realization of teleportation.\\
% \end{abstract}

% \begin{figure}[h]
% 	\begin{center}
% 		\includegraphics[width=0.95\textwidth]{~/Notes/PhDNotes/PhDTopicDefenseAbstract/Seminar/Presentation/Images/SagnacWithSomeAddedColoursV3.png}
% 	\end{center}
% 	\caption{}\label{fig:}
% \end{figure}

\section{Objectives and the proposed research}
In the quickly developing fields of quantum communication and quantum computing there exists a great need for efficient transfer of secure quantum information. This can be done
by harnessing a key quantum resource, entanglement, in the form of quantum teleportation and entanglement swapping. To these ends, it is important to generate photons at wavelengths
compatible with existing fiber network infrastructures (references) and free air (references for OAM) means for metropolitain network areas. Fiber networks are typically in the O and C near-infrared band (1310nm and 1550nm, respectively). For
efficient use of these telecom networks for many users, the available bandwith needs to be split into many frequency windows to enable dense wavelength division multiplexing (DWDM).
In the present thesis I will implement a Sagnac interferometer source of polarization entanglement and optical angular momentum for photons around 1560nm, that is sufficiently narrow
to fit into specific DWDM frequency channels. The photons are generated using a 50 mm nonlinear periodically polled Lithium Niobate (PPLN, requires references) crystal in the center of the interferometer.
The source is first implemented and characterized in the laboratory by means of quantum state tomography, Clauser-Horne-Shimony-Holt measurement, Hansbury-Brown and Twiss Interferometry, and Hong-Ou-Mandel Interferometry.
Later, we will demonstrate quantum teleportation and entanglement swapping using the knowledge gained from the experiments mentioned above with our collegues at the Jozef Stefan Institute, who will build
an identical source with our help. The wavelength of the pump laser will be stabilized to an absorption line in a Rubidium gas cell (requires references). These techniques are a prerequisite for quantum repeaters (references),
which may be essential to distribute entanglement over arbitrarily long distances in the future global quantum internet networks. Even the low losses (0.18 dB/km (references)) of the O and C will grow exponentially with distance.
This limits the efficient distribution of entanglement to only a few hundred kilometers.
The goal of this thesis will not only be the first realization of a high yield source of entanglement at non-degenerate frequencies in Slovenia but also quantum teleportation, and entanglement swapping.
The entanglement is generated by bi-directionally pumping the PPLN crystal in the center of the Sagnac interferometer, thus introducing indistinguishability between the photons.\\
\textbf{Key words: Quantum Entanglement, Quantum Key Distribution, Entanglement Swapping}

\section{Survery of the literature}
First go over how to use SPDC for entanglement generation, advantages, disadvantages, types, ... 

Progress up to 2000s, 2010s, 2020, 2024, present. 

What is the work based on? Neumanns paper, introduce Bennik and Boyd for parameters,.... Introduce HBT, HOM, Teleportation, Entanglement Swapping, Zeillinger, Rainer, others,

How to then generate entanglement

How to stabilize the 780 laser to the Rubidium Gas Cell, some references for locking and such which we plan to use

\section{Research goals}

Goal 1 - Build the Sagnac source of entanglement and characterize it using QST, CHSH, HBT, and HOMI for different pairs of DWDM channels (references).

Goal 2 - Demonstrate Quantum Teleportation within the lab, and Entanglement swapping with IJS, other distant parties.

Goal 3 - Long-term stability, locking, ...?

Goal 4 - Free space application using OAM modes, try to do entanglement in this regime??

\section{Description of the planned work}

\section{Expected results}

\section{Contribution to the field}


\section{Contents and introduction}
Today I will be speaking to you about a part of my thesis, which is generating and teleporting entanglement for quantum networks. I will begin with a bit of theory,
where I will explain the basics of the Spontaneous Parametric Downconversion (SPDC). Here I will speak a bit about Phase Matching and how we try to optimize it in the lab. I will also
speak a bit about how we plan on Distributing Entanglement, so Quantum Teleportation and Entanglement Swapping.
Then I will speak about the present state that we are dealing with in the lab and how far we have gone so far.

\section{Motivation}
My project is in the scope of the SiQUID project. The goal of it is to demonstrate that a slovenian quantum network is possible to make,
train young researchers in the field of Quantum Technologies, as well as creating the sources of entanglement needed for entanglement based Quantum Key Distribution.
The main goal for me in this regard would be to create a bright source of polarization entanglemed photon pairs using a Sagnac interferometer.
The reason we chose a Sagnac in this case is due to its vibrational stability, and it's a convenient way to do so as it allows for a compact design.
A possible downside to this is that the alignment of this type of interferometer is a bit tedious, as you have to perfectly overlap two beams within the interferometer,
and in the crystal, and there is coupling between the various degrees of freedom that we have available to us.

\section{Theory}
\subsection{Spontaneous Parametric Downconversion}
Before talking about entanglement, we need to know how to generate photon pairs. A good way to do this is SPDC.
SPDC is a process in which a pump photon of frequency $\omega_p$ gets "downconverted" to two photons of lower frequency,
usually denoted as $\omega_s$ and $\omega_i$. These output photons can have the same frequency,
and thus this can be a degenerate process, or they can have different frequencies, being non-degenerate.
When we started this entire design process, we wanted to have degenerate frequencies, as we would be able to separate
the two photons easily based on their polarization. This design utilited a type-II process, meaning that we would be able 
to separate the two photons on the output of the PBS. Now we have decided to go for a type-0 process, thus altering the design a bit.

\subsection{Phase Matching}
What is Phase Matching? Phase matching refferrs to fixing the relative phase between two or more frequencies 
of light as they propagate through the crystal. Phase matching are the conditions at which some process will happen.
In the case of SPDC the Phase Matching conditions are energy conservation ($\hbar \omega_p = \hbar \omega_s + \hbar \omega_i$),
and momentum conservation ($k_p = k_s + k_i$). The efficiency of this process is also dictated by how well we satisfy these 
two conditions - being slightly off from $\Delta k = 0$ could result in large losses of photon generation efficiency.
% DONE NOTE: Add a slide after Phase Matching slide regarding type-0,-I,-II. DONE
% DONE NOTE: Mention the crystal properties - anisotropic stuff DONE
% NOTE: Add for quasi phase matching

\subsection{Crystals}
An important step is to choose the correct material for the job. In this case, no ordinary material can be used, 
as the phase matching condition $k_1 + k_2 = k_3 = \frac{n \omega_3}{c} = \frac{n \omega_1}{c} + \frac{n \omega_2}{c}$ is
impossible to satisfy in most cases as
$n_1 \left(	\omega_1 \right) < n_2 \left( \omega_2 \right) < n_3 \left(	\omega_3 \right)$ for $\omega_1 < \omega_2 < \omega_3$.
For this, we choose a birefringent material, which has a fast and a slow axis.

\subsubsection{Crystal Size}
Before periodic poling you could only have a limited size of the nonlinear material along the propagation direction.
This means that you would only be able to generate broad signals out of the crystals and at lower intensities, as the 
intensity and bandwidth of the process goes proportionally to $I \propto \text{sinc}^2\left(\frac{L \Delta k}{2}\right)$.
Today we have periodic poling available to us which allows us to create oppositely oriented fields inside the
crystal. This allows for higher photon generation rates, longer crystals and more narrow bandwidths.

\subsection{Types of processes}
There are three different types of phase matching.
\begin{itemize}
	\item Type-0 is one in which a photon of some linear polarization, say $\ket{H}$ gets downconverted
into two photons of lower energies of the same polarization - o $\rightarrow$ o + o. This process used to be physically impossible to produce 
as the phase matching condition $k_1 + k_2 = k_3 = \frac{n \omega_3}{c} = \frac{n \omega_1}{c} + \frac{n \omega_2}{c}$ is impossible to achieve
in most cases.
	\item Type-I is similar to type-0, except that the produced two photons are orthogonally polarized to the input photon - o $\rightarrow$ e + e
	\item In type-II instead you get two orthogonal polarizations out from the crystal. e $\rightarrow$ e + o
\end{itemize}

\subsection{Phase Matching Temperature}
The first important step is to determine the correct Phase Matching Temperature for our crystal. For different types of poling
periods these temperatures will vary, and in general, the lower the poling period the higher the Phase Matching Temperature.
The opposite seems to be true for the temperature bandwidth, which would be the acceptable temperature range for the material.
Similarly, one should not heat up certain crystals too much as this could destroy the Periodic Poling properties of the material.
For Periodically Poled Lithium Niobate (PPLN), the safe operating regime up to around 200 °C.

\subsection{Bandwidths}
Next would be the bandwidth of the process. In general, you would like it to be as narrow as possible - especially for certain
applications as entanglement swapping, and there are other methods to achieve this which I won't go into now. In general,
type-II processes are much more narrow than type-0 ones. In the plot I have on the slide, there's about a factor of 120 in 
bandwidth difference. This is why Type-II was the primary choice for our group. Unfortunately, this also comes at the price 
of reduced intensity. The difference in intensity was around 25 times lower compared to the type-0 crystal which led to us
using it in the end design.

\subsection{Different Designs}
First figure
Schematic of one method to produce and select the polarization-entangled state from the down-conversion crystal.
The extra birefringent crystals Cl and C2, along with the half wave plate HWPO, are used to compensate the birefringent
walk-off effects from the production crystal. By appropriately setting half wave plate HWP1 and quarter wave plate QWP1,
one can produce all four of the orthogonal EPR-Bell states. Each polarizer P1 and P2 consisted of two stacked polarizing
beam splitters preceded by a rotatable half wave plate.

Interesting thing is that here with the use of various compensating crystals and wave plates they can create any
Bell state.

2nd figure
Similar to 1, but doesn't require realignment, also much brighter.

3rd figure
Shows a use of noncollinear phase matching at room temperature (not really,
it's 32 °C), robust source of high brightness

4th figure
They present a source of polarization-entangled photon pairs based on time-reversed
Hong-Ou-Mandel interference. By superimposing four pair-creation possibilities on a polarization beam
splitter, pairs of identical photons are separated into two spatial modes without the usual requirement for
wavelength distinguishability or noncollinear emission angles. Our source yields high-fidelity polarization entanglement and high pair-generation rates without any requirement for active interferometric
stabilization, which makes it an ideal candidate for a variety of applications, in particular those requiring
indistinguishable photons.

\section{Entanglement}
If we have two systems, $\ket{\psi_i}$ and $\ket{\phi_i}$ described by two separate Hilbert spaces, the state of those two systems can be described as 
a tensor product $\ket{\psi_i} \otimes \ket{\phi_i}$ of their state spaces. You can write this as a Schmidt decomposition: $\ket{\xi} = \sum_{i,j} 
a_{ij} \ket{\psi_i}\ket{phi_i} \rightarrow \sum_i b_i \ket{\psi_i'} \ket{\phi_i'}$. If $b_i \ne 0$ and $b_{i \ne j} = 0$ the state is said to be separable.
If more than one $b_i \ne 0$ then $\ket{\xi}$ is said to be entangled, and the states can no longer be described without their comprising states.
An example of entangled states are Bell states, they are also maximally entangled:

\begin{center}
	\begin{aligned}
		\begin{equation}
			$\ket{\Psi^-} &= \frac{1}{\sqrt{2}} \left( \ket{H}\ket{V} - \ket{V}\ket{H} \right)$\\
			$\ket{\Psi^+} &= \frac{1}{\sqrt{2}} \left( \ket{H}\ket{V} + \ket{V}\ket{H} \right)$\\
			$\ket{\Phi^-} &= \frac{1}{\sqrt{2}} \left( \ket{H}\ket{H} - \ket{V}\ket{V} \right)$\\
			$\ket{\Phi^+} &= \frac{1}{\sqrt{2}} \left( \ket{H}\ket{H} + \ket{V}\ket{V} \right)$
		\end{equation}
	\label{eq:bsm}
	\end{aligned}
\end{center}

\subsection{Why do we care about entanglement?}
Entanglement sources have many applications. Some notable ones are Quantum Computation, Quantum Imaging,
and Quantum Sensing (to be explained). % TODO: 
As we wish to distribute these entangled states over long distances, due to fiber losses it isn't viable to do so 
over distances larger than a couple hundred kilometers.
\begin{center}
	\begin{table}[h]
		\caption{Relevant fiber loss. \textit{Source: Thorlabs}}
		\label{tab:fiberloss}
		\begin{tabular}{|c|c|c|c|c|c|c|}
			\hline
			$\lambda$ [nm] & 430 & 532 & 780 & 1310 & 1550 & 1900\\
			\hline
			Loss [dB/km] & 50 & 30 & 12 & 0.32 &  0.18 & 5\\
			\hline
		\end{tabular}
	\end{table}
\end{center}
\begin{exampleblock}{Example: Loss in fiber for 1550/1560 nm}
	200 km of fiber $\rightarrow$ -36 dB $\rightarrow$ $10^4$ loss.
	Start with 1 W, end up with 0.0001 W.
\end{exampleblock}
It would be desirable to have a more robust way to transport photons from A to B.
In our case, we generate a pair of $\ket{V}$ polarized photons in each of the branches of the Sagnac interferometer,
or in the case of type-II we get $\ket{H} + \ket{V}$, totalling to 4 photons being created "at the same time".
This leads us to the entangled state for type-II/0 SPDC:
\begin{equation*}
	\ket{\Psi_{p}} = \frac{1}{\sqrt{2}} ( a_{H}^{\dagger} ( \omega_p ) + a_{V}^{\dagger} ( \omega_p ) )\ket{0}\\
\end{equation*}
\begin{minipage}[l]{0.48\textwidth}
	\begin{equation*}
		\begin{aligned}
			\ket{\Psi_{\text{Type-2}}} &= \frac{1}{\sqrt{2}}(a_{H}^{\dagger}(\omega_s)a_{V}^{\dagger}(\omega_i)+\\
								&a_{V}^{\dagger}(\omega_i)a_{H}^{\dagger}(\omega_s))\ket{0}\\
		\end{aligned}
	\end{equation*}
\end{minipage}
\begin{minipage}[r]{0.48\textwidth}
\begin{equation*}
	\begin{aligned}
		\ket{\Psi_{\text{Type-0}}} &= \frac{1}{\sqrt{2}}(a_{H}^{\dagger}(\omega_s)a_{H}^{\dagger}(\omega_i)+\\
								   &a_{V}^{\dagger}(\omega_i)a_{V}^{\dagger}(\omega_s))\ket{0}
	\end{aligned}
\end{equation*}
\end{minipage}

\par as one branch of the phons will be rotated in polarization due to a wave plate being in one of the branches of the Sagnac interferometer.
This leads us to one of the Bell States mentioned above.
The reason why this is important is that if the conditions for this to happen are satisfied, you can no longer 
predict the result of the polarization measurement. You can no longer tell which photon is which and from where it came from.
But the measurements will be perfectly anti-correlated in polarization. 

\subsection{Distributing Entanglement}
Due to condisderable losses in fibers it is not feasible to transport entangled photons over distances greater than a few hundred kilometers. Thus 
we must find a different solution to long distance Entanglement Dirstribution. This could be solved with quantum repeater, but currently one does not exist, 
or is difficult to create. That being said, it is reasonable to establish an entanglement swapping network beforehand.

\subsection{Quantum Teleportation}
The basis of Quantum Teleportation is that the sender and receiver share an initial entangled pair -
in our case polarization entangled photons of the sort described above, in one of the 4 Bell States
so that the photons are just as likely to be horizontally or vertically polarized relative to the pump,
then perform a Bell State Measurement (BSM) on the receiving entangled 
photon and an initial state which we would like to teleport to the receiver. 
By entangling the initial state photon with one of the other entangled photons, and measuring the type of entanglement that they
share, we can then send that information to B through a classical signal change B so that it has the same polarization as X,
all without ever knowing their polarizations.

\begin{description}
	\item[BSM] 
		A BSM is a coincidence measurement between different detectors. The simplest one would
		be the $\ket{\Psi^-}$ of \ref{eq:bsm} measurement. In it you only have a 50:50 beam splitter and two detectors. The only
		time both photons will fall on their own separate detector is when either both of them get transmitted,
		or both get reflected. This can be improved and the 2nd state can be measured if a PBS is added one 
		of the branches after the BS. Thus we'd have a 50\% complete BSM. By adding another PBS in the other 
		branch we can increase the completeness to 75\%. In order to reach a complete BSM, one must use nonlinear
		elements\footnote{PhysRevLett.86.1370}.
\end{description}

Then we report the result of the BSM via a classical communications channel to the receiver.
This same procedure does not work if the entangled pair does not exist.
The amazing thing is that this should work regardless of the distance between the entangled states,
and doesn't vialate causality, as the cassical message is being sent at most at the speed of causality to the receiver.

It is also important to notice that the Bell-state measurement does not reveal any information
on the properties of any of the particles. This is the very reason why
quantum teleportation using coherent two-particle superpositions works,
while any measurement on one-particle superpositions would fail.
The fact that no information whatsoever is gained on either particle is also the reason
why quantum teleportation escapes the verdict of the no-cloning theorem. After successful teleportation
particle 1 is not available in its original state any more,
and therefore particle 3 is not a clone but is really the result of teleportation.

\subsection{Entanglement Swapping}
The principle behind Entanglement swapping can be described in the exact same way as Quantum Teleportation. The only difference is now that the arbitrary
state which we're trying to teleport is part of an entangled pair. This means that if the same process of Quantum Teleportation is performed by Alice, Bob,
Charlie and Dora, where Alice and Bob share one entangled pair, and Charlie and Dora share another. Now the teleported state will be the state of 
one of the entangled photons, thus swapping the entanglement between Charlie and Bob.

\section{Present State}
\subsection{Parameters}
\subsection{Building a Sagnac interferometer}

\end{document}
