\documentclass{article}
%%%%%%%%%%%%%%%%%%%%%%%%%%%%%%%%%
% PACKAGE IMPORTS
%%%%%%%%%%%%%%%%%%%%%%%%%%%%%%%%%
\usepackage[framemethod=TikZ]{mdframed}
\usepackage{amsthm}
\usepackage{tikzsymbols}
\usepackage[tmargin=2.5cm,rmargin=2cm,lmargin=2.5cm,margin=0.85in,bmargin=2.5cm,footskip=.2in]{geometry}
\linespread{1.3}
\usepackage{amsmath,amsfonts,amsthm,amssymb,mathtools}
\usepackage[varbb]{newpxmath}
\usepackage{xfrac}
\usepackage[makeroom]{cancel}
\usepackage{bookmark}
\usepackage{enumitem}
\usepackage{hyperref,theoremref}
\hypersetup{
	pdftitle={Assignment},
	colorlinks=true, linkcolor=doc!90,
	bookmarksnumbered=true,
	bookmarksopen=true
}
\usepackage[most,many,breakable]{tcolorbox}
\usepackage{xcolor}
\usepackage{varwidth}
\usepackage{etoolbox}
\usepackage{nameref}
\usepackage{multicol,array}
\usepackage{tikz-cd}
\usepackage[ruled,vlined,linesnumbered]{algorithm2e}
\usepackage{comment} % enables the use of multi-line comments (\ifx \fi) 
\usepackage{xifthen}
\usepackage{pdfpages}
\usepackage{transparent}
\usepackage[bottom]{footmisc}
\usepackage[utf8]{inputenc} % allow utf-8 input
\usepackage{braket}
\usepackage{titling} % For customizing title
\usepackage{abstract} % For abstract formatting
\usepackage{wrapfig}			  %Kontekstom osjetljivo navođenje
\usepackage{csquotes}			  %Kontekstom osjetljivo navođenje
\usepackage{graphicx}       %Slike i slično
\usepackage{url}            % simple URL typesetting
\usepackage{booktabs}       % professional-quality tables
\usepackage{nicefrac}       % compact symbols for 1/2, etc.
\usepackage{float}
\usepackage{caption}
\usepackage{subcaption}
\usepackage{listings}
\usepackage[english]{babel}
\usepackage{titlesec}				%Za naslovnu stranicu
\usepackage[T1]{fontenc}
\usepackage{import}
\usepackage{svg}



\counterwithin{figure}{section}
\urlstyle{same}
\numberwithin{equation}{section}
\hyphenation{pergamon} % riječ u argumentu (pergamon) se ne rastavlja s crticom; ne smije imat specijalna slova. č. ž... - rastavljam ju naredbom \- (npr. išče\-zava)
\setlength\parindent{0pt} % u novom paragrafu: indent=0
\setlength{\parskip}{10pt} % postavlja željeni vertikalni razmak između paragrafa
\setlength{\skip\footins}{2cm} % razmak između glavnog teksta i fusnota
\renewcommand{\thefootnote}{$\ddagger$} % želim dagger za oznaku fusnota
\newcommand{\HRule}{\rule{\linewidth}{0.4mm}} % nova naredba za horizontalne linije na naslovnoj stranici
\newlength{\mylen}
\setcounter{secnumdepth}{4}
\titleformat{\paragraph}
{\normalfont\normalsize\bfseries}{\theparagraph}{1em}{}


%%%%%%%%%%%%%%%%%%%%%%%%%%%%%%%%%%%%%%%%%%%
% TABLE OF CONTENTS
%%%%%%%%%%%%%%%%%%%%%%%%%%%%%%%%%%%%%%%%%%%

\usepackage{tikz}
\definecolor{doc}{RGB}{0,60,110}
\usepackage{titletoc}
\contentsmargin{0cm}
\titlecontents{chapter}[3.7pc]
{\addvspace{30pt}
	\begin{tikzpicture}[remember picture, overlay]
		\draw[fill=doc!60,draw=doc!60] (-7,-.1) rectangle (-0.9,.5);
		\pgftext[left,x=-3.5cm,y=0.2cm]{\color{white}\Large\sc\bfseries Chapter\ \thecontentslabel};
	\end{tikzpicture}\color{doc!60}\large\sc\bfseries}
{}
{}
{\;\titlerule\;\large\sc\bfseries Page \thecontentspage
	\begin{tikzpicture}[remember picture, overlay]
		\draw[fill=doc!60,draw=doc!60] (2pt,0) rectangle (4,0.1pt);
	\end{tikzpicture}}
\titlecontents{section}[1.7pc]
{\addvspace{8pt}}
{\contentslabel[\thecontentslabel]{1pc}}
{}
{\ \hrulefill\ \small\thecontentspage}
[\vspace{-0.4cm}]
\titlecontents*{subsection}[3.7pc]
{\addvspace{-0pt}\small}
{}
{}
{\dotfill \small\thecontentspage\\}
[][\vspace{-1cm}]

\makeatletter
\renewcommand{\tableofcontents}{
	\chapter{
	  \vspace{-20\p@}
	  \begin{tikzpicture}[remember picture, overlay]
		  \pgftext[right,x=15cm,y=0.2cm]{\color{doc!60}\Huge\sc\bfseries \contentsname};
		  \draw[fill=doc!60,draw=doc!60] (13,-.75) rectangle (20,1);
		  \clip (13,-.75) rectangle (20,1);
		  \pgftext[right,x=15cm,y=0.2cm]{\color{white}\Huge\sc\bfseries \contentsname};
	  \end{tikzpicture}}
	\@starttoc{toc}}
\makeatother


%%%%%%%%%%%%
%% KUTIJE %%
%%%%%%%%%%%%
%%%%%%%%%%%%%%%%%%%%%%%%%%%%%%
%Theorem
\newcounter{teorem}[section] \setcounter{teorem}{0}
\renewcommand{\theteorem}{\arabic{section}.\arabic{teorem}}
\newenvironment{teorem}[2][]{%
	\refstepcounter{teorem}%
	\ifstrempty{#1}%
	{\mdfsetup{%
			frametitle={%
					\tikz[baseline=(current bounding box.east),outer sep=0pt]
					\node[anchor=east,rectangle,fill=blue!20]
					{\strut Teorem~\thetheo};}}
	}%
	{\mdfsetup{%
			frametitle={%
					\tikz[baseline=(current bounding box.east),outer sep=0pt]
					\node[anchor=east,rectangle,fill=blue!20]
					{\strut Teorem~\thetheo:~#1};}}%
	}%
	\mdfsetup{innertopmargin=10pt,linecolor=blue!20,%
		linewidth=2pt,topline=true,%
		frametitleaboveskip=\dimexpr-\ht\strutbox\relax
	}
	\begin{mdframed}[]\relax%
		\label{#2}}{\end{mdframed}}
%%%%%%%%%%%%%%%%%%%%%%%%%%%%%%
%Proof
\newcounter{dokaz}[section]\setcounter{dokaz}{0}
\renewcommand{\thedokaz}{\arabic{section}.\arabic{dokaz}}
\newenvironment{dokaz}[2][]{%
	\refstepcounter{dokaz}%
	\ifstrempty{#1}%
	{\mdfsetup{%
			frametitle={%
					\tikz[baseline=(current bounding box.east),outer sep=0pt]
					\node[anchor=east,rectangle,fill=red!20]
					{\strut Dokaz~\theprf};}}
	}%
	{\mdfsetup{%
			frametitle={%
					\tikz[baseline=(current bounding box.east),outer sep=0pt]
					\node[anchor=east,rectangle,fill=red!20]
					{\strut Dokaz~\theprf:~#1};}}%
	}%
	\mdfsetup{innertopmargin=10pt,linecolor=red!20,%
		linewidth=2pt,topline=true,%
		frametitleaboveskip=\dimexpr-\ht\strutbox\relax
	}
	\begin{mdframed}[]\relax%
		\label{#2}}{\qed\end{mdframed}}

%%%%%%%%%%%%%%%%%%%%%%%%%%%%%%
%Primjer
\newcounter{primjer}[section]\setcounter{primjer}{0}
\renewcommand{\theprimjer}{\arabic{section}.\arabic{primjer}}
\newenvironment{primjer}[2][]{%
	\refstepcounter{primjer}%
	\ifstrempty{#1}%
	{\mdfsetup{%
			frametitle={%
					\tikz[baseline=(current bounding box.east),outer sep=0pt]
					\node[anchor=east,rectangle,fill=red!20]
					{\strut Primjer~\theprf};}}
	}%
	{\mdfsetup{%
			frametitle={%
					\tikz[baseline=(current bounding box.east),outer sep=0pt]
					\node[anchor=east,rectangle,fill=red!20]
					{\strut Primjer~\theprf:~#1};}}%
	}%
	\mdfsetup{innertopmargin=10pt,linecolor=red!20,%
		linewidth=2pt,topline=true,%
		frametitleaboveskip=\dimexpr-\ht\strutbox\relax
	}
	\begin{mdframed}[]\relax%
		\label{#2}}{\qed\end{mdframed}}
%%%%%%%%%%%%%%%%%%%%%%%%%%%%%%
%================================
% NOTE BOX
%================================

\usetikzlibrary{arrows,calc,shadows.blur}
\tcbuselibrary{skins}
\newtcolorbox{Bilješka}[1][]{%
	enhanced jigsaw,
	colback=gray!20!white,%
	colframe=gray!80!black,
	size=small,
	boxrule=1pt,
	title=Bilješka,
	halign title=flush center,
	coltitle=black,
	breakable,
	drop shadow=black!50!white,
	attach boxed title to top left={xshift=1cm,yshift=-\tcboxedtitleheight/2,yshifttext=-\tcboxedtitleheight/2},
	minipage boxed title=1.5cm,
	boxed title style={%
			colback=white,
			size=fbox,
			boxrule=1pt,
			boxsep=2pt,
			underlay={%
					\coordinate (dotA) at ($(interior.west) + (-0.5pt,0)$);
					\coordinate (dotB) at ($(interior.east) + (0.5pt,0)$);
					\begin{scope}
						\clip (interior.north west) rectangle ([xshift=3ex]interior.east);
						\filldraw [white, blur shadow={shadow opacity=60, shadow yshift=-.75ex}, rounded corners=2pt] (interior.north west) rectangle (interior.south east);
					\end{scope}
					\begin{scope}[gray!80!black]
						\fill (dotA) circle (2pt);
						\fill (dotB) circle (2pt);
					\end{scope}
				},
		},
	#1,
}

\input{macros}
\input{letterfonts}

% Customizing title
\renewcommand{\maketitlehooka}{\centering}
\renewcommand{\maketitlehookb}{\vspace{-1.5em}}

% Customizing abstract
\renewcommand{\abstractnamefont}{\normalfont\large\bfseries}
\renewcommand{\abstracttextfont}{\normalfont\normalsize}
\graphicspath{{./Images/}} % Where to take images from
%%%%%%%%%%%%%%%%%%%%%%%%%% Define some useful colors %%%%%%%%%%%%%%%%%%%%%%%%%%
\definecolor{ocre}{RGB}{243,102,25}
\definecolor{mygray}{RGB}{243,243,244}
\definecolor{deepGreen}{RGB}{26,111,0}
\definecolor{shallowGreen}{RGB}{235,255,255}
\definecolor{deepBlue}{RGB}{61,124,222}
\definecolor{shallowBlue}{RGB}{235,249,255}
%%%%%%%%%%%%%%%%%%%%%%%%%%%%%%%%%%%%%%%%%%%%%%%%%%%%%%%%%%%%%%%%%%%%%%%%%%%%%%%

%%%%%%%%%%%%%%%%%%%%%%%%%% Define an orangebox command %%%%%%%%%%%%%%%%%%%%%%%%
\newcommand\orangebox[1]{\fcolorbox{ocre}{mygray}{\hspace{1em}#1\hspace{1em}}}
%%%%%%%%%%%%%%%%%%%%%%%%%%%%%%%%%%%%%%%%%%%%%%%%%%%%%%%%%%%%%%%%%%%%%%%%%%%%%%%

%%%%%%%%%%%%%%%%%%%%%%%%%%%% English Environments %%%%%%%%%%%%%%%%%%%%%%%%%%%%%
\newtheoremstyle{mytheoremstyle}{3pt}{3pt}{\normalfont}{0cm}{\rmfamily\bfseries}{}{1em}{{\color{black}\thmname{#1}~\thmnumber{#2}}\thmnote{\,--\,#3}}
\newtheoremstyle{myproblemstyle}{3pt}{3pt}{\normalfont}{0cm}{\rmfamily\bfseries}{}{1em}{{\color{black}\thmname{#1}~\thmnumber{#2}}\thmnote{\,--\,#3}}
\theoremstyle{mytheoremstyle}
\newmdtheoremenv[linewidth=1pt,backgroundcolor=shallowGreen,linecolor=deepGreen,leftmargin=0pt,innerleftmargin=20pt,innerrightmargin=20pt,]{theorem}{Theorem}[section]
\theoremstyle{mytheoremstyle}
\newmdtheoremenv[linewidth=1pt,backgroundcolor=shallowBlue,linecolor=deepBlue,leftmargin=0pt,innerleftmargin=20pt,innerrightmargin=20pt,]{definition}{Definition}[section]
\theoremstyle{myproblemstyle}
\newmdtheoremenv[linecolor=black,leftmargin=0pt,innerleftmargin=10pt,innerrightmargin=10pt,]{problem}{Problem}[section]
%%%%%%%%%%%%%%%%%%%%%%%%%%%%%%%%%%%%%%%%%%%%%%%%%%%%%%%%%%%%%%%%%%%%%%%%%%%%%%%
\begin{document}
%%%%%%%%%%%%%%%%%%%%%%%%%%%%%%% Title & Author %%%%%%%%%%%%%%%%%%%%%%%%%%%%%%%%
\begin{titlepage}
	\begin{center}
		{\LARGE Univerza \textit{v Ljubljani}} \\[0.1cm]
		{\LARGE Fakulteta za \textit{matematiko in fiziko}} \\[1cm]
		\begin{figure}[h]
			\centering
			\includegraphics[width=0.25\textwidth]{UL logo.png}
			\label{fig:UL logo}
		\end{figure}
		{\large \textbf{Adrian Udovičić, mag. phys.}}\\[0.1cm]
		{\sc DISPOZICIJA DOKTORSKE DISERTACIJE}\\
		\vspace{1cm}

		{\bf \Large Ustvarjanje in teleportiranje prepletenosti za kvantna omrežja}\\
		\vspace{1cm}

		{\bf \Large Generating and teleporting entanglement for quantum networks}\\


		\vspace{3cm}

		{\large ADVISER: Rainer O. Kaltenbaek, assoc. prof. dr.\\

			\vspace{2cm}
			Znanstveno področje: Fizika\\
			\vspace{1cm}
			Ljubljana, 2025\date{}}
	\end{center}
\end{titlepage}
%%%%%%%%%%%%%%%%%%%%%%%%%%%%%%%%%%%%%%%%%%%%%%%%%%%%%%%%%%%%%%%%%%%%%%%%%%%%%%%
%-----------------------------------------------------------------------------------------------
%         APPLICATION
%----------------------------------------------------------------------------------------------
\clearpage
\pagestyle{plain}
\pagenumbering{roman}

\vspace{1cm}

\noindent Senat UL FMF\\
Fakulteta za matematiko in fiziko\\
Jadranska ulica 19\\
1000 Ljubljana\\

\vspace{.5cm}

\begin{center}
	\textbf{Zadeva: Prošnja za odobritev teme doktorske disertacije}
\end{center}
\vspace{.5cm}

Spoštovani člani odbora,
\vspace{1cm}

Pišem vam, da bi se uradno prijavil za temo svoje doktorske disertacije. Sem na doktorskem študijskem programu Fizika.
Moje raziskave na področju kvantne komunikacije potekajo pod mentorstvom izrednega profesorja Dr. Rainerja Oliverja Kaltenbaeka,
Osredotočam se na razvoj visoko zmogljivega vira prepletenih fotonov. Naslov disertacije je „Generiranje in teleportiranje prepletenosti za kvantna omrežja“.
Cilj moje študije je zasnovati vir, ki je dovolj širokopasovni, da lahko hkrati služi več odjemalcem,
s čimer bi izboljšali praktičnost in razširljivost kvantnih omrežij.

Svoje raziskave želim nadaljevati na Univerzi v Ljubljani,
Fakulteti za matematiko in fiziko, in se veselim te priložnosti,
da bom lahko prispeval k temu vznemirljivemu področju.

Zahvaljujem se vam za vaš čas in razmislek. Veselim se vašega odgovora.



\vspace{1cm}
S spoštovanjem,\\
Adrian Udovičić\\
adrian.udovicic@fmf.uni-lj.si\\
Ulica Ante Kovačića 12A, 23000 Zadar, Hrvaška\\
Fakulteta za matematiko in fiziko, Oddelek za fiziko

%-----------------------------------------------------------------------------------------------
%         Življenjepis or CV or Biography
%----------------------------------------------------------------------------------------------

\clearpage
\pagestyle{plain}

\vspace{1cm}

\begin{center}
	\textbf{Short CV}
\end{center}
I am a PhD candidate in physics at the University of Ljubljana, Faculty of Mathematics and Physics (FMF),
working in the Laboratory for Quantum Optics under the supervision of Assoc. Prof. Dr. Rainer O. Kaltenbaek.
My research focuses on developing a high-yield, broadband source of entangled photons for quantum communication.
I received a Master’s degree in physics from the University of Rijeka,
where I conducted research on transient signals in dark matter detection,
and a Bachelor’s degree in physics, with a thesis on spectral analysis of AGN Markarian 421 in the very high-energy gamma region.
I have experience in quantum and nonlinear optics from my work at my supervisors laboratory at FMF.

\vspace{1cm}

\begin{center}
	\textbf{Kratki življenjepis}
\end{center}
Sem doktorski kandidat fizike na Fakulteti za matematiko in fiziko Univerze v Ljubljani (FMF),
kjer delam v Laboratoriju za kvantno optiko pod mentorstvom doc. dr. Rainerja O. Kaltenbaeka.
Moje raziskave se osredotočajo na razvoj visokozmogljivega, širokopasovnega vira prepletenih fotonov za kvantno komunikacijo.
Na Univerzi na Reki sem magistriral iz fizike, kjer sem raziskoval prehodne signale pri odkrivanju temne snovi,
in diplomiral iz fizike z nalogo o spektralni analizi AGN Markarian 421 v območju zelo visokih energij gama.
Izkušnje na področju kvantne in nelinearne optike imam iz dela v laboratoriju svojega mentorja na FMF.


%-----------------------------------------------------------------------------------------------
%         Soglasje mentorja or Mentor's consent
%----------------------------------------------------------------------------------------------

\clearpage
\pagestyle{plain}

\vspace{1cm}

\begin{center}
	\textbf{Mentor's consent}
\end{center}

\vspace{1cm}

\noindent (Mentor's consent addressing Senate UL FMF)

%-----------------------------------------------------------------------------------------------
%         Predlog za odobritev pisanja doktorske disertacije v angleščini
%----------------------------------------------------------------------------------------------

\clearpage
\pagestyle{plain}

\begin{center}
	\textbf{Application for writing a doctoral dissertation in English}\\
\end{center}

\noindent Senat UL FMF\\
Faculty of mathematics and physics\\
Jadranska ulica 19\\
1000 Ljubljana\\

\vspace{1cm} % Adjust spacing below the title if needed
Dear Committee Members,
\vspace{1cm}

I hope this message finds you well. I am writing to formally request permission to write my PhD thesis in English.
As an international student and non-native speaker of Slovenian, I believe that completing my thesis in English would be beneficial for both academic and practical reasons.
Firstly, English is the primary language in my field of study, and the majority of relevant literature, research articles, and publications are available in English.
Writing my thesis in English would enable me to engage more directly with this body of work and ensure that my research is positioned within the global academic discourse.
Secondly, my supervisor, Assoc. Prof. Dr. Rainer Oliver Kaltenbaek, who is also a non-native speaker of Slovenian, has advised that conducting and evaluating the research in
English would facilitate clearer communication and collaboration throughout the thesis process. Furthermore, writing in English would allow for smoother peer review
and potential publication in international journals.
Lastly, most, if not all of the literature that I am using in my doctoral studies are in English and I believe it would slow down my progress to translate
all of the terminoligy and nomenclature to Slovenian.
I greatly appreciate your understanding and consideration of this request. I am confident that writing my thesis in English will enhance its academic
impact and contribute positively to my development as a researcher. Please let me know if further clarification or documentation is required to support this appeal.
Thank you for your time and attention. I look forward to your response.

\vspace{1cm}
Yours sincerely,\\
Adrian Udovičić\\
Faculty of Mathematics and Physics, Department of Physics\\
adrian.udovicic@fmf.uni-lj.si



%-----------------------------------------------------------------------------------------------
%         Ph.D. thesis disposition (in English)
%----------------------------------------------------------------------------------------------

\clearpage
\pagestyle{plain}

\begin{center}
	\textbf{\Large Disposition of doctoral dissertation (in English)}
\end{center}

%-----------------------------------------------------------------------------------------------
%         Ph.D. thesis disposition (in Slovene)
%----------------------------------------------------------------------------------------------

% \clearpage
% \pagestyle{plain}
% \pagenumbering{roman}
% \newpage
% \tableofcontents
% \newpage
\pagenumbering{arabic}

\section{Description of the immediate research area and its problems}
% (approximately half a page)
%• State and briefly describe the immediate research area in which research will be conducted, and to which an original contribution is expected.
% Emphasise the importance and timeliness of the narrower field.
%• In a few sentences, describe the essence of the problem that will be tackled in the doctoral research and motivate it.
% (a more detailed description is required later in the "dispozicija").

In the rapidly advancing fields of quantum communication and quantum computing, sensing, and simulators
the efficient transfer of secure quantum information is of great importance.
A key quantum resource is entanglement, which facilitates experiments such as quantum teleportation and entanglement swapping.
Conducting these experiments over long distances through optical fibers presents significant challenges due to transmission losses.
To mitigate this, photons must be generated at wavelengths compatible with existing fiber-optic networks,
particularly in the C near-infrared band where transmission losses are minimal.
These advances contribute to the broader goal of realizing quantum networks,
which require robust capabilities for generating and characterizing aforementioned quantum processes. %) quantum coherence and entanglement.
Such networks rely on quantum interconnects, which convert quantum states between physical systems in a reversible manner,
enabling the distribution of entanglement and the teleportation of quantum states across network nodes \cite{Kimble_2008}.
Additionally, free-space communication methods \cite{Kržić_et_al_2023} are being explored for applications in metropolitan areas.
\par While these technologies are essential for local and metropolitan quantum networks, scaling to a global level requires overcoming
the inherent limitations of photon loss over long distances. This is where quantum repeaters and high-yield entanglement sources
play a critical role for the future global quantum internet. High-yield entanglement sources in in-between nodes, coupled to
quantum repeaters may enable the distribution of entanglement over arbitrarily long distances by overcoming exponential loss scaling,
even in fiber networks where attenuation is low. Without such sources and repeaters, entanglement distribution is limited to distances
of only a few hundred kilometers.
\par This work seeks to establish the technical bedrock for future scalable quantum networks. These efforts do not exist in isolation;
they directly feed into the broader mission of transforming theoretical quantum advantages into real-world systems.
I aim to achieve not only the first realization of a high-yield polarization entanglement source at non-degenerate
frequencies in Slovenia but also to demonstrate quantum teleportation and entanglement swapping using continuous wave lasers.
Ongoing research efforts aim to bridge the gap between theoretical advancements and practical applications,
driving the quest for more efficient and accessible quantum systems that could transform various sectors,
including telecommunications and healthcare. As these technologies mature, they are poised to redefine our
understanding of information processing and secure communications in the quantum era.
\par\textbf{Key words: Quantum Entanglement, Quantum Communication, Entanglement Swapping}

\section{Overview of related research and relevant literature}
% (approximately one page)
% • Briefly summarise the state-of-the-art. Review and briefly analyse the relevant literature including the most recent research in the area.
% • Explain how the proposed topic leads on from existing research, and outline the importance of the proposed line of research and the challenges it poses.
% • Explain the relevance and timeliness of the proposed research in the context of the literature reviewed above.
Quantum entanglement sources are pivotal components in the field of quantum mechanics, enabling the generation of entangled states
that are essential for a range of applications, including quantum computing, cryptography, simulations, and communication.
These sources can produce pairs of entangled photons through various techniques such as cavity-enhanced configurations,
quantum dot mechanisms, and by far the most widely used method being Spontaneous Parametric Down-Conversion \cite{jesseSPDC} (SPDC).
The ability to create reliable and efficient entangled states has garnered significant interest due to their
implications for advancing quantum technologies and facilitating secure information transfer across long distances.
\par Entanglement sources are crucial for quantum technologies, but scalability, efficiency,
and reliability remain challenges. Advances like quantum repeaters and high-yield distribution have improved fidelity and signal loss,
yet practical implementation faces hurdles such as room-temperature operation and protocol complexity.
\par After one of the first \cite{Kwiat_1995} demonstrations of a high-intensity polarization entangled source was realized it also
became apparent that they can be fully done on chip \cite{S_G_S_C_F_B_L_G_B_2022} for frequency-bin entanglement,
polarization entanglement \cite{L_Z_F_F_L_L_W_R_D_X_etal._2017}, and also for hybrid frequency-polarization
entangled states \cite{F_R_D_F_L_M_A_B_D_2023}. Latest research in this field is advancing quickly, specifically for
Pulsed Laser (PL) sources which offer higher peak power and fewer synchronization constraints.
A benefit to using Continuous Wave lasers (CW) as compared to PL is less maintenance and lower cost, for instance in an industrial or government
setting where access may be limited. Some notable mentions using a similar design with a Sagnac loop
\cite{Neumann_Buchner_Bulla_Bohmann_Ursin_2022_CW,Chen_Ecker_Wengerowsky_Bulla_Joshi_Steinlechner_Ursin_2018_CW}
in which the entanglement is generated due to an ambiguity of the origin of the photons.
There are also many linear, or single pass, designs such as \cite{Lee_Kim_Cha_Moon_2016,Kwiat_Mattle_Weinfurter_Zeilinger_Sergienko_Shih_1995}
where the entanglement is a product of ambiguity of momentum conservation, as only specific cross sections of the two generated SPDC light
cones are spatially indistinguishable.
\par An important measure on whether a source is performing well is its brightness, bandwidth, and heralding \cite{Bennink_2010,Ljunggren_Tengner_Marsden_Pelton_2006}.
The brightness being a measure of how many photon pairs are being produced, bandwidth corresponds to how well defined they are in frequency,
as this is a limiting factor for certain interference measurements like Hanburry Brown and Twiss (HBT) \cite{Brown_Twiss_1954},
Hong Oh and Mandel (HOM) \cite{Hong_Ou_Mandel_1987}, and also for coupling to quantum memories,
and the heralding being the probability, when measuring two photon correlations, of finding a correlated photon when detecting the 1st one.
Brightness and heralding should be as high as possible in order to mitigate loss in fiber for fiber based networks,
reduce preprocessing load, and the bandwidth to be as narrow as needed for efficient coupling to quantum devices such as quantum memories or repeaters,
or for certain measurements as HBT and HOM, which will need to be performed for a full characterisation of the source.
We will also perform Quantum State Tomography (QST) measurements, and CHSH inequality measurements \cite{Clauser_Horne_Shimony_Holt_1969}.
\par One of the first \cite{Halder_Beveratos_Jorel_Zbinden_Simon_Scarani_Gisin_2007} experimental demonstrations of entanglement swapping by use of CW lasers has shown that it
is entirely possible to not use pulsed lasers for this purpose, but with less efficiency. Afterwards, there were very few reports
on this. Somewhat recently, two \cite{Samara_Maring_Martin_Raja_Kippenberg_Zbinden_Thew_2021,Tsujimoto_Tanaka_Iwasaki_Ikuta_Miki_Yamashita_Terai_Yamamoto_Koashi_Imoto_2018}
interesting papers came out showing entanglement swapping using a micro-ring resonator and a PPLN waveguide. In the case of the micro-ring resonator, four-wave mixing
was used in order to generate the entangled pairs in the two source setups, while in the case of the PPLN waveguide a single laser was used for pumping both SPDC crystals.
To the best of my knowledge, we will be the first to try to show entanglement swapping by two completely independent bulk sources, using completely independent measurement
and analysis tools.
% NOTE: What you want to do, how to do it, why important and how different from others
\par We want to build a new high-yielding source of entangled photons which can later be used as part of a research network
for conducting multiple experiments simultaneously. It should be broad enough to supply these demands, and bright enough
that further filtering does not diminish the signal to an unusable amount. This will become important for future quantum
networks as well, as they may require narrower bandwidths for efficient coupling to quantum memories or quantum repeaters.
\par The main reason for using a CW laser is due to our involvement in a EuroQCI project called the Slovenian Quantum Communication Infrastructure Demonstration (SiQUID).
The source will be used also for various QKD protocols based on entanglement. For this purpose, a CW laser was chosen.


\section{Statement of hypotheses, research questions and research goals}
% (approximately one page)
% First, briefly recall the research problem.
% Next, clearly present the principal hypotheses (H) and/or research questions (R) and/or detailed research objectives (C).
% In most cases, just one of these three categories will be relevant. Clearly state which category it is; list the individual hypotheses (H1, H2, ..),
% research questions (R1, R2, ..) and/or detailed objectives (C1, C2, ..); and explain them briefly.
% In most PhD projects, the options of defining hypotheses or setting research questions are recommended.
% However, the possibility of stating detailed research objectives is permitted under the new University rules.
% Doctoral research projects are usually based on around three main research hypotheses,
% research questions or more detailed objectives, but this depends on the nature of the research.


The overarching goal of this thesis is to develop a high-yield,
broadband source of entangled photons suitable for quantum networks and laboratory-based quantum research.
Future quantum networks must be capable of supplying entangled photon pairs to multiple users with high fidelity and efficiency.
This requires a source that is not only bright and stable but also spectrally broad enough to support multiple experiments and network nodes.
The main gaols (C) and questions (R) of this thesis are listed as follows:
% The proposed research will address key challenges in entanglement distribution,
% including entanglement swapping, entanglement distillation, and interfacing with quantum memories.

% TODO: 2nd quantum revolution bla bla
% NOTE: Add points from Rainers' new proposal (1 and 4) to this part and 
% NOTE: Add stuff about entanglement swapping, entanglement distillation, coupling to QM

C1 - Develop a broadband entanglement source that generates high-fidelity Bell states with high tangle.
The source should be bright enough to support multiple concurrent experiments and users in a research setting.

C2 - Demonstrate quantum teleportation and entanglement swapping between the entanglement source and quantum nodes at FMF and JSI,
contributing to the development of real-world quantum network architectures.

% C3 - Ensure long-term stability of the source by implementing additional filtering of the
% DWDM outputs and employing atomic spectroscopy-based locking mechanisms with Doppler broadening compensation.

% NOTE: Need funding, very short range (JSI), 
% Add background information JSI Cesium memory, no way to for my source, mention Fatis here because her photons will be 
C3 - Investigate the feasibility of implementing entanglement swapping over a short free-space link without relying on orbital angular momentum (OAM) modes.
This could serve as a valuable test-bed for alternative methods of entanglement distribution within Ljubljana.

R1 - Can this entanglement source achieve performance surpassing the current state of the art relating to CW entanglement swapping using existing technology?

R2 - How large must the cavities be in order to reduce the bandwidth of the SPDC photons for the HOM interference measurement?

%  Rephrase can into under which condition/how/why this DWDM can be utilized -- TODO: Just rephrase it into paragraphs, not bullet points
R3 - Under which conditions can Dense Wavelength Division Multiplexing (DWDM) channels, each with a 100 GHz bandwidth,
be utilized to perform quantum optical measurements such as HOM interference, and after filtering what would be the
new bandwidth by performing a HBT experiment?

These objectives and research questions define the scope of this thesis,
guiding the experimental work and theoretical analysis required to advance the state of broadband entanglement sources for quantum communication and networking.
% NOTE: Outlook for future with free space stuff
% \newpage
\section{Outline of research and research methods}
% (approximately one page)
% • Briefly outline the research and the planned research methodology.
% • Provide a more detailed outline of the planned research and its methodology;
% for example, approaches, methods (e.g., theoretical, experimental, simulation), phases of research, if applicable, also the planned scheduling of the phases.

% NOTE: Why use CW instead of Pulsed Laser (PL)?
The focus of this thesis is to implement a Sagnac interferometer source of polarization-entangled photons centered around 1560 nm,
designed to be sufficiently broadband to accommodate multiple DWDM frequency channels.
In the case of the current thesis I will use a 50 mm Periodically polled Lithium Niobate (PPLN)
Type-0 SPDC ($e_{pump} \rightarrow e_{signal} + e_{idler}$,
e meaning extraordinary polarization) crystal placed in a Sagnac interferometer
which will be bi-directionally pumped by a CW 780.24 nm laser.
\par The pump will be set to a diagonal polarization state ($\frac{1}{\sqrt{2}}(\ket{H_{pump}} + \ket{V_{pump}})$). On arriving to the
Polarization Beam-Splitters (PBS) the beam is split into two. The reflected ($\ket{V_{pump}}$) beam first passes through a half-wave plate (HWP) in order to rotate the
polarization from $\ket{V_{pump}}$ to $\ket{H_{pump}}$, as is required by phase matching conditions (in the actual setup the crystal is rotated by 90° along the beams axis), then through
the crystal where it generates two $\ket{H_{signal\ (idler)}}$ photons around 1560 nm, and then through the PBSs $\ket{H}$ output where it gets recombined
with the now two $\ket{V_{signal\ (idler)}}$ photons from the counter propagating branch.

After the two bi-photon pairs pass through the PBS they are reflected by a dichroic mirror, and diverted into a collimating lens
after which they are finally coupled into fiber. The photons from opposing directions will then be in a
$\Phi = \frac{1}{\sqrt{2}}\left( \ket{H_{signal}H_{idler}} + e^{i \phi}\ket{V_{signal}V_{idler}}\right)$ Bell state.
In order to choose any of the four available Bell states (including the two complex ones), in the pump beam path, we will have
a relative phase setter between the two counter propagating paths, consisting of a two quarter-wave plates (QWP), and a HWP between them.
The QWPs are set to $\frac{\pi}{4}$, and the HWP is used to set the appropriate phase to select one of the $\Phi$ Bell states.
\parTo maximize the efficiency of these telecom networks for multiple users, the available bandwidth (approximately 7400 GHz, or 60 nm)
is divided into many frequency channels using a DWDM.

For all of the tests a DWDM of roughly 100 GHz channel bandwidth, or 0.81 nm, will be used. For simply checking whether the source
produces entangled pairs it is enough to do a CHSH measurements. We have chosen to do this in the linear basis as it requires
only two HWPs and two PBSs. The measurement basis for this are on each channel are offset by $\frac{\pi}{8}$. After this
we will perform a QST measurement to reconstruct \cite{James_Kwiat_Munro_White_2001} the density matrix of the entangled state.
In order to measure the actual bandwidth of our DWDM channels a HBT measurement will be performed, and also a HOM measurement.
Subsequently, quantum teleportation \cite{Bouwmeester_Pan_Mattle_Eibl_Weinfurter_Zeilinger_1997}
and entanglement swapping \cite{Jennewein_Weihs_Pan_Zeilinger_2001} experiments will be conducted in collaboration with the Jožef Stefan Institute JSI
entanglement lab, who will develop an identical entanglement source.

To ensure stability, the pump laser wavelength will be locked to an absorption line in Rubidium gas using atomic spectroscopy
via the $^{87}Rb$ $D_2$ transition \cite{metger2017sas}. A small percentage of the pump beam will be diverted into a separate
setup where it will be further split into counter two propagating beams passing through a gas cell for cancelling Doppler broadening,
and a reference beam going straight through the gas cell. The reference beam and one of the counter propagating signal beams will be
coupled to a balanced photodiode, and the error signal will be fed into the laser controller for locking.

% TODO/DONE More about the synchronization
In order for two distant parties to be able to measure the correct time tags for analysis, each party will have to synchronize
to a reference. Options are currently being explored on how to exactly do this. A proposed idea is to use GPS clocks which are disciplined
regularly.

% TODO: Not going to write about this but I will do it: Free space stuff 852 nm laser 
Regarding the short distance free space test, we will likely engineer or buy a basic telescope sender and receiver station.
We will follow the procedure developed here by \cite{Gross_Jena}. Upon completion, I will test entanglement swapping via free space
link. This setup will be used primarily for another project of the group, where 1560 nm photons will be generated in a non-degenerate SPDC process
along with 853 nm photons in order to couple to a quantum memory device built by JSI. For this purpose, they must teleport the entanglement from a local 853 nm photon
generated by our group to the IJS group, and in order to do so we need the 1560 nm free space link continuous wave swapping protocol completed.

The last part of the thesis will be about active polarization control % via a closed loop algorithm such as neural networks.
\cite{CCSHDCDRS}. In order to measure the correct states, the
idea is to use an electronic polarization controller in the experimental network to create an algorithm which will be able to
ensure the correct polarization state is being received on the measurement stage.
% As a part of the SiQUID project, this knowledge will provided
% to the industrial partners, and the research regarding this will be published

All of the measurements for this thesis will be performed using Superconducting Nanowire Single Photon Detectors (SNSPDs) ID281 from IDQuantique. The
measurements at JSI will be performed using SNSPDs from Single Quantum.


\section{Expected results and original contributions to science}
% (approximately one side)
% • State the expected results of the research.
% • Clearly identify the expected original contributions to science deriving from the above results.
% (It is recommended to enumerate these contributions as an itemised list of usually up to five contributions.)
% • Briefly elaborate on the expected contributions. Since they are typically hypothetical at this stage, the contributions may be expressed in very broad or narrow terms, as appropriate.
% Original contributions include, e.g., original knowledge or findings, new theoretical models, original experimental or simulation methods, new approaches, newly opened areas, etc.
% Original contributions do not include, e.g., implementation of known methods, solutions of practical problems with known methods and similar.
% Developing and demonstrating methods of postprocessing for entanglement swapping from CW polarization entangled to be used in existing telecom infrastructure.
Lastly, I will talk about the expected entanglement swapping rate that we hope to achieve, and also the HOM interference visibility, which
is an important measurement for achieving entanglement swapping. Currently, without any optimizations in regards to heralding
or brightness the source is capable of producing around 35000 correlated pairs per second, per branch of the Sagnac loop.
Using this, and the estimated coincidence window of 300 ps, a rate of around 0.3 four-folds per second should be achievable.
This would mean that in order to gather enough four-folds for a HOM measurement, one would have to integrate for roughly 10 minutes per point.
To the best of my knowledge, this would be a new record for CW HOM, and it might open other possibilities for further research.
I have assumed that the jitter profile of the detectors and time tagger is Gaussian.

Assuming 100 GHz bandwidth of the SPDC photons, which corresponds to roughly 10 ps coherence time, and assuming the total timing jitter is
44 ps by using the Root Sum of Variances method,
the maximum visibility of the HOM interference measurement one can hope for is around 20\%, which is well below the minimal value for violating
Bells inequality.
This then leads to a rough value for minimum of the coherence time of our SPDC photons to around 100 ps for the HOM measurement, possibly even larger.
Hypothetically, assuming such a total timing jitter without including synchronization jitter we can expect to get a visibility of at most 91\%. Clearly,
adding in synchronization will reduce this number even farther for that coherence time.

In order for the integration to not take too long, but still give desirable results, there might need to be a compromise
between the bandwidth and integration time for the HOM measurement.
Filtering will of course also reduce the amount of coincidences we will be able to see.

Depending on visibility of the HOM this reduces the tangle of the source and whatnot
Check when visibility destroys entanglement

Original contributions: More engineering - filtering, jitter,
Getting good SNR with CW -> In future maybe go to PL - working principle may be the same and might bring great imrpovement of HOM

Currently making good progress with brightness - but needs improvement of coupling/heralding

Working together with a company (mention maybe the experimental network from SiQuid)

% NOTE: What you want to do, how to do it, why important and how different from others
\section{Draft plan for management of research data}
% (Note the data management plan is a required part of the "dispozicija" from 1/10/2021.)
% (approximately half a page)
% • Outline a preliminary plan for the management of any research data that will be obtained or created during the doctoral project.
% • Declare in which data repositories research data is expected to be made available, how the data will be organised,
% etc. (* For more information, see Article 50 of the Rules on Doctoral Study at the University of Ljubljana since 2020, copied below,
% as well as other relevant published instructions of the University of Ljubljana.).
During my doctoral research, I will collect and analyze time tagger data,
which records photon detection events with precise timestamps.
The data will be stored in CSV files and analyzed using C++ and Python scripts.

To ensure data integrity and reproducibility, I will organize my research data as follows:
\begin{itemize}
	\item Raw data (time tagger outputs) will be stored in a structured directory on one of our laboratory computers, sorted by experiment date and parameters.
	\item Processed data (results of filtering, calibration, and analysis) will be saved in separate CSV files, maintaining a clear relationship with the raw data.
	\item Analysis scripts (C++ and Python code) will be version-controlled using Git.
\end{itemize}
For long-term storage and accessibility, I plan to deposit my research data in an appropriate open-access data repository,
such as Zenodo, Figshare, or the University of Ljubljana Repository. The dataset will include:
Raw and processed CSV data.
Metadata describing the experiment setup, parameters, and conditions.
Documentation explaining the data structure and how to reproduce the results using the provided scripts.
The data will be made available upon request unless confidentiality or ethical restrictions apply.
When sharing, I will ensure compliance with FAIR principles (Findability, Accessibility,
Interoperability, and Reusability) by providing proper documentation and referencing my datasets in publications.
\newpage

% In the current state of the field there exist various sources which
% The thesis goal is to produce a high-yield broadband source of entanglement for use in
% quantum networks, and to also be used in a research network. There is a great demand for this in order for future
% quantum networks to be able to supply multiple users.
% This is a repetition of it self: The source is supposed to be broadband enough to facilitate many experiments and also possibly supply a large amount of users with roughly the same efficiency.

% H2 - Be able to perform various quantum tests on specific combinations of DWDM channels and get satisfying results for the bandwidth
% which we can produce.

% C1 - Build an entanglement source which would be bright enough to supply entangled photons with high
% fidelity to Bell States and high tangle, and also broadband enough to supply multiple experiments and users in the lab.

% R1 - Can we obtain better results than the current state of the art with current technologies?

% R2 - Is it possible, by using the Dense Wavelength Division Multiplexer (DWDM) channels, which are 100 GHz bandwidth channels, to perform certain interferometric
% measurements such as HBT or HOM?


% % used for - Entangled photons have a ?mysterious? phase. Mine will be a testbed for checking if this works.
% C2 - There may exist a possibility to also try to do entanglement swapping over a short distance free space application without using OAM modes.
% , try to do entanglement in this regime

% R3 - Do we need OAM modes?

% C3 - If needed, introduce extra filtering of the DWDM outputs, ensure long term stability of the source with locking via atomic spectroscopy with
% Doppler broadening compensation.

% C3 - Demonstrate quantum teleportation and entanglement swapping between FMF and JSI.
%
% These are the rough goals which we plan on doing for this thesis.

% Possibly put this in the future \ket{H;+;\omega_s}\ket{H;+;\omega_i}+\ket{V;-;\omega_s}\ket{V;-;\omega_i}
% Put an SLM in one branch of the Sagnac and generate also OAM modes
% Currently source creates Polarization and Frequency entanglement, adding the SLM thin film would make it a 3 for 1 source. A pulsed laser source would give
% also the possibility of time bin entanglement.
\bibliographystyle{IEEEtran}
\bibliography{reference}
\end{document}
